% pdflatex 3wme
% bibtex 3wme
\documentclass[twocolumn]{scrartcl}

\usepackage[utf8]{inputenc}
\usepackage[T1]{fontenc}
\usepackage{lmodern}
\usepackage[ngerman]{babel}
\usepackage{amsmath}

\usepackage{fancyvrb}
\usepackage{fvextra}

% \usepackage{palatino}
%
% \usepackage{newpxtext,newpxmath}
%
\usepackage[sc]{mathpazo} % or option osf
\usepackage{newpxmath}

\usepackage{natbib}
\bibliographystyle{natdin}

\usepackage{adjustbox}
\usepackage[a4paper, margin=1mm, includefoot, footskip=15pt]{geometry}

\usepackage[pdftitle={Funktionen und Koeffizienten des Waldwachstumssimulators 3wm}
, pdfauthor={Georg Kindermann}
, pdfsubject={Waldwachstum}
, pdfkeywords={Waldwachstum, Modell, Oberhöhe, Standort, Bonität, Zuwachs, Wald, Forst, 3wme, Wiener Wald Wachstums Modell}
, pdflang={de-AT-1996}
, hidelinks
, pdfpagemode=None]{hyperref}

\nonfrenchspacing
%\sloppy
\usepackage{breqn}
\usepackage{enumitem}

\title{Funktionen und Koeffizienten des Waldwachstumssimulators 3WME}
\author{Georg Kindermann}
% \date{18. Mai 2021}
\begin{document}

\maketitle
\tableofcontents

\section{Einleitung}

Die Entwicklung von Beständen wird durch deren Behandlung
beeinflusst. Das Ergebnis von prognostizierten Bestandsentwicklungen
dient als Entscheidungsgrundlage für die Planung im
Forstbetrieb. Solange lediglich die Wachstumskomponente der
Bestandsentwicklung betrachtet wird, werden traditionell Ertragstafeln
verwendet. Ertragstafeln können dieser Aufgabe nur bedingt gerecht
werden, da in der Regel nur wenige Behandlungsvarianten zur Auswahl
stehen. Sollen stattdessen Waldwachstumsmodelle verwendet werden,
zeigt sich gelegentlich, dass nicht alle benötigten Daten vorhanden
sind bzw. nur mit erheblich höherem Aufwand aufgenommen werden können.

Die Messung der Baumhöhe war in der Vergangenheit relativ aufwändig
und es wurden daher Baumhöhen mittels Höhenkurven rechnerisch
ergänzt. Heutzutage kann die Höhenmessung mittels Fernerkundung mit
geringem Aufwand flächendeckend durchgeführt werden.
Standortsparameter wie bodenchemische und -physikalische
Eigenschaften, Hangneigung, Exposition, Temperatur oder Niederschlag
können aus relativ hochauflösenden Karten abgelesen werden. Um dieser
Entwicklung Rechnung zu tragen, wurden Wachstumsfunktionen auf Basis
der im Zuge der Österreichischen Waldinventur beobachteten
Einzelbaumzuwächse sowie Verwendung von flächendeckenden
Standortskarten parametrisiert.

Um diese Neuentwicklung zu verdeutlichen wurde für die hier
beschriebenen Waldwachstumsmodelle ein neuer Name, \textbf{W}iener
\textbf{W}ald \textbf{W}alchstums \textbf{M}odell\textbf{E}
($\mathit{3}W \atop M\mathcal{E}$), gewählt. Die Neuentwicklung
basiert auf den Erfahrungen, welche von den Modellen\emph{Caldis
Vatis}, \emph{PrognAus}, \emph{g4m}, \emph{MOSES} und deren
Vorgängern gewonnen wurden.

Diese Modelle wurden im \emph{Managementplan Forst}, dem
Internetangebot der Landwirtschaftskammer für alle WaldbesitzerInnen
in Österreich, (\url{www.lko.at/forstprogramme}) umgesetzt.

\section{Daten}

Datengrundlage bilden die Aufnahmen der Österreichischen Waldinventur
\citep{hauk2009waldinventur} in den Jahren 1981--2018. Dabei wurden
BHD, H, Kronenansatz, Nutzungsart, Baumart, geogr.\ Lage und
Aufnahmezeitpunkt verwendet.

Als Wetterdaten wurden Monatsmitteltemperatur~[$^{\circ}$C],
Monatsniederschlag~[mm/30~Tage] und Monatsmittel der
Globalstrahlung~[W] von den Stationen der Zentralanstalt für
Meteorologie und Geodynamik und den Niederschlagswerten der Stationen
des Hydrographisches Zentralbüros verwendet. Diese Wetterdaten wurden
nach \cite{kindermann2010jahrringanalyse} auf den jeweiligen
Waldinventurpunkt in Abhängigkeit von dessen Seehöhe und
geographischer Lage interpoliert.

Von \cite{geoland2021dgm} wurde das digitale Geländemodell mit einer
Auflösung von 10\,m x 10\,m verwendet. Von \cite{geologie2015} wurde
die geologische Karte verwendet.

Von den physikalische Bodeneigenschaften wurde die
Wasserspeicherkapazität von \cite{Ballabio2016soilPhysical} und die
Bodentiefe von \cite{Hengl2017soilgrids} entnommen. Die chemischen
Bodeneigenschaften (pH-Wert, C/N Verhältnis, N, P, K, Ca) wurden von
\cite{Ballabio2019soilChemical} entnommen.

\section{Methoden}

\subsection{Oberhöhenfächer}

Der Oberhöhenfächer wurde in Anlehnung an
\cite{kindermann2018siteIndexCurves} erstellt.
Dazu wird mit Hilfe der Oberhöhenfunktion.

%h=c0*log(1+exp(c2)*t^c3)^c1
\begin{equation}
  \label{eq:ohFun}
  h = c_0 \log(1 + e^{c_2}t^{c_3})^{c_1}
\end{equation}

wobei $h$ die \emph{Oberhöhe} im \emph{Alter} $t$ ist. Der \emph{periodische
Höhenzuwachs} $dh$ aus dem \emph{Zuwachszeitraum} $dt$ berechnet sich:

%dh=c0*log(1+exp(c2)*(t+dt)^c3)^c1–h
\begin{equation}
  \label{eq:ohFunDhT}
  dh = c_0 \log(1 + e^{c_2}(t + dt)^{c_3})^{c_1} - h
\end{equation}

Durch umformen der Oberhöhenfunktion kann das \emph{Alter} $t$ mit

%t=((exp((h/c0)^(1/c1))-1)/exp(c2))^(1/c3)
\begin{equation}
  \label{eq:ohFunT}
  t = \left(\frac{e^{\left(\frac{h}{c_0}\right)^{1/c_1}} - 1}{e^{c_2}}\right)^{1/c_3}
\end{equation}

ersetzt werden. Damit kann der Höhenzuwachs ohne das Alter beschrieben werden:

%dh=c0*log(1 + exp(c2)*((((exp((h/c0)^(1/c1)) - 1) / exp(c2)) ^ (1/c3)) + dt)^c3)^c1 – h
\begin{equation}
  \label{eq:ohFunDh}
  dh = c_0 \log\left(1 + e^{c_2}\left(\left(\frac{e^{\left(\frac{h}{c_0}\right)^{1/c_1}} - 1}{e^{c_2}}\right)^{1/c_3} + dt\right)^{c_3}\right)^{c_1} - h
\end{equation}

Die Koeffizienten $c_0$, $c_1$, $c_2$ und $c_3$ wurden mittels
nichtlinaren gemischten Modellen (nlmer vom paket lme4
(Version~1.1-26) in R (Version~4.0.4)) geschätzt. Dabei wurden
Zufallseffekte des Probepunktes bei den Koeffizienten $c_0$ und $c_2$
unterstellt. Es zeigte sich das $c_2$ mit $c_0$ mit

%c2 = cc0 + cc1 * c0 ^ cc2
\begin{equation}
  \label{eq:ohFunC2}
  c_2 = cc_0 + cc_1 * c_0^{cc_2}
\end{equation}

beschrieben werden kann. Die Koeffizienten $cc_0$, $cc_1$ und $cc_2$
wurden mittels nichtlinearer Regression bestimmt.

Da allerdings eine Veränderung des Standortes während der beobachteten
Höhenzuwächse nicht ausgeschlossen werden kann, und dies einen
Einfluss auf den Höhenzuwachsgang hätte, wurde in einem
Zwischenschritt die Oberhöhenbonität aus Standortsparmetern geschätzt
und der sich damit ergebende Trend von den Beobachtungen
herausgerechnet.

\subsection{Schätzung der Oberhöhenbonität aus Standortsparametern}

Mit dem zuvor erstellen Oberhöhenfächer wurde unter Verwendung von
Oberhöhe, Oberhöhenzuwachs und Zuwachszeitraum auf jedem Inventurpunkt
der Koeffizient $c_0$ und damit die Oberhöhenbonität bestimmt. In
einem ersten Schritt wurde dieser $c_0$ mit \emph{Temperatur} und
\emph{Niederschlag} beschrieben wobei die
\emph{Wasserspeicherkapazität} des Bodens berücksichtigt wurde. Es
wurden \emph{Monatsmitteltemperatur}~[$^{\circ}$C], \emph{Strahlung als
  Monatsmittel}~[$\text{W}/\text{m}^2$] und Summe des
\emph{Monatsniederschlages}~[mm/30~Tage] der einzelnen Jahre in denen
der Zuwachs beobachtet wurde, verwendet.
Der Koeffizient $c_0$ ergibt sich als Summe der einzelnen Monate:

% c0 = c0Jan + c0Feb + c0Mär + c0Apr + c0Mai + c0Jun + c0Jul + c0Aug + c0Sep + c0Okt + c0Nov + c0Dez
% c0Monat = cI0 * exp(-nn/7990) * Anzahl Tage in diesem Monat
%   * max(0, cI15 + Strahlung)^cI16
% * max(0, (cI1 + Temperatur))^cI2
% * max(0, 1 – Potentielle Evapotranspiration /tanh(cI5 * pflanzenverfügbares Wasser)/cI3)^cI4
% * (1. + tanh(cI14 * (tmax- tmin)))
\begin{align}
  \label{eq:c0Monat}
  c_0 &= c_{0_{\text{Jan}}} + c_{0_{\text{Feb}}} + \cdots + c_{0_{\text{Nov}}} + c_{0_{\text{Dez}}} \\
  c_{0_{\text{Monat}}} &= c_{I_0} \cdot e^{-\frac{nn}{7990}} \cdot \text{Anzahl Tage in diesem Monat} \notag \\
      &\phantom{{}= } \cdot \max(0, c_{I_{15}} + \text{Strahlung})^{c_{I_{16}}} \notag \\
      &\phantom{{}= } \cdot \max(0, c_{I_{1}} + \text{Temperatur})^{c_{I_{2}}} \notag \\
  \noalign{\hfill $\cdot \max(0, 1 - \frac{\text{Potentielle Evapotranspiration}}{\tanh(c_{I_5} \cdot \text{pflanzenverfügbares Wasser}) * cI3})^{c_{I_4}}$} \notag \\
  &\phantom{{}= } \cdot (1 + \tanh(c_{I_{14}} (t_{max}- t_{min})))
\end{align}

Wobei das Bodenwasser nach folgendem Pseudocode modelliert wurde:

%\begin{figure*}
%  \begin{Verbatim}[breaklines=true,frame=single,fontsize=\small]
  \begin{Verbatim}[breaklines=true,frame=single,fontsize=\scriptsize]
pflanzenverfügbares Wasser = Niederschlag in den Boden + verfügbares Bodenwasser
Niederschlag in den Boden = Niederschlag – (Interzeption + Oberflächenabfluss)
verfügbares Bodenwasser = Bodenwasser * (Bodenwasser / nutzbare Feldkapazität)^cI9
Interzeption = cI7
Oberflächenabfluss = max(0, Niederschlag – Interzeption – (nutzbare Feldkapazität – Bodenwasser))
Potentielle Evapotranspiration = 30. * exp(17.62 * Temperatur / (243.12 + Temperatur))
Evapotranspiration = Potentielle Evapotranspiration * cI6
Bodenwasser im nächsten Monat = min(nutzbare Feldkapazität, max(0, Bodenwasser + Niederschlag in den Boden – Evapotranspiration))
nutzbare Feldkapazität = nutzbare Feldkapazität je cm Bodentiefe * min(cI18, Bodengründigkeit)
\end{Verbatim}
%\end{figure*}

Die Koeffizienten wurden mittels nichtlinearer Regression bestimmt.

Dabei wird für Lagen oberhalb der Waldgrenze, wenn auch eine geringe,
gelegentlich aber dennoch eine Oberhöhe geschätzt. Um dieses Verhalten
abzuschwächen wird der zuvor bestimmte wert von c0 noch modifiziert
mit:

\begin{dgroup*}
  \begin{dmath}
    c_{0W} = c_0 \cdot \max(0, \min(1, \frac{\frac{1}{1 + e^x} - 0.1}{0.8}))
  \end{dmath}
  \begin{dmath*}
    x = c_{W_1} + c_{W_2} \cdot t10 + c_{W_3} \cdot \text{Schneemenge} + c_{W_4} * \cos(\text{Exposition}) \cdot \sin(\text{Hangneigung})^0.3 + c_{W_5} \cdot \sin(\text{Hangneigung}) + c_{W_6} \cdot \text{form1} + c_{W_7} \cdot \text{form2} + c_{W_8} \cdot \text{form3} + c_{W_9} \cdot 1013.25 \cdot (1 - 0.0065 * nn / 288.15)^{5.255}
  \end{dmath*}
\end{dgroup*}
  
Hier wurde das 30--jährige Mittel des Zeitraumes (1981-2010) für
Monatsmitteltemperatur und Monatsniederschlag verwendet wobei:

\begin{description}[font=\normalfont\itshape]
\item[Schneemenge:] Summe der Monatsniederschläge für Monate in denen
  die Monatsmitteltemperatur kleiner als 1°C ist.
\item[t10:] Monatsmitteltemperatur des dritt wärmsten Monats.
\item[form1:] 1 wenn es sich um einen Graben bei einem Suchradius von
  210\,m handelt, sonst 0.
\item[form2:] 1 wenn es sich um ein Tal bei einem Suchradius von 3\,km
  handelt, sonst 0.
\item[form3:] 1 wenn es sich um einen Gipfel bei einem Suchradius von
  3\,km handelt, sonst 0.
\end{description}

Im nächsten Schritt wird der Koeffizient mit Bodenparametern modifiziert.

\begin{dgroup*}
  \begin{dmath}
    c_{0_{WB}} = c_{0_W} \frac{2}{1 + e^{-x}}
  \end{dmath}
  \begin{dmath*}
    x = so_{\text{Intercept}} + so_{\text{CN}} \cdot CN + so_N \cdot N + so_P \cdot P + so_{pH} \cdot pH + so_{pH2} \cdot pH^2 + so_{P2} \cdot P^2
  \end{dmath*}
\end{dgroup*}

Wobei \emph{CN} das C/N Verhältnis, \emph{N} der Bodenstickstoff,
\emph{P} der Phoshorgehalt und \emph{pH} der Boden pH--Wert in CaCl
darstellen.

Im nächsten Schritt wird der Koeffizient mittels Hangneigung und
Exposition modifiziert.

\begin{dgroup*}
  \begin{dmath}
    c_{0_{WBH}} = c_{0_{WB}} \frac{2}{1 + e^{-x}}
  \end{dmath}
  \begin{dmath*}
    x = ha_{\text{Intercept}} + ha_{si} \cdot si + ha_{co} \cdot co
    + ha_{\text{Hangneig}} \cdot \text{Hangneigung}
    + ha_{si2} \cdot si^2 + ha_{co2} * co^2
  \end{dmath*}
  \begin{dmath*}
    si = \sin(\text{Exposition}) \cdot \text{Hangneigung}/90
  \end{dmath*}
  \begin{dmath*}
    co = \cos(\text{Exposition}) \cdot \text{Hangneigung}/90
  \end{dmath*}
\end{dgroup*}

Es zeigte sich, dass die Bestandesdichte die Oberhöhenbonität beeinflusst.

\begin{dgroup}
  \begin{dmath}
    c_{0_{WBHD}} = c_{0_{WBH}} \cdot \frac{2}{1 + e^{- bd_{\text{Intercept}} - bd_b \cdot b}}
  \end{dmath}
  \begin{dmath*}
    b = pmin(1, ggha / g_{max})
  \end{dmath*}
  \begin{dmath}
    gmax = bd_{c_0} * (1 - \frac{1}{1+h*bd_{c_1}})
  \end{dmath}
\end{dgroup}

Hierbei ist h die Höhe in [m], ggha die Grundfläche in $m^2/ha$ wobei
allerdings bei Bäumen bis 1.3\,m Höhe deren Höhe in m und bei Bäumen
über 1.3\,m deren BHD in cm um 1.3 erhöht, verwendet wird.

Auch der Grundflächenanteil, ebenfalls mit dem um 1.3\,cm erhöhtem
BHD, der beigemischten anderen Baumarten hatte einen Einfluss auf die
Bonität.

\begin{dmath}
  c_{0_{WBHDA}} = c_{0_{WBHD}} \cdot \frac{2}{1 + e^{- \sum_{ba} \text{baCoef}_{ba} \cdot \text{GAnteil}_{ba} }}
\end{dmath}

Wobei die Summe der Anteile der Baumarten 1 ergibt.

\subsection{Höhenzuwachs}

Bei bekanntem $c_0$ bzw.\ bekannter Oberhöhenbonität unter Verwendung
eines Oberhöhenfächer, ist der potentiell mögliche Höhenzuwachs
vorgegeben. Aufgabe des Höhenzuwachsmodells ist es, diesen aufgrund
von Konkurrenz gegebenenfalls zu reduzieren.

\begin{dgroup}
  \begin{dmath}
    ih/ih_{Pot} = \frac{1}{1 + e^{-x \pm y}}
  \end{dmath}
  \begin{dmath}
    x = ch_0 + ch_1 \dot \ln(1 + gh) + ch_2 \dot h/c_0 + ch_3 \dot h/(1.3 + d) + ch_4 \dot (h/(1.3 + d))^2 + \sum(ch_{5Sp} \dot sqrt(hh_{Sp}))
  \end{dmath}
  \begin{dmath}
    y = ch_6 + ch_7 * abs(x)
  \end{dmath}
  \begin{dmath}
    gh = \sum_{i=0}^{Nachbarn} \frac{max(0, h_i - h)}{h_i} \cdot \frac{(1.3 + d_i)^2 \cdot \pi}{40000} \cdot nRep_i \cdot chGh_{\text{Baumart}_i}
  \end{dmath}
  \begin{dmath}
    hh_{Sp} = \sum_{i=0}^{\text{Nachbarn der Baumartengruppe} Sp} max(0, h_i \cdot chHh_{\text{Baumart}_i} - h)  \cdot nRep_i
  \end{dmath}
\end{dgroup}

Wobei $ih$ der Höhenzuwachs, $ih_{Pot}$ der pottentielle Höhenzuwachs,
welche aus dem Höhenfächer abgelesen werden kann, $ch_?$
Koeffizienten, $gh$ die nach Baumarten gewichtete
($chGh_{\text{Baumart}}$) Grundfläche der Nachbarn in Wipflehöhe des
betrachteten Baumes in $\text{m}^2\text{/ha}$ wobei die Kreisfläche
als linear mit der Höhe abnehmend angenommen wird, $h$ die
Ausgansbaumhöhe in m, $d$ der BHD in cm und $hh_{Sp}$ die nach
Baumartengruppen differenzierte Summe der überragenden Kronenlängen
der Konkurrenten ist. Mittels dem Faktor $y$ kann die
Standardabweichugn des Schätzwertes bestimmt werden.

\subsection{Durchmesserzuwachs}

Der Durchmesserzuwachs wird nicht direkt geschätzt sondern aus der
Volumsleistung je Hektar rückgerechnet. Dabei wird der Einfluss von
Höhenzuwachs und Formveränderung derzeit noch nicht berücksichtigt. In
einem ersten Schritt wird die Bestandesfläche auf die einzelnen Bäume
aufgeteilt. Dazu wird derzeit proportional zur, nach Baumarten
gewichteten, Kreisfläche aufgeteilt. Weitere Möglichkeiten zur
Bestimmung der Standfläche sind in
\cite{kindermann2019einzelbaumBestandesdichte} angeführt.

\begin{dgroup}
  \begin{dmath}
    \text{Standfläche0} = 10000 \cdot \frac{d^{cd} \cdot cdw{\text{Baumart}}}{\sum_{i=0}^{Konkurrenten} d_i^{cd} \cdot cdw{\text{Baumart}_i} \cdot nrep_i}
  \end{dmath}
  \begin{dmath}
     \label{eq:bhd2d}
    d =
    \begin{cases}
      bhd + 1.3 & \text{wenn } h\geq 1.3 m\\
      h         & \text{wenn } h < 1.3 m\\
    \end{cases}
  \end{dmath}
  \begin{dmath}
    \text{Standfläche} =
    \begin{cases}
      \text{Standfläche0} & \text{wenn } \text{Standfläche0} < h^2 \pi\\
       h^2 \pi            & \text{wenn } \text{Standfläche0} > h^2 \pi\\
     \end{cases}
  \end{dmath}
\end{dgroup}

Dabei ist zu berücksichtigen das bei $nrep > 1$ der betrachtete Baum
als Konkurrent mit $nrep - 1$ eingeht. Es wird eine maximale mögliche
Standflächengröße mit $h^2 \pi$ angenommen. Ein, dem geleisteten
Volumszuwachs je Hektar proportionaler Wert $ivHaProp$, wird mittels
Gleichung~\ref{eq:ivHaProp} bestimmt wobei auch hier $d$ nach
Gleichung~\ref{eq:bhd2d} verwendet wurde.

\begin{dgroup}
  \begin{dmath}
    \label{eq:ivHaProp}
    ivHaProp = cd_0
    + cd_1 \cdot log(d / \text{Standfläche})
    + cd_2 \cdot log(c_0)
    + cd_3 \cdot log(1 + h/d)
    + cd_4 \cdot h/d
    + cd_5 \cdot log(1 + d)
    + \sum_{i=0}^{\text{Baumart}} cd_{6_i} \cdot cba_i
    + \sum_{i=0}^{\text{Baumart}} cd_{7_i} \cdot hh_i
  \end{dmath}
  \begin{dmath}
    \label{eq:idCba}
    cba_{\text{Baumart}} = \sum_{i=0}^{Konkurrenten der Baumart} max(0, d_i^2 - (cd_{cba} * d)^2) * nrep_i
  \end{dmath}
  \begin{dmath}
    \label{eq:idHa}
     hh_{\text{Baumart}} = \sum_{i=0}^{Konkurrenten der Baumart} max(0, h_i - cd_{hh} * h) * nrep_i
  \end{dmath}
\end{dgroup}

Als Konkurrenzindices wurde zum einen die konkurrenzierende
Kreisfläche der Nachbarn je Baumartengruppe (Gleichung~\ref{eq:idCba})
sowie Konkurrenz aufgrund von Höhendifferenzen
(Gleichung~\ref{eq:idHa}) verwendet. Für den Schätzwert $ivHaProp$
wird eine Obergrenze definiert (Gleichung~\ref{eq:ivHaPropMax}) sowie
dessen Streuung mit Gleichung~\ref{eq:ivHaPropSL},
\ref{eq:ivHaPropSH}) bestimmt.

\begin{dgroup}
  \begin{dmath}
    \label{eq:ivHaPropMax}
    ivHaPropMax =  c_0 *  cd_{\text{ivMax}}
  \end{dmath}
  \begin{dmath}
    \label{eq:ivHaPropSL}
    ivHaProp_L = ivHaProp - exp(cd_{sl_0} + cd_{sl_1} * ivHaProp)
  \end{dmath}
  \begin{dmath}
    \label{eq:ivHaPropSH}
    ivHaProp_H = ivHaProp + 1/(1+exp(cd_{sh_0} + cd_{sh_1} * ivHaProp))
  \end{dmath}
\end{dgroup}

Die Umrechnung dieses Schätzwertes $ivHaProp$ in einen
Durchmesserzuwachs $id$ erfolgt mit Gleichung~\ref{eq:iv2id}.
\begin{dmath}
  \label{eq:iv2id}
  id = \sqrt{d^2 + \frac{e^{ivHaProp} \cdot \text{Standfläche}}{h} \cdot 4 / \pi} - d
\end{dmath}

\subsection{Durchmesser aus Baumhöhe}

Bei Baumhöhenmessungen aus Luftbilder oder mittels Airborne
Laserscanning wird meist der BHD nicht erfasst. Es wurde der BHD~[cm]
in Abhängigkeit von der Eigentumsart, der Höhe $h$~[m], der
Hangneigung (0 bei Hangneigungen bis, 1 bei über $25^{\circ}$), der
Seehöhe $nn$~[m] und der aktuellen Grundflächenhaltung~[$m^2/ha$]
beschrieben.

\begin{dgroup}
  \begin{dmath}
    \label{eq:dEsti}
    bhd = (h - 1.3)/e^z
  \end{dmath}
  \begin{dmath}
    z = c_{d_{\text{Eigentumsart}}} + c_{d_1} \cdot h + c_{d_2} \cdot log(h) + c_{d_3} \cdot (hangneigung > 25) + c_{d_4} \cdot nn + c_{d_5} \cdot  gha
  \end{dmath}
\end{dgroup}


\subsection{Ergänzung des Kronenansatzes}

Die Kronenlänge bzw.\ die Höhe des Kronenansatzes wird durch die
Lichtkonkurrenz der Nachbarbäume, die bisher über die \emph{gesamte}
Lebensspanne des Baumes geherrscht hat, dessen Baumart und künstliche
Aufastungen beeinflusst. Um das Modell einfach zu halten wurden
lediglich die Höhe und der der BHD verwendet.

\begin{dmath}
  \label{eq:hka}
  hka = h \cdot (1 - e^{chka_0 + chka_1 \cdot h/bhd + chka_1 \cdot d})
\end{dmath}

\subsection{Zufallsnutzung und Bruchereignisse}

Zufallsereignisse werden durch vielfältige Faktoren beeinflusst. Diese
sind oft schwer erfassbar und erstrecken sich auch auf die umliegenden
Bestände (Hiebszug) sowie die Topographie. Hier werden einfach zu
erhebende Baum und Bestandesmerkmale verwendet und eine
durchschnittliche Schadwahrscheinlichkeit bestimmt, die im
Einzelbestand deutlich abweichen kann.

\subsubsection{Zufallsnutzung}

In Gleichung~\ref{eq:pNutz}
wird die Wahrscheinlichkeit, dass es eine Zufallsnutzung in diesem
Jahr auf einem WZP--Punkt gibt, geschätzt. Dabei haben die
\emph{Geologie}, die Spitzenhöhe über alle Baumarten $h_{max}$ und je
Baumartengruppe $h_{max_{Ba}}$, der maximale H/D-Wert über alle
Baumarten $max(h/d)$ und je Baumartengruppe $max(h/d)_{Ba}$ sowie die
Grundfläche $gha$ einen Einfluss.

\begin{dgroup*}
  \begin{dmath}
    \label{eq:pNutz}
    p_{Zfn_{Punkt}} = \frac{1}{1 + e^-z}
  \end{dmath}
  \begin{dmath*}
    z = czp_{Geologie}
    + czp_1 \cdot h_{max}
    + \sum_{Ba} (czp_{2_{Ba}} \cdot h_{max_{Ba}})
    + czp_3 \cdot max(h/d)
    + czp_4 \cdot max(h/d)^2
    + \sum_{Ba} (czp_{5_{Ba}} \cdot max(h/d)_{Ba})
    + czp_6 \cdot \ln(1 + gha)
  \end{dmath*}
\end{dgroup*}

Die Wahscheinlichkeit, dass auf einem Punkt mit Zufallsnutzung, alle
Bäume genutzt werden, errechnet sich mit Gleichung~\ref{eq:pNutzAlle}
wobei $h_{max}$ die maximale Baumhöhe~[m], $gha$ die
Grundfläche~[$m^2/ha$] und $h_l$ die loreysche Mittelhöhe~[m] ist.

\begin{dgroup*}
  \begin{dmath}
    \label{eq:pNutzAlle}
    p_{Zfn_{Alle}} = \frac{1}{1 + e^-z}
  \end{dmath}
  \begin{dmath*}
    z = czpa_{Geologie}
    + czpa_1 \cdot \ln(1+ h_{max})
    + czpa_2 \cdot \ln(1 + gha)
    + czpa_3 \cdot h_l
  \end{dmath*}
\end{dgroup*}

Die Nutzungswarscheinlichkeit des Einzelbaumes, auf einem Punkt mit
Zufallsnutzung, auf dem nicht alle Bäume genutzt wurden, errechnet mit
Gleichung~\ref{eq:pNutzBaum} wobei $h$ die Baumhöhe~[m], $gha$ die
Grundfläche~[$m^2/ha$] und $nn$ die Seehöhe~[m] ist.

\begin{dgroup*}
  \begin{dmath}
    \label{eq:pNutzBaum}
    p_{Zfn_{Baum}} = \frac{1}{1 + e^-z}
  \end{dmath}
  \begin{dmath*}
    z = czpb_{Geologie}
    + czpb_1 \cdot h
    + czpb_2 \cdot gha
    + czpb_3 \cdot nn
  \end{dmath*}
\end{dgroup*}


\subsubsection{Bruchereignisse}

In Gleichung~\ref{eq:pBruch} wird die Wahrscheinlichkeit, dass es ein
Bruchereigniss in diesem Jahr auf einem WZP--Punkt gibt,
geschätzt. Dabei haben die \emph{Geologie}, die maximale Höhe
$h_{max}$, der maximale H/D-Wert *max(h/d)* [1] die Grundfläche $gha$
[$m^2/ha$], die loreysche Mittelhöhe $h_l$ [m] (mit der Kreisfläche
gewichtete Höhe), der Durchmesser des Grundflächenmittelstamm $d_g$
[cm], die Seehöhe [m] und die Hangneigung [$^\circ$] einen Einfluss.

\begin{dgroup*}
  \begin{dmath}
    \label{eq:pBruch}
    p_{Bruch_{Punkt}} = \frac{1}{1 + e^-z}
  \end{dmath}
  \begin{dmath*}
    z = cbp_{Geologie}
    + cbp_1 \cdot h_{max}
    + cbp_2 \cdot max(h/d)
    + cbp_3 \cdot \l(1 + gha)
    + cbp_4 \cdot h_l
    + cbp_5 \cdot d_g
    + cbp_6 \cdot \ln(100 + nn)
    + cbp_7 \cdot nn
    + cbp_8 \cdot \text{Hangneigung}
  \end{dmath*}
\end{dgroup*}

Die Wahrscheinlichkeit, dass auf einem Punkt mit Bruchereignissen alle
Bäume gebrochen sind, errechnet sich mit Gleichung~\ref{eq:pBruchAlle}
wobei $gha$ die Grundfläche [$m^2/ha$] ist.

\begin{dgroup*}
  \begin{dmath}
    \label{eq:pBruchAlle}
    p_{Bruch_{Alle}} = \frac{1}{1 + e^-z}
  \end{dmath}
  \begin{dmath*}
    z = cbpa_0
    + cbpa_1 \cdot \ln(1 + gha)
  \end{dmath*}
\end{dgroup*}

Die Bruchwarscheinlichkeit des Einzelbaumes, auf einem Punkt mit Bruch
bei dem nicht alle Bäume gebrochen sind, errechnet sich mit
Gleichung~\ref{eq:pBruchBaum} wobei die \emph{Geologie}, die Baumhöhe
$h$ [m], das H/D-Verhältniss [1], die Grundfläche $gha$ [$m^2/ha$],
der SDI (Stand density index
$SDI = N \cdot (\frac{d_g}{25})^{-1.605}$) [$N_{25}/ha$] und die
Seehöhe $nn$ [m] verwendet werden.

\begin{dgroup*}
  \begin{dmath}
    \label{eq:pBruchBaum}
    p_{Bruch_{Baum}} = \frac{1}{1 + e^-z}
  \end{dmath}
  \begin{dmath*}
    z = czpb_{Geologie}
    + cbpb_1 \cdot h
    + cbpb_2 \cdot h/d
    + cbpb_3 \cdot gha
    + cbpb_4 \cdot sdi
    + cbpb_5 \cdot nn
  \end{dmath*}
\end{dgroup*}

\subsection{Erstellen einer Ertragstafel}

Mit den vorgestellten Funktionen lässt sich ein Waldwachstumsmodell,
welches die Entwicklung von Höhe und BHD des Einzelstammes sowie
dessen verbleib im Bestand beschreibt, erstellen. Wobei das
Ausscheiden eines Baumes durch Eingriffe, Zufallsnutzungen und
Mortalität aufgrund von Konkurrenz erfolgen kann. Eine Methode zur
Überprüfung des Modells stellt die Erstellung einer Ertragstafel
dar. Um die weiteren Möglichkeiten aufzuzeigen wurden die einzelnen
Stämme mit Hilfe der Einzelstamm--Sortentafel auf Sortimente
aufgeteilt und auch die Biomasse bestimmt.

Bei den Zuwachsfunktionen wurden deren Zufallsstreuung angegeben die,
bei der Anwendung hier, zufällig über den Schätzwert gelegt
wurden. Zusätzlich wurde noch angenommen das der Standort und die
einzelnen Bäume nicht vollkommen identisch sind. Diese Heterogenität
wurde durch eine zufällige aber über den Simmulationszeitraum konstant
gehaltenen Streuung der Oberhöhenbonität von $\pm 1.5$\,m Rechnung
getragen.

Für die Simulation wurden Baumpositionen eines ca.\ 1\,Ha großen
Bestand erzeugt und die Konkurrenz abstandsabhängig bestimmt. Bei der
Erstdurchforstung wurde zwischen den Pflanzreihen 3.5\,m breite
Rückegassen mit einem maximalen Abstand von 20\,m angelegt. Der
Pflanzverband wurde so gewählt das die Baumverteilung, der
angestrebten Endstammzahl, einen regelmäßigen Dreiecksverband
ergibt. Auf diese Position wurde ein Baum gepflanzt. Drei weiter Bäume
wurden im Abstand von 1/3 des Endbaumabstandes um diesen Baum
gepflanzt. Damit entstand in \emph{unregelmäßiger Weitverband} mit
jeweils vier Bäumen als Gruppe. Für die erzeugten Baumposition wurde
angenommen das diese auf 30\,cm genau gepflanzt werden konnten.

Von den gepflanzten Bäumen wurde angenommen dass 20\,\% ausfallen
bevor sie eine Höhe von 1.3\,m erreicht haben. Diese Ausfälle wurden
zufällig über der Fläche verteilt unter der Annahme das horstweise
Ausfälle in Zuge von Nachbesserungen beseitigt werden
konnten. Zusätzlich wurde das Auftreten von Zufallsnutzungen, Brüchen
und konkurrenzbedingte Mortalität realisiert unter der Annahme das
diese aus dem Bestand ausscheiden.

Als Behandlung wurden drei Eingriffe vorgesehen bevor es zu erhöhter
konkurrenzbedingte Mortalität kommt. Durch Auswahl entsprechender
Anfangsstammzahlen wurden diese im Alter von 20, 45 und 115 Jahren
durchgeführt. Der erste Eingriff wurde als Hochdurchforstung
durchgeführt und reduziert die ursprünglichen vier Bäume der Gruppe
auf zwei. Beim zweiten Eingriff wurde zufällig auf einen Baum
reduziert womit die Endstammzahl erreicht wurde, solange nicht alle
Bäume der Vierergruppe von Ausfällen betroffen waren. Der letzte ein
Eingriff wurde als Niederdurchforstung, mit einer Reduktion der
Grundläche um 20\,\%, durchgeführt um das ankommen von Naturverjüngung
zu erleichtern. Alle weiteren Entnahmen, die in der Regel jensets der
üblichen Umtriebszeit liegen werden, sind zum Teil durch Bruch und
Zufallsnutzungen aber überwiegend durch Überschreiten der maximalen
Bestandesdichte bedingt und können als Plenterbewirtschaftung in der
Praxis umgesetzt werden.

Da sich sowohl die Zuwachsleistung als auch die Oberhöhenbonität durch
unterschiedliche Bewirtschaftung verändern, ist dennoch eine
Bonitätsangabe, je Tafel, nötig um einen Vergleich zwischen
verschiedenen Behandlungen zu ermöglichen. In diesem Fall wurde eine
Refferenzoberhöhe gewählt, die sich aus der Oberhöhenkurve
(Gleichung~\ref{eq:ohFun}) ergibt.


\section{Ergebnisse}

\subsection{Oberhöhenfächer}

Da eine Schätzung von $c_0$, $c_1$, $c_2$ und $c_3$ gleichzeitig mit
den vorhandenen Daten nicht möglich war wurden die Koeffizienten
$c_1=0.75$ und $c_3=3$ konstant gehalten. $c_2$ errechnet sich aus
$c_0$ mit den Koeffizienten: $cc_0 = -11.146124$, $cc_1 = 0.020238$
und $cc_2 = 1.903105$. $c_0$ ist variabel und bestimmt die Bonität.

\subsection{Schätzung der Oberhöhenbonität aus Standorsparametern}

Es ergaben sich folgende Koeffizienten für den Einfluss von
Temperatur, Strahlung und Niederschlag:

\texttt{cI0 = 6.103775e-05, cI1 = 7.002676, cI2 = 1.428587, cI3 = 1.476124e+02, cI4 = 1.066501, cI5 = 4.080917e-02, cI6 = 1.040640, cI7 = 0, cI8 = 0, cI9 = 3.553800e-06, cI10 = 5.483664, cI11 = 0, cI12 = 0, cI13 = 0, cI14 = 6.319343e-02, cI15 = -5.992448, cI16 = 0.6198672, cI17 = 5.058839e+02}

Es ergaben sich folgende Koeffizienten zur Bestimmung der oberen Waldgrenze:

\texttt{cW1 = 23.537192679, cW2 = - 0.549791635, cW3 = 0.002743781, cW4 = 0.367703576, cW5 = - 1.730203792, cW6 = 0.418663281, cW7 = 0.556252777, cW8 = 0.091375384, cW9 = 0.022929243}

Es ergaben sich folgende Koeffizienten zur Bestimmung des Bodeneisflusses:

\texttt{soIntercept = -0.5033563268, soEsdacCN = -0.0173330532, soEsdacN = 0.0242756719, soEsdacP = 0.0068670425, soEsdacpH_CaCl = 0.3640807711, soEsdacpH_CaCl2 = -0.0469397314, soEsdacP2 = -0.0001540427}

Es ergaben sich folgende Koeffizienten zur Bestimmung des Einflusses von Exposition und Hangneigung:

\texttt{haIntercept = -0.03307927, haSi = 0.10513010, haCo = 0.07524978, haHangneig = 0.00647353, haSi2 = -1.41254718, haCo2 = -1.57046515}

Die Koeffizienten zur Bestandesdichte:

\texttt{bdC0 = 160.88389592, bdC1 = 0.04631407, bdIntercept = -0.09700211, bdB = 0.17728830}

Der Koeffizient baCoef ist für die einzelnen Baumarten:

\texttt{Fichte = 0.04599804, Tanne = -0.12496514, Lärche = -0.08785919, Weißkiefer = -0.40278417, Schwarzkiefer = -0.35177806, Zirbe = -0.40715214, Rotbuche = -0.17560870, Eiche = -0.11356914, Hainbuche = -0.24940727, Esche = 0.08089571, Ahorn = 0.11923299, Ulme = 0.40572294, Edelkastanie = 0.25672288, Robinie = 0.24202747, Sorbus- und Prunusarten = -0.33233703, Birke = -0 .03686508, Schwarzerle = 0.20264060, Grauerle = 0.11576176, Linde = -0.14250231, Aspe, Weiß-, Silberpappel = 0.10775285, Baumweiden = 0.42065542, Rest = 0.09451984}

\subsection{Höhenzuwachs}

Die Koeffizienten für den Höhenzuwachs der Fichte sind:

\texttt{ch0 = 2.478770320, ch1 = -0.214554399, ch2 = -0.712918493, ch3 = 3.481531436, ch4 = -2.999678187\\
ch5: Fichte=-0.023742528, Tanne=-0.007042547, Rotbuche=-0.028132932, Lärche=-0.011868042, Rest=-0.014606372, Weißkiefer=-0.009344628\\
chGh: Fichte=1.4525345, Tanne=1.2212739, Rotbuche=1.1591178, Lärche=1.0646313, Rest=0.9495616, Weißkiefer=0.9512430\\
chHh: Fichte=1.1495333, Tanne=0.9032277, Rotbuche=0.8445919, Lärche=0.8214862, Rest=0.8677137, Weißkiefer=0.5961510\\
ch6= 1.7031438, ch7 = 0.1198871}


\subsection{Durchmesserzuwachs}

Die Koeffizienten für den Durchmesserzuwachs der Fichte sind:

\texttt{
  cd = 1.2423323\\
  cdw: Fichte=1, Tanne=0.9827017, Rotbuche=0.7934112, Lärche=1.1037928, Rest=0.7442270, Weißkiefer=0.6551159\\
  cd0 = -4.7340084, cd1 = 0.9392332, cd2 = 1.3491139, cd3 = 14.8226513, cd4 = -8.1405769, cd5 = 0.7164459\\
  cd6: Fichte=-0.0122210, Tanne=-0.0103623, Rotbuche=-0.0289442, Lärche=-0.0135921, Rest=-0.0043686, Weißkiefer=-0.0102314\\
  cd7: Fichte=-0.0119269, Tanne=-0.0036543, Rotbuche=-0.0111465, Lärche=-0.0052169, Rest=-0.0109765, Weißkiefer=-0.0050748\\
  cdCba = 0.5210695, cdHh = 1.0712452\\
  cdIvMax = 6.255492\\
  cdSl0 = 1.3832451, cdSl1 = -0.5352921, cdSh0 = -2.4570928, cdSh1 = 0.7596363\\
}

\subsection{Durchmesser aus Baumhöhe}

Die Koeffizienten zur Abschätzung der BHD's der Fichte sind:

\texttt{
  cd: Kleinwald bis 200Ha = -0.69595545, Bundesforste = -0.6982561, Privat 200-1000ha = -0.6962923, Privat > 1000ha = -0.6773922, Gebieskörperschaften = -0.7316092\\
  cd1 = -0.02158519, cd2 = 0.31476845, cd3 = -0.013345940, cd4 = -0.0002650310, cd5 = 0.0031319565\\
}

\subsection{Ergänzung des Kronenansatzes}

Die Koeffizienten zur Bestimmung der Kronenansatzhöhe der Fichte sind:

\texttt{
  chk0 = 0.176194138297613, chk1 = -0.563810526391337, chk2 = -0.00232276544050889
}

\subsection{Zufallsnutzung und Bruchereignisse}

Die verwendete Geologiekarte unterschied folgende typen: \texttt{1 =
  Quartär i. Allg. (Alluvium; Pleistozän entlang der
  Hauptentwässerungslinien und Moränen im Alpenvorland), 2 =
  Molassezone; Obereozän - Miozän; Inneralpine Becken; Neogen, 3 =
  Allochthone und parautochthone Molasse; Obereozän - Miozän/Oberjura,
  4 = Alttertiärklippen (Ernstbrunner Klippen), 5 = Andesit, Dazit,
  Trachyt; Karpat, Baden, 6 = Basalt, Basanit, Nephelinit, Tuff,
  Sarmat/Pannon - Plio-/Pleistozän, 7 = Post-variszische Klastika
  (Perm von Zöbing); Perm, 8 = Granitoid (Südböhmischer Pluton);
  Karbon, 9 = Metamorphite i. Allg.: meist Paragneis, Glimmerschiefer
  (Moldanubikum, Moravikum), 10 = Orthogneis, 11 = Amphibolit, 12 =
  Marmor, Kalksilikatgestein, 13 = Granulit, 14 = Ultrabasit, 15 =
  Kontinentalrandsediment (Helvetikum i.w.S. inkl. Grestener- und
  Hauptkilppenzone); Jura - Mitteleozän, 16 = Liebensteiner- und
  Feuerstätter Decke (nicht differenziert); Lias - Eozän, 17 =
  Rhenodanubischer Flysch; Unterkreide - Eozän, 18 = Tiefmarines
  Sediment - Ophiolith (Ybbsitzer- , Sulzer- und St. Veiter-Klippen,
  Nordrandzone); Jura - Kreide, 19 = Ozeanisches Metasediment,
  z. T. flyschartig (Bündner Schiefer, Rechnitzer Serie,
  Prättigauflysch); Jura - Kreide, z. T. Alttertiär, 20 =
  Grünschiefer, Prasinit, Serpentinit, 21 = Eklogit führendes
  Metasediment, 22 = Metasediment (in Falknis- und Sulzfluh-Decke
  nicht differenziert ); Permomesozoikum, z. T. Alttertiär, 23 =
  Metasediment (Brennkogel-, Kaserer-Serie); Jura - Kreide, 24 =
  Metasediment (Hochstegen-Serie); Malm, 25 = Metasediment
  (Wustkogel-, Seidlwinkel-, Schrovin-Serie); Permotrias, 26 =
  Orthogneis (Zentralgneis); Permokarbon, 27 = Metasediment,
  Metavulkanit (Habach-, Greiner-, Storz-, Kareck-Serie); Paläozoikum,
  28 = Migmatit, Anatexit, migmatischer Paragneis (Altes Dach,
  Altkristallin i. Allg.), 29 = Amphibolit
  (Zwölferzug-Basisamphibolit), 30 = überwiegend Karbonatgestein;
  Mitteltrias - Unterkreide, 31 = Siliciklastika; Permoskyth, 32 =
  meist Klastika (Gosau - Schichten); Oberkreide - Eozän, 33 =
  Karbonatgestein, Klastika (Karbon von Nötsch); oberes Vise -
  Oberkarbon, 34 = Karbonatgestein, Klastika (Grauwackenzone /
  Veitscher Decke); oberes Vise - Oberkarbon, 35 = Phyllit,
  Metaklastika, Metavulkanit (Grauwackenzone/Silbersberg-Decke
  westl. Aflenz nicht ausgeschieden); Altpaläozoikum
  i. Allg. ?Permoskyth, 36 = Post-variszische Klastika; Oberkarbon, 37
  = überwiegend pelitisch-psammitisches Sediment; Oberordovicium, 38 =
  Karbonatgestein, 39 = Basischer Vulkanit, 40 = Porphyroid
  (Blasseneck Porphyroid); Oberordovicium, 41 = Granitoid;
  Permokarbon, 42 = Altkristallin i. Allg. (meist Paragneis,
  Glimmerschiefer lokal, auch Granatphyllit), 43 = Orthogneis, 44 =
  Amphibolit, 45 = Marmor, 46 = Ultrabasit, 47 = Schladminger
  Kristallin, 48 = Bundschuh Kristallin, 49 = Tonalit, Granodiorit,
  Oligozän, 50 = Karbonatgestein; Trias, 51 = Post-variszische
  Klastika und Karbonatgestein; Oberkarbon - Perm, 52 = Kalk,
  Feinklastika, 53 = Phyllit; Oberordovicium - Unterkarbon, 54 =
  Quarzphyllite, z. T. Phyllonite, 55 = Siliciklastika; Permoskyth, 56
  = überwiegend Karbonatgestein; Mitteltrias - Jura, 57 = Migmatit, 58
  = Porphyroid; Perm, 59 = Tektonische Melange ostalpiner und
  penninischer Gesteine (Matreier Zone - Nordrahmenzone,
  Richbergkogel-Serie, Arosa-Zone); Permomesozoikum, 60 = Metasediment
  (Tasna-Decke); Permotrias, 61 = Firn, Gletscher, 62 = Gewässer}

\subsubsection{Zufallsnutzung}

Die Koeffizienten zur Bestimmung der Wahrscheinlichkeit von
Zufallsnutzungen sind:

\texttt{czpGeologie: 1 = -8.30256937894367, 2 = -8.08871507353082, 3 = -8.11571058388957, 4 = -19.8359812836458, 5 = -6.82468098910134, 6 = -7.04153475076478, 7 = -20.2783461786802, 8 = -8.08536642999795, 9 = -8.43950837757173, 10 = -7.96618422806721, 11 = -7.93160504833828, 12 = -20.3510236763067, 13 = -7.797394212408, 14 = -20.2374616766888, 15 = -8.8267711687451, 16 = -8.32301475646734, 17 = -8.22808736258911, 18 = -9.30912676971561, 19 = -8.2261592655735, 20 = -8.19748944486436, 21 = -8.30256937894367, 22 = -8.32032252576032, 23 = -8.86892881623049, 24 = -8.30256937894367, 25 = -7.94227532115599, 26 = -9.50353270954064, 27 = -8.61883128578808, 28 = -8.96201007890408, 29 = -20.9182396848282, 30 = -8.30759753164257, 31 = -8.34072917101393, 32 = -8.56880910493686, 33 = -21.0999193295738, 34 = -8.19910780452667, 35 = -8.30256937894367, 36 = -8.09550723106612, 37 = -8.33387009702767, 38 = -8.38040503797902, 39 = -8.94279395912906, 40 = -10.1720605784638, 41 = -8.54360409910846, 42 = -8.63990356156941, 43 = -8.81494970721182, 44 = -8.16534032760471, 45 = -8.86820323258083, 46 = -8.30256937894367, 47 = -9.39331673979358, 48 = -8.70205678141078, 49 = -20.6054854813017, 50 = -9.19944031929362, 51 = -9.47647446376869, 52 = -8.49590402219882, 53 = -20.5578874256253, 54 = -8.85006624342649, 55 = -8.07988633644507, 56 = -8.18821526443021, 57 = -8.07205282276932, 58 = -20.8440037233556, 59 = -8.39984891208148, 60 = -20.7951032133098, 61 = -8.30256937894367, 62 = -21.2283852424871\\
  czp1 = 0.0225333258856312\\
  czp2: Fichte = 0.014373484995117, Tanne = 0.00129157419682988,
  Lärche = 0.00706441753754056, Weißkiefer = -0.0253178078716667,
  Rotbuche = 0.00468822998768224, Eiche = 0.0106845085774845, Sonstige
  Nadelbäume = 0.0190035463975034, Sonstige Laubbäume =
  0.00216545238614761\\
  czp3 = 0.019204809264182, czp4 = -6.82706934633721e-05\\
  czp5: Fichte = 0.00256315899523833, Tanne = -0.00232696837895208,
  Lärche = -0.0025486163406973, Weißkiefer = 0.00510876429361565,
  Rotbuche = -0.00107484225869725, Eiche = -0.0040440513111897,
  Sonstige Nadelbäume = -0.00697160802096001, Sonstige Laubbäume =
  -0.000392100957238878 czp6 = 0.241305715058825 }

Die Koeffizienten zur Bestimmung dass alle Bäume genutzt werden wenn
es ein Zufallsnutzung gibt sind:

\texttt{czpaGeologie: 1 = 11.0790627922303, 2 = 10.3246556124911, 3 = 11.8788008985786, 4 = 11.0790627922303, 5 = -6.1562640919763, 6 = -4.3823086720443, 7 = 11.0790627922303, 8 = 9.1211939373004, 9 = 11.5699158809105, 10 = 10.1811835557015, 11 = 11.2955684980698, 12 = 11.0790627922303, 13 = 11.7329621164063, 14 = 11.0790627922303, 15 = -4.9125618880595, 16 = -3.8420331037367, 17 = 8.98805894534786, 18 = -3.8844346755181, 19 = 10.7820049483129, 20 = -4.0174491810746, 21 = 11.0790627922303, 22 = -4.1924667854116, 23 = -3.6113243708454, 24 = 11.0790627922303, 25 = -4.2054804224754, 26 = 13.8646429088023, 27 = -5.5327353576535, 28 = -3.466495024734, 29 = 11.0790627922303, 30 = 10.4480483657256, 31 = 9.732027155678, 32 = 11.4246487714958, 33 = 11.0790627922303, 34 = -6.4848393647968, 35 = 11.0790627922303, 36 = 33.0908087903055, 37 = 10.6445806474352, 38 = 11.1348986706791, 39 = -4.8895616929252, 40 = -5.7568342630096, 41 = -5.0149782092159, 42 = 9.8247371944927, 43 = 32.3470141839399, 44 = 10.8317270293504, 45 = -4.918269109839, 46 = 11.0790627922303, 47 = -5.0100454091237, 48 = 11.2000405291458, 49 = 11.0790627922303, 50 = -3.7795688774657, 51 = 27.4288370970397, 52 = 10.8850357149786, 53 = 11.0790627922303, 54 = -5.1123668978069, 55 = 11.756518514744, 56 = 10.2188893040119, 57 = 10.8112224292404, 58 = 11.0790627922303, 59 = 11.169988216078, 60 = 11.0790627922303, 61 = 11.0790627922303, 62 = 11.0790627922303\\
  czpa1 = -4.15036298957266
  , czpa2 = -1.35713028813666
  , czpa3 = 0.21843439092469
}

Die Koeffizienten zur Bestimmung der Wahrscheinlichkeit, dass ein Baum zufällig genutzt wird, auf einem Punkt mit Zufallsnutzung, auf dem nicht alle Bäume genutzt wurden lauten:

\texttt{czpbGeologie: 1 = -4.44226686221951, 2 = -4.62630813794232, 3 = -20.6945113490088, 4 = -4.44226686221951, 5 = -19.8027129126629, 6 = -21.1670318160569, 7 = -4.44226686221951, 8 = -5.94203107558192, 9 = -4.73146600251547, 10 = -5.60213761720105, 11 = -20.9904730711994, 12 = -4.44226686221951, 13 = -20.4586252579172, 14 = -4.44226686221951, 15 = -20.6693659158126, 16 = -20.53968761141, 17 = -4.56337664838091, 18 = -3.86058581940821, 19 = -3.64316169019144, 20 = -20.1007315895403, 21 = -4.44226686221951, 22 = -19.936693791771, 23 = -20.4899423968611, 24 = -4.44226686221951, 25 = -20.2359505982548, 26 = -20.5188403148562, 27 = -20.4883103934481, 28 = -20.2392253860276, 29 = -4.44226686221951, 30 = -5.0705346547036, 31 = -4.94372032799955, 32 = -5.64971678713203, 33 = -21.0330416050764, 34 = -3.96394429312823, 35 = -4.44226686221951, 36 = -4.44226686221951, 37 = -4.7074222463544, 38 = -3.8290993320417, 39 = -20.6324392425798, 40 = -20.8710555272345, 41 = -4.34448198597979, 42 = -5.03708041503771, 43 = -20.4930095031568, 44 = -4.76205061241755, 45 = -20.5736336642792, 46 = -4.44226686221951, 47 = -20.5217746140384, 48 = -4.75396295704817, 49 = -20.2584270978137, 50 = -20.3537209329255, 51 = -20.5080051141535, 52 = -5.38336243008069, 53 = -20.1098707177856, 54 = -6.57936398899202, 55 = -4.56766571088515, 56 = -20.4277897487765, 57 = -21.0387219807324, 58 = -20.7520917739882, 59 = -20.8009625117877, 60 = -4.44226686221951, 61 = -4.44226686221951, 62 = -4.44226686221951\\
  czpb1 = 0.017618747589559
  , czpb2 = -0.0103807085088108
  , czpb2 = -0.000841956057332202
}

\subsubsection{Bruchereignisse}

Die Koeffizienten zur Bestimmung eines Bruchereignisses der Fichte sind:

\texttt{cbpGeologie: 1 = 5.64723103375928, 2 = 5.58538131688251, 3 = 5.68070638953452, 4 = -5.10046577141812, 5 = 6.51906361089464, 6 = 6.56343106667525, 7 = -5.76715845667112, 8 = 5.89456305953475, 9 = 5.65751147031432, 10 = 5.91198159799151, 11 = 6.23460575462552, 12 = -5.49242984064802, 13 = 4.3303027299514, 14 = -5.08418122106922, 15 = 5.54463276733589, 16 = 6.17823539940015, 17 = 5.77557945769026, 18 = 5.60922226624466, 19 = 6.16625449787241, 20 = 5.26050384048405, 21 = 5.64723103375928, 22 = 6.57877733538479, 23 = 5.26872573381007, 24 = 5.64723103375928, 25 = 6.04201993464926, 26 = 5.61875680565068, 27 = 5.3742349387598, 28 = 6.54286017647141, 29 = -6.75209130102572, 30 = 6.21789307030023, 31 = 6.20051901329182, 32 = 6.06409547544508, 33 = 7.00961590259922, 34 = 6.2065740457706, 35 = 5.64723103375928, 36 = -5.97682288953952, 37 = 5.8613801743823, 38 = 5.92370927421033, 39 = 6.21784091688131, 40 = 5.77924880635635, 41 = 5.75406947375954, 42 = 5.6706786685704, 43 = 5.55256870285861, 44 = 5.51400868562721, 45 = 5.50466977405054, 46 = 5.64723103375928, 47 = 5.31318411291283, 48 = 5.60580604160981, 49 = 5.06285438515232, 50 = 6.052843911185, 51 = 5.89592629191158, 52 = 5.81896952987401, 53 = 7.0374399419465, 54 = 5.9505738843445, 55 = 6.18562241189085, 56 = 5.72918883495697, 57 = 6.15443964122795, 58 = 6.94597823506487, 59 = 5.67120582773727, 60 = -6.45420507038732, 61 = 5.64723103375928, 62 = -6.25997614906692\\
  cbp1 = 0.0492826146914046
  , cbp2 = 0.00670945689359753
  , cbp3 = 0.998958294586338
  , cbp4 = -0.0853681028833942
  , cbp5 = 0.0206966105362673
  , cbp6 = -2.12812636272471
  , cbp7 = 0.00158758559899495
  , cbp8 = 0.00679447611444564
}

Die Koeffizienten zur Bestimmung dass alle Bäume gebrochen sind lauten:

\texttt{cbpa0 = 2.10186159615126, cbpa1 = -2.66059736152267}

Die Koeffizienten zur Bestimmung der Wahrscheinlichkeit, dass ein Baum bricht, auf einem Punkt mit Bruch, auf dem nicht alle Bäume brechen lauten:

\texttt{cbpbGeologie: "1 = -3.11404123818367, 2 = -3.2089199885061, 3 = -3.48552667028197, 4 = -3.11404123818367, 5 = -2.50180230410734, 6 = -1.61361628039196, 7 = -3.11404123818367, 8 = -3.28225392022144, 9 = -3.23137804080334, 10 = -3.21700480655566, 11 = -2.54347051752028, 12 = -3.11404123818367, 13 = -4.06576888107169, 14 = -3.11404123818367, 15 = -3.44513948052119, 16 = -3.71575963755247, 17 = -3.09168650916898, 18 = -3.75860454049094, 19 = -3.25102281122018, 20 = -2.76257749625755, 21 = -3.11404123818367, 22 = -3.54210080990557, 23 = -3.68057484695281, 24 = -3.11404123818367, 25 = -3.28721197847364, 26 = -3.81895034370782, 27 = -3.23798037404374, 28 = -3.31669222326313, 29 = -3.11404123818367, 30 = -3.14929834368431, 31 = -3.05073910467329, 32 = -3.18861475165285, 33 = -3.51735910853901, 34 = -2.99594239934687, 35 = -3.11404123818367, 36 = -3.11404123818367, 37 = -3.14244798973901, 38 = -3.34993635921998, 39 = -3.2617121314878, 40 = -3.13705067762549, 41 = -2.89415121696867, 42 = -3.42150667505539, 43 = -3.80007466721556, 44 = -3.38097479231494, 45 = -3.37291813813285, 46 = -3.11404123818367, 47 = -2.52067347737501, 48 = -3.59489668438266, 49 = -3.79840257056817, 50 = -1.98508788636281, 51 = -3.46111281717833, 52 = -3.56446241779075, 53 = -3.46066058861148, 54 = -3.46453215543565, 55 = -3.12019648171032, 56 = -3.28599854843101, 57 = -3.20012320004806, 58 = -3.8291003859388, 59 = -3.42184759310081, 60 = -3.11404123818367, 61 = -3.11404123818367, 62 = -3.11404123818367\\
  cbpb1 = -0.0412216231815078 , cbpb2 = 0.0102208274713109 , cbpb3 =
  0.02935633409254 , cbpb4 = -0.00223226976745298 , cbpb5 =
  0.000253006766449054 }


\subsection{Ertragstafel}

Die Raumdichte der Fichte [$\text{kg/m}^3$] wurde mit
Gleichung~\ref{eq:raumdichte} bestimmt, wobei \emph{rw} die
Jahrringbreite [mm] und \emph{BHD} der Durchmesser in 1.3\,m Höhe [cm]
ist.
\begin{dmath}
  \label{eq:raumdichte}
  \text{Raumdichte} = e^{-1.456987 - 0.048796 * rw + 0.114872 * \ln(1 + BHD)}
\end{dmath}

Die Ast-- und Nadelmasse wurde nach
\cite{eckmuellner2006NadelAstmasse} mit Gleichung~\ref{eq:blattmasse}
und \ref{eq:astmasse} [Atro-kg] bestimmt wobei der \emph{BHD} [cm], die
Baumhöhe \emph{h} [m] und der Kronenverhätniss \emph{cr} [1] verwendet
werden.
\begin{dgroup}
  \begin{dmath}
    \label{eq:blattmasse}
    \text{Nadelmasse} = e^{-1.706416 + 1.9710 * log(BHD) - 0.3982 * log(h) - 0.9881 * (1-cr)}
  \end{dmath}
  \begin{dmath}
    \label{eq:astmasse}
    \text{Astmasse} = e^{-2.79494 + 2.8420 * log(BHD) - 0.8525 * log(h) - 0.9760 * (1-cr)}
  \end{dmath}
\end{dgroup}

Die Wurzelmasse [Atro-kg] wurde nach
\cite{petersson2006belowGroundBiomass} über den BHD [cm] bestimmt.

\begin{dmath}
  \label{eq:wurzelmasse}
  \text{Wurzelmasse} = \frac{e^{4.58761 + 10.44035 * (BHD*10) / ((BHD*10) + 138)}}{1000}
\end{dmath}

Der Rindenanteil am Schaftvolumen wurde nach
\cite{peintinger1973kubierungstabelle} und
\cite{guede1988kubierungstabelle}, ausgeglichen von Günter Rössler, mit
dem BHD [cm] bestimmt.

\begin{dmath}
  \label{eq:rindenanteil}
  \text{Rindenanteil} = BHD^2 - \frac{(BHD - 0.2583 - 0.0406*BHD)^2}{d^2}
\end{dmath}

Das Schaftholzvolumen wurde mit Hilfe der Formzahl nach
\cite{pollanschuetz1974Formzahlen}
bzw. \cite{schieler1997DisWaldinventur} bestimmt, wobei für
$5 <= \text{BHD} < 10.6$\,cm die Koeffizienten b1=0.563443,
b2=-0.12731, b3=-8.55022, b4=0, b5=0, b6=7.6331 und b7=0 sowie für BHD
> 10.6\,cm b1=0.46818, b2=-0.013919, b3=-28.213, b4=0.37474,
b5=-0.28875, b6=28.279 und b7=0 sowie für BHD < 5\,cm fz=0.5 verwendet
wurde. Wobei bei der Berechnung der Formzahl sowohl BHD als auch Höhe
in dm einzusetzen sind.

\begin{dmath}
  \label{eq:Schaftholz}
  fz = b1 + b2*log(d)^2 + b3/h + b4/d + b5/d^2 + b6/(d*h) + b7/(d^2*h)\\
  V_s = BHD^2* \pi / 4 * h * fz
\end{dmath}

\begin{description}
\item[age] Alter [Jahre]
\item[ho] Oberhöhe [m]
\item[hg] Höhe des Grundflächemittelstammes [m]
\item[dg] Durchmesser des Grundflächenmittelstammes [cm]
\item[hKa] Höhe des Kronenansatzes [m]
\item[g] Grundfläche [m2/ha]
\item[v] Schaftholzvolumen Vorratsfestmeter [m3/ha]
\item[s.NH] Erntefestmeter Sonstiges Nutzholz [m3/ha]
\item[1b] Erntefestmeter der Stärkeklasse 15-20\,cm [m3/ha]
\item[2a] Erntefestmeter der Stärkeklasse 20-25\,cm [m3/ha]
\item[2b] Erntefestmeter der Stärkeklasse 25-30\,cm [m3/ha]
\item[3a] Erntefestmeter der Stärkeklasse 30-35\,cm [m3/ha]
\item[3b] Erntefestmeter der Stärkeklasse 35-40\,cm [m3/ha]
\item[4+] Erntefestmeter der Stärkeklasse >40\,cm [m3/ha]
\item[n] Stammzahl [1/ha]
\item[BmS] Trockenbiomasse Schaftholz [t/ha]
\item[BmA] Trockenbiomasse Äste [t/ha]
\item[BmN] Trockenbiomasse Nadel [t/ha]
\item[BmR] Trockenbiomasse Rinde [t/ha]
\item[BmW] Trockenbiomasse Wurzeln [t/ha]
\item[BmHarv] Trockenbiomasse die bei einer Ernte üblicherweise entnommen wird [t/ha]
\item[BmResid] Trockenbiomasse die bei einer Ernte üblicherweise zurück bleibt [t/ha]
\item[..Aus] Für den Ausscheidenden Bestand
\item[hoRef] Referenzoberhöhe [m]
\item[lfz] Laufender Zuwachs Erntefestmetern [m3/ha/Jahr]
\item[dgz] Durchschnittlicher Gesamtzuwachs Erntefestmetern [m3/ha/Jahr]
\end{description}

\begin{table}[ht]
  \begin{adjustbox}{angle=90,height=\textheight}
\centering
\begin{tabular}{rrrrrrrrrrrrrrrrrrrrrr|rrrrrrrrrrrrrrr|rrr}
  \hline
age & ho & hg & dg & hKa & g & v & s.NH & 1b & 2a & 2b & 3a & 3b & 4+ & n & BmS & BmA & BmN & BmR & BmW & BmHarv & BmResid & hgAus & dgAus & hKaAus & gAus & vAus & s.NHAus & 1bAus & 2aAus & 2bAus & 3aAus & 3bAus & 4+Aus & nAus & BmHarvAus & BmResidAus & hoRef & lfz & dgz \\
\hline
0 & 0 & 0 & 0 & 0 & 0 & 0 & 0 & 0 & 0 & 0 & 0 & 0 & 0 & 2400 & 0 & 0 & 0 & 0 & 0 & 0 & 0 & 0 & 0 & 0 & 0 & 0 & 0 & 0 & 0 & 0 & 0 & 0 & 0 & 0 & 0 & 0 & 0 & 0 & 0 \\
10 & 0.4 & 0 & 0 & 0 & 0 & 0 & 0 & 0 & 0 & 0 & 0 & 0 & 0 & 2131.4 & 0 & 0 & 0 & 0 & 0 & 0 & 0 & 0 & 0 & 0 & 0 & 0 & 0 & 0 & 0 & 0 & 0 & 0 & 0 & 268.6 & 0 & 0 & 0.4 & 0 & 0 \\
20 & 1.8 & 1.7 & 3.3 & 0.1 & 0.8 & 0.7 & 0 & 0 & 0 & 0 & 0 & 0 & 0 & 957.4 & 0.3 & 2.7 & 2.7 & 0.1 & 0.7 & 0 & 6.5 & 1.6 & 3.1 & 0.1 & 0.7 & 0.6 & 0 & 0 & 0 & 0 & 0 & 0 & 0 & 1147.1 & 0 & 6 & 1.8 & 0 & 0 \\
30 & 3.7 & 3.4 & 7.6 & 0.2 & 4.3 & 8.7 & 0 & 0 & 0 & 0 & 0 & 0 & 0 & 957.4 & 3 & 9.7 & 7.5 & 0.4 & 3.8 & 0 & 24.4 & 0 & 0 & 0 & 0 & 0 & 0 & 0 & 0 & 0 & 0 & 0 & 0 & 0 & 0 & 0 & 3.9 & 0 & 0 \\
40 & 5.7 & 5.2 & 11 & 0.5 & 9 & 24.4 & 11.8 & 0 & 0 & 0 & 0 & 0 & 0 & 955.7 & 8 & 16.3 & 11.5 & 1 & 9.6 & 4.8 & 41.6 & 4.8 & 10 & 0.4 & 0 & 0 & 0 & 0 & 0 & 0 & 0 & 0 & 0 & 1.7 & 0 & 0.1 & 6.1 & 1.2 & 0.3 \\[1em]
50 & 7.5 & 6.9 & 14.1 & 0.7 & 8.7 & 28.4 & 17.6 & 0 & 0 & 0 & 0 & 0 & 0 & 560 & 9.2 & 14.1 & 9.3 & 1.1 & 10.8 & 7 & 37.4 & 6 & 12.1 & 0.6 & 4.5 & 13.5 & 7.6 & 0 & 0 & 0 & 0 & 0 & 0 & 395.7 & 3.1 & 20 & 8.1 & 1.3 & 0.5 \\
60 & 9.2 & 8.5 & 16.9 & 1 & 12.5 & 47.4 & 31.4 & 0.1 & 0 & 0 & 0 & 0 & 0 & 554.8 & 15.3 & 18.8 & 11.7 & 1.7 & 17.2 & 12.5 & 52.2 & 8 & 15.9 & 0.9 & 0.1 & 0.4 & 0.2 & 0 & 0 & 0 & 0 & 0 & 0 & 5.2 & 0.1 & 0.4 & 9.9 & 1.4 & 0.7 \\
70 & 10.6 & 9.8 & 19.5 & 1.3 & 16.4 & 70.6 & 48 & 0.9 & 0.2 & 0 & 0 & 0 & 0 & 550.4 & 22.8 & 23.6 & 14 & 2.5 & 24.4 & 19.3 & 68.1 & 9.6 & 19.6 & 1.2 & 0.1 & 0.5 & 0.4 & 0 & 0 & 0 & 0 & 0 & 0 & 4.3 & 0.1 & 0.5 & 11.5 & 1.8 & 0.8 \\
80 & 11.9 & 11 & 21.8 & 1.5 & 20.5 & 97.5 & 58.5 & 2.4 & 8.9 & 0 & 0 & 0 & 0 & 547.8 & 31.6 & 28.7 & 16.3 & 3.4 & 32.3 & 27.3 & 85 & 10.9 & 22.6 & 1.4 & 0.1 & 0.5 & 0.3 & 0 & 0.1 & 0 & 0 & 0 & 0 & 2.6 & 0.1 & 0.4 & 12.8 & 2.1 & 1 \\
90 & 13.1 & 12.1 & 24 & 1.7 & 24.6 & 127.2 & 59.7 & 3.7 & 29.4 & 0.1 & 0 & 0 & 0 & 544.3 & 41.3 & 33.9 & 18.6 & 4.3 & 40.5 & 36.2 & 102.3 & 11.8 & 23 & 1.7 & 0.1 & 0.7 & 0.4 & 0 & 0.1 & 0 & 0 & 0 & 0 & 3.5 & 0.2 & 0.6 & 14 & 2.4 & 1.1 \\[1em]
100 & 14.1 & 13.1 & 26 & 1.9 & 28.5 & 157.6 & 58.4 & 4 & 47.9 & 6.6 & 0 & 0 & 0 & 535.7 & 51.3 & 38.8 & 20.6 & 5.3 & 48.4 & 45.6 & 118.8 & 12.6 & 24.8 & 1.8 & 0.4 & 2.2 & 1 & 0.1 & 0.6 & 0 & 0 & 0 & 0 & 8.7 & 0.6 & 1.7 & 15 & 2.6 & 1.3 \\
110 & 15.1 & 13.9 & 28 & 2.1 & 32.1 & 188.1 & 55.6 & 3.3 & 56.1 & 26 & 0.2 & 0 & 0 & 523.5 & 61.4 & 43.4 & 22.4 & 6.2 & 55.9 & 55.1 & 134.1 & 13.6 & 26.5 & 2.1 & 0.7 & 3.9 & 1.3 & 0.1 & 1.2 & 0.3 & 0 & 0 & 0 & 12.2 & 1.1 & 2.7 & 15.9 & 2.7 & 1.4 \\
120 & 15.8 & 14.7 & 29.8 & 2.2 & 28.6 & 175.5 & 42.5 & 2 & 46.7 & 38.1 & 3.7 & 0 & 0 & 411.3 & 57.4 & 38.6 & 19.3 & 5.7 & 50.7 & 51.9 & 119.8 & 14.4 & 29 & 2.1 & 7.4 & 44.5 & 11.7 & 0.6 & 12.4 & 8.6 & 0.3 & 0 & 0 & 112.2 & 13.1 & 31 & 16.8 & 2.5 & 1.5 \\
130 & 16.5 & 15.4 & 31.7 & 2.3 & 31.6 & 201.6 & 39.2 & 1.3 & 44.9 & 52.6 & 16.1 & 0 & 0 & 400.9 & 66.1 & 42.8 & 20.8 & 6.5 & 56.7 & 60.2 & 132.7 & 14.8 & 28.7 & 2.3 & 0.7 & 4.2 & 1.1 & 0.1 & 1.2 & 0.7 & 0.1 & 0 & 0 & 10.4 & 1.2 & 2.7 & 17.5 & 2.4 & 1.6 \\
140 & 17.1 & 16.1 & 33.5 & 2.4 & 33.9 & 222.8 & 35 & 0.9 & 41.2 & 61.5 & 32.4 & 0.8 & 0 & 385.2 & 73.2 & 46.1 & 21.7 & 7.1 & 61.3 & 67.1 & 142.4 & 16.4 & 33.3 & 2.6 & 1.4 & 9.3 & 1.5 & 0 & 1.8 & 2.7 & 1.2 & 0 & 0 & 15.7 & 2.8 & 5.7 & 18.2 & 2.5 & 1.6 \\[1em]
150 & 17.8 & 16.6 & 35.3 & 2.5 & 35.6 & 241.4 & 30.7 & 0.5 & 36.4 & 63.3 & 47.6 & 8.8 & 0 & 365.2 & 79.5 & 48.8 & 22.4 & 7.7 & 64.9 & 73.2 & 150.1 & 16.5 & 33.2 & 2.6 & 1.7 & 11.8 & 1.9 & 0 & 2.3 & 3.4 & 1.4 & 0 & 0 & 20 & 3.6 & 7.1 & 18.9 & 2.5 & 1.7 \\
160 & 18.3 & 17.2 & 36.9 & 2.6 & 36.5 & 253.3 & 27.2 & 0.3 & 32 & 61.4 & 56.8 & 20 & 0 & 340.9 & 83.6 & 50.3 & 22.6 & 8 & 66.7 & 77.4 & 153.8 & 17.4 & 36.7 & 2.7 & 2.6 & 18.2 & 2 & 0 & 2.4 & 4.5 & 4.4 & 1 & 0 & 24.3 & 5.5 & 10.8 & 19.5 & 2.5 & 1.7 \\
170 & 18.8 & 17.7 & 38.6 & 2.7 & 37.9 & 268.9 & 24.6 & 0.2 & 28.2 & 58.4 & 65.1 & 34.5 & 0 & 323.5 & 89 & 52.7 & 23 & 8.4 & 69.3 & 82.7 & 159.8 & 17.7 & 37.3 & 2.8 & 1.9 & 13.7 & 1.4 & 0 & 1.7 & 3.3 & 3.3 & 0.9 & 0 & 17.4 & 4.2 & 7.9 & 20 & 2.4 & 1.8 \\
180 & 19.3 & 18.2 & 40.3 & 2.8 & 39.5 & 286 & 23.2 & 0.1 & 25.8 & 57.1 & 74 & 45.2 & 0 & 309.6 & 94.9 & 55.3 & 23.6 & 8.9 & 72.2 & 88.4 & 166.5 & 18.1 & 38.4 & 2.9 & 1.6 & 11.8 & 1.1 & 0 & 1.3 & 2.6 & 2.9 & 1.4 & 0 & 13.9 & 3.6 & 6.7 & 20.5 & 2.4 & 1.8 \\
190 & 19.7 & 18.6 & 41.8 & 2.9 & 41.5 & 306.4 & 22.6 & 0.1 & 24.3 & 56.9 & 79.9 & 58.8 & 0 & 301.7 & 101.8 & 58.6 & 24.5 & 9.5 & 75.8 & 95.2 & 175 & 18.8 & 42.3 & 2.9 & 1.1 & 8.1 & 0.6 & 0 & 0.7 & 1.5 & 2.2 & 1.5 & 0 & 7.8 & 2.5 & 4.6 & 21 & 2.4 & 1.8 \\[1em]
200 & 20.2 & 19.1 & 43.5 & 2.9 & 43 & 322.5 & 22 & 0.1 & 22.8 & 55.8 & 82.4 & 73.2 & 0 & 289.6 & 107.4 & 61.3 & 25.1 & 10 & 78.3 & 100.8 & 181.2 & 19.1 & 40.8 & 3.1 & 1.6 & 12.1 & 0.9 & 0 & 1 & 2.4 & 3.2 & 2 & 0 & 12.2 & 3.8 & 6.5 & 21.5 & 2.3 & 1.9 \\
210 & 20.5 & 19.4 & 45 & 3 & 43.8 & 333.8 & 21.7 & 0 & 21.9 & 54.7 & 83.2 & 84.3 & 0.4 & 275.7 & 111.4 & 63.1 & 25.3 & 10.3 & 79.5 & 104.8 & 184.8 & 19.7 & 44.1 & 3.1 & 2.1 & 16.4 & 1.1 & 0 & 1.1 & 2.8 & 4.1 & 3.9 & 0 & 13.9 & 5.1 & 8.8 & 21.9 & 2.3 & 1.9 \\
220 & 20.9 & 19.9 & 46.6 & 3.1 & 45 & 347.8 & 21.5 & 0 & 21.4 & 53.9 & 84.7 & 94.1 & 2.8 & 263.5 & 116.3 & 65.3 & 25.6 & 10.7 & 81.2 & 109.7 & 189.4 & 19.2 & 43.3 & 3 & 1.8 & 13.4 & 0.9 & 0 & 0.9 & 2.3 & 3.3 & 3.2 & 0 & 12.2 & 4.2 & 7.4 & 22.3 & 2.3 & 1.9 \\
230 & 21.3 & 20.2 & 48.1 & 3.1 & 46.4 & 363.1 & 21.6 & 0 & 21.2 & 53.7 & 86.1 & 100.9 & 8 & 254.8 & 121.7 & 67.9 & 26.2 & 11.1 & 83.2 & 115.1 & 195 & 20.3 & 46.2 & 3.3 & 1.4 & 11.6 & 0.7 & 0 & 0.7 & 1.8 & 2.9 & 3.1 & 0 & 8.7 & 3.7 & 6 & 22.7 & 2.2 & 1.9 \\
240 & 21.6 & 20.6 & 49.6 & 3.2 & 46.8 & 371 & 21.3 & 0 & 20.8 & 52.6 & 85.5 & 103.6 & 15 & 241.7 & 124.6 & 69.1 & 26.1 & 11.3 & 83.4 & 118.1 & 196.5 & 20.3 & 48.2 & 3.2 & 2.3 & 18.4 & 1.1 & 0 & 1.1 & 2.8 & 4.4 & 5 & 0.5 & 13 & 5.8 & 9.8 & 23.1 & 2.2 & 1.9 \\[1em]
250 & 21.9 & 20.9 & 51.1 & 3.3 & 47.1 & 378 & 20.8 & 0 & 20.3 & 51.2 & 83.9 & 104.6 & 24.3 & 229.6 & 127.2 & 70.1 & 26.1 & 11.5 & 83.4 & 120.8 & 197.5 & 20.6 & 50.2 & 3.2 & 2.3 & 18.7 & 1.1 & 0 & 1.1 & 2.7 & 4.3 & 5.1 & 0.9 & 12.2 & 5.9 & 9.8 & 23.4 & 2.1 & 1.9 \\
260 & 22.2 & 21.3 & 52.8 & 3.4 & 46.9 & 380.6 & 20 & 0 & 19.5 & 48.9 & 80.4 & 102.9 & 36.5 & 213.9 & 128.3 & 70.5 & 25.7 & 11.5 & 82.2 & 122.1 & 196.1 & 20.4 & 48.1 & 3.2 & 2.8 & 22.4 & 1.3 & 0 & 1.3 & 3.3 & 5.4 & 6.3 & 0.3 & 15.7 & 7.1 & 11.9 & 23.8 & 2.1 & 1.9 \\
270 & 22.5 & 21.6 & 54.4 & 3.4 & 47.3 & 387.4 & 19.4 & 0 & 18.9 & 47.4 & 77.9 & 102 & 48.9 & 203.5 & 130.8 & 71.8 & 25.7 & 11.7 & 82.2 & 124.8 & 197.4 & 21.5 & 51.1 & 3.5 & 2.1 & 17.6 & 1 & 0 & 0.9 & 2.3 & 3.9 & 4.6 & 1.5 & 10.4 & 5.6 & 8.8 & 24.1 & 2.1 & 1.9 \\
280 & 22.8 & 21.9 & 55.9 & 3.5 & 47.5 & 393.2 & 18.7 & 0 & 18.2 & 45.7 & 75 & 99.9 & 62.4 & 193.9 & 133 & 72.8 & 25.6 & 11.8 & 81.9 & 127.1 & 198.1 & 21.4 & 54.2 & 3.4 & 2.2 & 17.8 & 0.9 & 0 & 0.9 & 2.2 & 3.7 & 4.8 & 2 & 9.6 & 5.7 & 9.2 & 24.4 & 2 & 2 \\
290 & 23 & 22.2 & 57.3 & 3.5 & 47.8 & 398.3 & 18.1 & 0 & 17.6 & 44 & 72.2 & 97.2 & 75.7 & 185.2 & 135 & 73.8 & 25.6 & 11.9 & 81.6 & 129.2 & 198.8 & 22.3 & 56.3 & 3.6 & 2.1 & 17.9 & 0.8 & 0 & 0.8 & 2 & 3.3 & 4.4 & 3.1 & 8.7 & 5.8 & 8.7 & 24.7 & 1.9 & 2 \\[1em]
300 & 23.2 & 22.4 & 58.8 & 3.6 & 47.6 & 399.8 & 17.2 & 0 & 16.8 & 42.1 & 69 & 93 & 88.6 & 175.7 & 135.8 & 74.2 & 25.3 & 12 & 80.6 & 130.1 & 197.7 & 22.4 & 57.9 & 3.6 & 2.5 & 20.9 & 0.9 & 0 & 0.9 & 2.3 & 3.8 & 5.1 & 4 & 9.6 & 6.8 & 10.2 & 25 & 1.9 & 1.9 \\
   \hline
\end{tabular}
\end{adjustbox}
\end{table}
\begin{table}[ht]
  \begin{adjustbox}{angle=90,height=\textheight}
\centering
\begin{tabular}{rrrrrrrrrrrrrrrrrrrrrr|rrrrrrrrrrrrrrr|rrr}
  \hline
age & ho & hg & dg & hKa & g & v & s.NH & 1b & 2a & 2b & 3a & 3b & 4+ & n & BmS & BmA & BmN & BmR & BmW & BmHarv & BmResid & hgAus & dgAus & hKaAus & gAus & vAus & s.NHAus & 1bAus & 2aAus & 2bAus & 3aAus & 3bAus & 4+Aus & nAus & BmHarvAus & BmResidAus & hoRef & lfz & dgz \\
\hline
0 & 0 & 0 & 0 & 0 & 0 & 0 & 0 & 0 & 0 & 0 & 0 & 0 & 0 & 2200 & 0 & 0 & 0 & 0 & 0 & 0 & 0 & 0 & 0 & 0 & 0 & 0 & 0 & 0 & 0 & 0 & 0 & 0 & 0 & 0 & 0 & 0 & 0 & 0 & 0 \\
10 & 0.5 & 0 & 0 & 0 & 0 & 0 & 0 & 0 & 0 & 0 & 0 & 0 & 0 & 1931.8 & 0 & 0 & 0 & 0 & 0 & 0 & 0 & 0 & 0 & 0 & 0 & 0 & 0 & 0 & 0 & 0 & 0 & 0 & 0 & 268.2 & 0 & 0 & 0.5 & 0 & 0 \\
20 & 2.3 & 2.2 & 4.7 & 0.1 & 1.7 & 1.9 & 0 & 0 & 0 & 0 & 0 & 0 & 0 & 960.8 & 0.7 & 4.8 & 4.3 & 0.1 & 1.4 & 0 & 11.2 & 2.1 & 4.6 & 0.1 & 1.3 & 1.4 & 0 & 0 & 0 & 0 & 0 & 0 & 0 & 944.1 & 0 & 8.9 & 2.3 & 0 & 0 \\
30 & 4.7 & 4.3 & 9.4 & 0.3 & 6.7 & 15.9 & 0.9 & 0 & 0 & 0 & 0 & 0 & 0 & 960 & 5.3 & 13.4 & 9.8 & 0.7 & 6.5 & 0.4 & 35.3 & 2.6 & 6 & 0.1 & 0 & 0 & 0 & 0 & 0 & 0 & 0 & 0 & 0 & 0.8 & 0 & 0 & 5 & 0.1 & 0 \\
40 & 7.2 & 6.6 & 13.1 & 0.7 & 13 & 41 & 24.5 & 0 & 0 & 0 & 0 & 0 & 0 & 955 & 13.3 & 21 & 14.3 & 1.6 & 15.3 & 9.8 & 55.7 & 5.6 & 11.4 & 0.5 & 0 & 0.1 & 0.1 & 0 & 0 & 0 & 0 & 0 & 0 & 5 & 0 & 0.2 & 7.7 & 2.4 & 0.6 \\[1em]
50 & 9.3 & 8.7 & 16.6 & 1.1 & 11.6 & 45.8 & 30.4 & 0 & 0 & 0 & 0 & 0 & 0 & 533.3 & 14.8 & 16.7 & 10.7 & 1.7 & 15.8 & 12 & 47.7 & 7.5 & 14.4 & 0.9 & 6.9 & 24.4 & 15.3 & 0 & 0 & 0 & 0 & 0 & 0 & 421.7 & 6.1 & 28.8 & 10.1 & 2.1 & 0.9 \\
60 & 11.3 & 10.5 & 19.8 & 1.5 & 16.3 & 76.3 & 52.2 & 1.1 & 0.1 & 0 & 0 & 0 & 0 & 530.8 & 24.7 & 22.1 & 13.4 & 2.7 & 24.5 & 20.9 & 66.6 & 9.1 & 17.4 & 1.2 & 0.1 & 0.2 & 0.2 & 0 & 0 & 0 & 0 & 0 & 0 & 2.5 & 0.1 & 0.2 & 12.2 & 2.3 & 1.2 \\
70 & 13 & 12.1 & 22.7 & 1.8 & 21.3 & 113.3 & 64 & 3.1 & 14.9 & 0 & 0 & 0 & 0 & 529.2 & 36.7 & 27.8 & 16 & 3.9 & 34.2 & 31.9 & 86.7 & 11.2 & 22 & 1.6 & 0.1 & 0.3 & 0.2 & 0 & 0 & 0 & 0 & 0 & 0 & 1.7 & 0.1 & 0.3 & 13.9 & 2.9 & 1.4 \\
80 & 14.5 & 13.6 & 25.3 & 2.2 & 26.4 & 155.1 & 62.2 & 4.6 & 47.5 & 0.5 & 0 & 0 & 0 & 526.7 & 50.4 & 33.5 & 18.6 & 5.2 & 44.3 & 44.6 & 107.4 & 12.4 & 23.3 & 1.9 & 0.1 & 0.6 & 0.3 & 0 & 0.1 & 0 & 0 & 0 & 0 & 2.5 & 0.2 & 0.4 & 15.5 & 3.3 & 1.6 \\
90 & 15.8 & 14.8 & 27.7 & 2.4 & 31.2 & 198.7 & 61.6 & 3.9 & 63.5 & 20.4 & 0 & 0 & 0 & 519.2 & 64.8 & 38.9 & 20.8 & 6.6 & 54 & 58 & 127 & 14.3 & 26.7 & 2.3 & 0.4 & 2.6 & 0.9 & 0.1 & 0.9 & 0.1 & 0 & 0 & 0 & 7.5 & 0.8 & 1.7 & 16.8 & 3.7 & 1.9 \\[1em]
100 & 17 & 16 & 29.9 & 2.7 & 35.9 & 244.3 & 59.2 & 2.6 & 65.4 & 56.1 & 2.6 & 0 & 0 & 510.8 & 79.9 & 44.2 & 23 & 8 & 63.5 & 72.3 & 146.2 & 15.6 & 28.6 & 2.7 & 0.5 & 3.5 & 1 & 0.1 & 1.1 & 0.6 & 0 & 0 & 0 & 8.3 & 1 & 2.1 & 18 & 3.9 & 2.1 \\
110 & 18.1 & 17 & 32 & 2.9 & 40.1 & 288.9 & 53.9 & 1.6 & 62.1 & 83.3 & 21.3 & 0 & 0 & 498.3 & 94.8 & 49.1 & 24.8 & 9.3 & 72.1 & 86.5 & 163.6 & 16.2 & 30.1 & 2.8 & 0.9 & 6.1 & 1.5 & 0.1 & 1.6 & 1.5 & 0.1 & 0 & 0 & 12.5 & 1.8 & 3.6 & 19.1 & 4.1 & 2.3 \\
120 & 18.9 & 17.9 & 34 & 3.1 & 35.8 & 269.4 & 39.8 & 0.9 & 47.7 & 76.8 & 43.4 & 0.2 & 0 & 395 & 88.6 & 43.8 & 21.5 & 8.6 & 65 & 81.4 & 146.1 & 17.6 & 33.3 & 3 & 9 & 66.7 & 10.6 & 0.3 & 12.5 & 18.8 & 9.4 & 0 & 0 & 103.3 & 20.1 & 36.6 & 20 & 3.8 & 2.4 \\
130 & 19.8 & 18.7 & 36 & 3.3 & 38.9 & 304.2 & 35.6 & 0.5 & 42.9 & 78.4 & 68.2 & 12.1 & 0 & 381.7 & 100.3 & 47.9 & 22.8 & 9.6 & 71.1 & 92.7 & 159.1 & 18.5 & 34.6 & 3.3 & 1.3 & 9.7 & 1.4 & 0 & 1.7 & 2.9 & 1.6 & 0 & 0 & 13.3 & 3 & 5.1 & 20.9 & 3.6 & 2.5 \\
140 & 20.6 & 19.5 & 38 & 3.4 & 42 & 339.4 & 32.3 & 0.3 & 37.7 & 76.8 & 86 & 34 & 0 & 370 & 112.3 & 51.9 & 24.1 & 10.7 & 77 & 104.3 & 171.7 & 19.1 & 36.3 & 3.4 & 1.2 & 9.6 & 1.1 & 0 & 1.4 & 2.5 & 2.4 & 0.2 & 0 & 11.7 & 2.9 & 4.9 & 21.7 & 3.7 & 2.6 \\[1em]
150 & 21.3 & 20.2 & 40 & 3.6 & 43.7 & 363.6 & 29.4 & 0.2 & 32.7 & 73.6 & 94.9 & 56.9 & 0 & 347.5 & 120.6 & 54.3 & 24.5 & 11.4 & 80.1 & 112.5 & 178.3 & 19.6 & 38 & 3.5 & 2.5 & 20.6 & 2 & 0 & 2.3 & 4.8 & 5.1 & 1.9 & 0 & 22.5 & 6.3 & 10.3 & 22.4 & 3.7 & 2.6 \\
160 & 21.9 & 20.8 & 41.9 & 3.7 & 44.5 & 379.7 & 27.5 & 0.1 & 29.2 & 70.3 & 98.8 & 76 & 0.1 & 323.3 & 126.2 & 55.7 & 24.6 & 11.8 & 81.5 & 118.3 & 181.6 & 20.9 & 41.1 & 3.8 & 3.2 & 27.5 & 2 & 0 & 2.2 & 5.2 & 7.2 & 5.2 & 0 & 24.2 & 8.5 & 12.9 & 23.1 & 3.6 & 2.7 \\
170 & 22.5 & 21.5 & 43.8 & 3.8 & 46.3 & 403.8 & 26.8 & 0 & 27.3 & 68.3 & 101.9 & 96.1 & 2.3 & 307.5 & 134.6 & 58.4 & 25.1 & 12.5 & 84.4 & 126.5 & 188.4 & 20.9 & 41.5 & 3.8 & 2.1 & 18.4 & 1.3 & 0 & 1.4 & 3.4 & 4.6 & 3.8 & 0 & 15.8 & 5.7 & 8.7 & 23.7 & 3.5 & 2.7 \\
180 & 23.1 & 22.1 & 45.7 & 4 & 47.8 & 425.5 & 26.3 & 0 & 26.1 & 66.2 & 103.8 & 110.4 & 8.6 & 291.7 & 142.2 & 60.8 & 25.6 & 13.1 & 86.7 & 134.1 & 194.1 & 21.5 & 42.8 & 3.9 & 2.3 & 20 & 1.3 & 0 & 1.4 & 3.5 & 5.1 & 4.7 & 0 & 15.8 & 6.3 & 9.2 & 24.3 & 3.5 & 2.8 \\
190 & 23.7 & 22.6 & 47.5 & 4.1 & 49 & 444.4 & 25.8 & 0 & 25.4 & 64.3 & 103.8 & 117.9 & 20.8 & 276.7 & 148.8 & 62.9 & 25.9 & 13.6 & 88.3 & 140.8 & 198.6 & 22.3 & 45.7 & 4.1 & 2.4 & 22 & 1.4 & 0 & 1.4 & 3.5 & 5.4 & 5.6 & 0.4 & 15 & 6.9 & 9.9 & 24.9 & 3.4 & 2.8 \\[1em]
200 & 24.2 & 23.1 & 49.2 & 4.2 & 51.2 & 472.7 & 25.8 & 0 & 25.3 & 63.7 & 104.1 & 124.7 & 38.4 & 269.2 & 158.7 & 66.3 & 26.7 & 14.4 & 91.7 & 150.5 & 207.2 & 22.7 & 47.2 & 4.1 & 1.3 & 12 & 0.7 & 0 & 0.7 & 1.8 & 3 & 3.1 & 0.3 & 7.5 & 3.8 & 5.3 & 25.4 & 3.4 & 2.8 \\
210 & 24.6 & 23.6 & 51 & 4.3 & 52.4 & 490.3 & 25 & 0 & 24.4 & 61.4 & 100.8 & 125.9 & 60 & 256.7 & 164.9 & 68.3 & 27 & 14.9 & 92.9 & 156.9 & 211.1 & 23.6 & 49.2 & 4.4 & 2.4 & 22.4 & 1.2 & 0 & 1.2 & 3 & 4.9 & 5.9 & 1.9 & 12.5 & 7.1 & 9.5 & 25.9 & 3.4 & 2.9 \\
220 & 25 & 24 & 52.6 & 4.4 & 52.6 & 499.3 & 23.8 & 0 & 23.2 & 58.3 & 95.8 & 123.4 & 81.5 & 242.5 & 168.3 & 69.2 & 26.8 & 15.1 & 92.6 & 160.4 & 211.6 & 24.3 & 52.9 & 4.5 & 3.1 & 29.8 & 1.4 & 0 & 1.4 & 3.4 & 5.6 & 7.4 & 5 & 14.2 & 9.6 & 12.4 & 26.4 & 3.3 & 2.9 \\
230 & 25.4 & 24.5 & 54.2 & 4.5 & 54.1 & 519.5 & 23.1 & 0 & 22.5 & 56.3 & 92.5 & 122.1 & 107.1 & 234.2 & 175.5 & 71.7 & 27.3 & 15.7 & 94.2 & 167.6 & 216.8 & 24.5 & 52.7 & 4.6 & 1.8 & 17.6 & 0.8 & 0 & 0.8 & 2 & 3.3 & 4.3 & 3 & 8.3 & 5.7 & 7.2 & 26.8 & 3.2 & 2.9 \\
240 & 25.8 & 24.9 & 55.9 & 4.5 & 53.6 & 521.2 & 21.5 & 0 & 21 & 52.6 & 86.3 & 115.4 & 129.3 & 218.3 & 176.4 & 71.8 & 26.8 & 15.7 & 92.5 & 168.8 & 214.4 & 24.7 & 54 & 4.6 & 3.6 & 35.1 & 1.6 & 0 & 1.5 & 3.8 & 6.3 & 8.4 & 7 & 15.8 & 11.3 & 14.4 & 27.2 & 3.1 & 2.9 \\[1em]
250 & 26.2 & 25.3 & 57.7 & 4.6 & 52.1 & 511.9 & 19.7 & 0 & 19.1 & 48 & 78.7 & 106 & 148 & 199.2 & 173.6 & 70.5 & 25.8 & 15.3 & 88.9 & 166.5 & 207.7 & 25 & 55.3 & 4.6 & 4.6 & 44.7 & 1.9 & 0 & 1.8 & 4.6 & 7.6 & 10.2 & 10.3 & 19.2 & 14.5 & 18.2 & 27.7 & 3 & 2.9 \\
260 & 26.5 & 25.6 & 59.4 & 4.7 & 51.2 & 507.4 & 18.3 & 0 & 17.8 & 44.6 & 73.1 & 98.8 & 164.2 & 185 & 172.4 & 69.9 & 25.2 & 15.2 & 86.3 & 165.7 & 203.3 & 25.6 & 59.1 & 4.7 & 3.8 & 37.9 & 1.4 & 0 & 1.4 & 3.4 & 5.6 & 7.6 & 11.7 & 14.2 & 12.4 & 15.2 & 28 & 2.8 & 2.9 \\
270 & 26.8 & 26 & 61.2 & 4.8 & 52.4 & 524.1 & 17.6 & 0 & 17.2 & 43 & 70.5 & 95.4 & 187.7 & 178.3 & 178.5 & 72.2 & 25.5 & 15.6 & 87.3 & 171.8 & 207.3 & 25.2 & 55.7 & 4.7 & 1.6 & 16 & 0.7 & 0 & 0.6 & 1.6 & 2.6 & 3.6 & 3.9 & 6.7 & 5.2 & 6.4 & 28.4 & 2.8 & 2.9 \\
280 & 27.1 & 26.4 & 62.9 & 4.9 & 54 & 545.3 & 17.2 & 0 & 16.8 & 42 & 68.9 & 93.3 & 211.7 & 174.2 & 186.1 & 75.2 & 26.1 & 16.2 & 88.9 & 179.3 & 213.2 & 25.5 & 58.8 & 4.7 & 1.1 & 11 & 0.4 & 0 & 0.4 & 1 & 1.6 & 2.2 & 3.4 & 4.2 & 3.6 & 4.4 & 28.8 & 2.8 & 2.9 \\
290 & 27.4 & 26.7 & 64.5 & 5 & 54.2 & 550.3 & 16.4 & 0 & 16 & 40 & 65.6 & 88.9 & 228.1 & 165.8 & 188.1 & 76.1 & 26 & 16.3 & 88.1 & 181.6 & 213 & 26.8 & 63.2 & 5.1 & 2.6 & 26.6 & 0.8 & 0 & 0.8 & 2 & 3.3 & 4.5 & 10.6 & 8.3 & 8.8 & 10.1 & 29.1 & 2.7 & 2.9 \\[1em]
300 & 27.7 & 27 & 66.1 & 5 & 53.8 & 551.1 & 15.5 & 0 & 15.1 & 37.8 & 62 & 84 & 242.1 & 156.7 & 188.7 & 76.2 & 25.6 & 16.3 & 86.4 & 182.4 & 210.8 & 26.4 & 64.4 & 4.9 & 3 & 29.8 & 0.9 & 0 & 0.9 & 2.2 & 3.6 & 4.9 & 12.1 & 9.2 & 9.8 & 11.7 & 29.4 & 2.6 & 2.9 \\
   \hline
\end{tabular}
\end{adjustbox}
\end{table}
\begin{table}[ht]
  \begin{adjustbox}{angle=90,height=\textheight}
\centering
\begin{tabular}{rrrrrrrrrrrrrrrrrrrrrr|rrrrrrrrrrrrrrr|rrr}
  \hline
age & ho & hg & dg & hKa & g & v & s.NH & 1b & 2a & 2b & 3a & 3b & 4+ & n & BmS & BmA & BmN & BmR & BmW & BmHarv & BmResid & hgAus & dgAus & hKaAus & gAus & vAus & s.NHAus & 1bAus & 2aAus & 2bAus & 3aAus & 3bAus & 4+Aus & nAus & BmHarvAus & BmResidAus & hoRef & lfz & dgz \\
\hline
0 & 0 & 0 & 0 & 0 & 0 & 0 & 0 & 0 & 0 & 0 & 0 & 0 & 0 & 2040 & 0 & 0 & 0 & 0 & 0 & 0 & 0 & 0 & 0 & 0 & 0 & 0 & 0 & 0 & 0 & 0 & 0 & 0 & 0 & 0 & 0 & 0 & 0 & 0 & 0 \\
10 & 0.7 & 0 & 0 & 0 & 0 & 0 & 0 & 0 & 0 & 0 & 0 & 0 & 0 & 1757.9 & 0 & 0 & 0 & 0 & 0 & 0 & 0 & 0 & 0 & 0 & 0 & 0 & 0 & 0 & 0 & 0 & 0 & 0 & 0 & 282.1 & 0 & 0 & 0.7 & 0 & 0 \\
20 & 2.9 & 2.7 & 6.2 & 0.1 & 2.9 & 5.2 & 0 & 0 & 0 & 0 & 0 & 0 & 0 & 932.2 & 1.8 & 7.3 & 5.9 & 0.3 & 2.4 & 0 & 17.6 & 2.6 & 6.1 & 0.1 & 2.1 & 3.7 & 0 & 0 & 0 & 0 & 0 & 0 & 0 & 797.5 & 0 & 13 & 3 & 0 & 0 \\
30 & 5.8 & 5.3 & 11.2 & 0.5 & 9.2 & 25.2 & 13.1 & 0 & 0 & 0 & 0 & 0 & 0 & 929.8 & 8.2 & 16.6 & 11.6 & 1.1 & 9.9 & 5.3 & 42.1 & 3.7 & 8.5 & 0.2 & 0 & 0 & 0 & 0 & 0 & 0 & 0 & 0 & 0 & 2.5 & 0 & 0.1 & 6.2 & 1.3 & 0.4 \\
40 & 8.8 & 8 & 15.2 & 1 & 16.7 & 62.6 & 40.2 & 0 & 0 & 0 & 0 & 0 & 0 & 924 & 20.2 & 24.5 & 16.2 & 2.4 & 21.6 & 15.9 & 68.9 & 7 & 13.9 & 0.8 & 0.1 & 0.3 & 0.2 & 0 & 0 & 0 & 0 & 0 & 0 & 5.8 & 0.1 & 0.4 & 9.4 & 2.7 & 1 \\[1em]
50 & 11.2 & 10.5 & 18.9 & 1.6 & 14 & 66.5 & 45.3 & 0.8 & 0 & 0 & 0 & 0 & 0 & 497.6 & 21.5 & 18.4 & 11.5 & 2.4 & 20.6 & 18 & 56.2 & 9.1 & 16.5 & 1.3 & 9.2 & 38.6 & 25.7 & 0 & 0 & 0 & 0 & 0 & 0 & 426.4 & 10.1 & 36.9 & 12.2 & 3.2 & 1.4 \\
60 & 13.4 & 12.6 & 22.4 & 2.1 & 19.5 & 109.6 & 64.4 & 2.9 & 12 & 0 & 0 & 0 & 0 & 496.8 & 35.5 & 24.1 & 14.3 & 3.8 & 31.2 & 30.8 & 78 & 11.8 & 22 & 1.8 & 0 & 0.2 & 0.1 & 0 & 0 & 0 & 0 & 0 & 0 & 0.8 & 0 & 0.1 & 14.5 & 3.3 & 1.8 \\
70 & 15.4 & 14.5 & 25.5 & 2.5 & 25.2 & 160.3 & 62.8 & 4.7 & 51 & 0.5 & 0 & 0 & 0 & 495.1 & 52.1 & 30 & 17 & 5.4 & 42.4 & 46.2 & 100.7 & 14.5 & 26.2 & 2.5 & 0.1 & 0.6 & 0.2 & 0 & 0.2 & 0 & 0 & 0 & 0 & 1.7 & 0.2 & 0.4 & 16.5 & 4 & 2.1 \\
80 & 17.1 & 16.1 & 28.3 & 2.9 & 30.9 & 216.5 & 63.6 & 3.6 & 67.3 & 29.3 & 0 & 0 & 0 & 491.8 & 70.6 & 35.9 & 19.6 & 7.1 & 53.8 & 63.6 & 123.4 & 15.2 & 27.2 & 2.7 & 0.2 & 1.3 & 0.4 & 0 & 0.4 & 0.1 & 0 & 0 & 0 & 3.3 & 0.4 & 0.8 & 18.2 & 4.6 & 2.4 \\
90 & 18.6 & 17.5 & 30.8 & 3.3 & 36.2 & 274.6 & 59 & 2.1 & 66.4 & 76.1 & 7 & 0 & 0 & 485.2 & 89.9 & 41.5 & 21.9 & 8.9 & 64.6 & 81.9 & 144.9 & 17.4 & 30.8 & 3.2 & 0.5 & 3.7 & 0.8 & 0 & 0.9 & 1.1 & 0 & 0 & 0 & 6.6 & 1.1 & 2 & 19.7 & 5 & 2.7 \\[1em]
100 & 19.9 & 18.8 & 33.2 & 3.6 & 41.1 & 332.1 & 53.5 & 1.3 & 63.6 & 96.2 & 42.9 & 0 & 0 & 474.4 & 109.1 & 46.6 & 23.9 & 10.7 & 74.4 & 100.2 & 164.4 & 18.7 & 32.4 & 3.6 & 0.9 & 7.2 & 1.3 & 0 & 1.5 & 2.4 & 0.4 & 0 & 0 & 10.8 & 2.2 & 3.5 & 21 & 5.2 & 2.9 \\
110 & 21.1 & 20 & 35.5 & 3.9 & 45.5 & 387.3 & 48.2 & 0.8 & 58.6 & 102.8 & 87.2 & 5.4 & 0 & 460.3 & 127.6 & 51.3 & 25.5 & 12.3 & 83 & 118.1 & 181.6 & 19.7 & 34.4 & 3.9 & 1.3 & 11 & 1.6 & 0 & 2 & 3.1 & 1.8 & 0 & 0 & 14.1 & 3.3 & 5.1 & 22.2 & 5.4 & 3.2 \\
120 & 22.1 & 21 & 37.7 & 4.1 & 39.4 & 350 & 34.5 & 0.3 & 40.5 & 82 & 89 & 29.8 & 0 & 353.5 & 115.7 & 44.4 & 21.5 & 11 & 72.1 & 107.7 & 157 & 20.5 & 36.6 & 4 & 11.2 & 98 & 10.7 & 0.1 & 12.9 & 24.5 & 24.9 & 3.9 & 0 & 106.8 & 30 & 44.7 & 23.2 & 5 & 3.3 \\
130 & 23 & 21.9 & 39.9 & 4.3 & 43.2 & 398.7 & 32.3 & 0.2 & 35.8 & 81.1 & 102.6 & 64.4 & 0 & 346.1 & 132.2 & 49 & 23 & 12.5 & 79.3 & 123.7 & 172.2 & 21.5 & 39.1 & 4.2 & 0.9 & 8.1 & 0.7 & 0 & 0.8 & 1.8 & 2.2 & 0.9 & 0 & 7.5 & 2.5 & 3.6 & 24.2 & 4.7 & 3.4 \\
140 & 23.9 & 22.8 & 42.1 & 4.5 & 45.6 & 434 & 30.5 & 0.1 & 32 & 77.8 & 110.9 & 91.1 & 4.2 & 327.9 & 144.3 & 52 & 23.8 & 13.5 & 83.5 & 135.7 & 181.4 & 22.3 & 39.8 & 4.5 & 2.2 & 21.2 & 1.8 & 0 & 2 & 4.3 & 5.6 & 3.1 & 0 & 18.2 & 6.6 & 8.9 & 25.1 & 4.7 & 3.5 \\[1em]
150 & 24.7 & 23.6 & 44.2 & 4.7 & 47.8 & 467.4 & 29.5 & 0.1 & 29.9 & 74.9 & 114.7 & 109.1 & 17 & 311.3 & 155.9 & 54.9 & 24.5 & 14.4 & 87.1 & 147.1 & 189.7 & 23.5 & 42.4 & 4.8 & 2.3 & 22.8 & 1.6 & 0 & 1.7 & 4 & 5.7 & 4.7 & 0.6 & 16.6 & 7.1 & 9.1 & 25.9 & 4.7 & 3.6 \\
160 & 25.4 & 24.3 & 46.2 & 4.8 & 50.7 & 507.5 & 29.2 & 0 & 29 & 73.3 & 116.3 & 124.4 & 36.7 & 302.2 & 169.7 & 58.6 & 25.5 & 15.6 & 91.8 & 160.7 & 200.5 & 24.7 & 46.3 & 5 & 1.5 & 15.5 & 0.9 & 0 & 0.9 & 2.3 & 3.6 & 3.5 & 1.4 & 9.1 & 4.9 & 6 & 26.7 & 4.7 & 3.7 \\
170 & 26.1 & 25 & 48.2 & 5 & 53 & 542.2 & 28.4 & 0 & 27.9 & 70.6 & 114.2 & 133.7 & 63.9 & 290.6 & 181.7 & 61.8 & 26.3 & 16.5 & 95.3 & 172.7 & 208.9 & 25 & 46.8 & 5.1 & 2 & 20.5 & 1.1 & 0 & 1.1 & 2.8 & 4.5 & 4.9 & 2.1 & 11.6 & 6.5 & 7.8 & 27.4 & 4.6 & 3.7 \\
180 & 26.7 & 25.7 & 50.1 & 5.2 & 54.3 & 566.5 & 27.1 & 0 & 26.5 & 66.7 & 109 & 134.9 & 96.1 & 274.9 & 190.4 & 63.8 & 26.5 & 17.2 & 96.7 & 181.4 & 213.3 & 25.3 & 48.3 & 5.1 & 2.9 & 29.8 & 1.5 & 0 & 1.5 & 3.8 & 6.2 & 7.3 & 3.7 & 15.7 & 9.5 & 11.3 & 28 & 4.6 & 3.8 \\
190 & 27.3 & 26.3 & 52.2 & 5.3 & 55.2 & 586.6 & 25.5 & 0 & 24.9 & 62.5 & 102.6 & 132.3 & 130.4 & 258.3 & 197.6 & 65.5 & 26.6 & 17.7 & 97.4 & 188.8 & 216.1 & 25.9 & 49.1 & 5.3 & 3.1 & 33 & 1.6 & 0 & 1.6 & 4 & 6.6 & 7.5 & 5.4 & 16.6 & 10.5 & 12.2 & 28.6 & 4.5 & 3.8 \\[1em]
200 & 27.9 & 26.9 & 54.1 & 5.4 & 55.3 & 596.7 & 23.7 & 0 & 23.1 & 58 & 95.2 & 125.9 & 162.1 & 240.1 & 201.5 & 66.2 & 26.4 & 18 & 96.5 & 193 & 215.5 & 26.6 & 51.9 & 5.5 & 3.8 & 41.3 & 1.8 & 0 & 1.8 & 4.4 & 7.3 & 9 & 9.3 & 18.2 & 13.3 & 14.9 & 29.2 & 4.3 & 3.8 \\
210 & 28.4 & 27.4 & 56.1 & 5.6 & 55.2 & 604.3 & 22.1 & 0 & 21.5 & 54 & 88.5 & 118.7 & 190.8 & 223.5 & 204.6 & 66.7 & 26 & 18.1 & 95.2 & 196.4 & 214.1 & 27 & 54.3 & 5.5 & 3.8 & 41.5 & 1.6 & 0 & 1.6 & 4 & 6.6 & 8.8 & 11.4 & 16.6 & 13.4 & 14.9 & 29.8 & 4.2 & 3.8 \\
220 & 28.9 & 28 & 58 & 5.7 & 56.4 & 626.4 & 21.1 & 0 & 20.6 & 51.5 & 84.5 & 114.2 & 223.3 & 213.6 & 212.5 & 68.9 & 26.3 & 18.7 & 96.2 & 204.5 & 218.2 & 27.5 & 55 & 5.7 & 2.3 & 25.8 & 1 & 0 & 1 & 2.4 & 3.9 & 5.3 & 7.6 & 9.9 & 8.4 & 9 & 30.3 & 4.1 & 3.8 \\
230 & 29.4 & 28.5 & 59.9 & 5.8 & 57.6 & 647.2 & 20.2 & 0 & 19.7 & 49.3 & 80.9 & 109.6 & 254 & 204.5 & 220.1 & 70.9 & 26.6 & 19.3 & 96.9 & 212.2 & 221.7 & 28 & 57.5 & 5.7 & 2.3 & 26.1 & 0.9 & 0 & 0.9 & 2.2 & 3.6 & 4.8 & 9.1 & 9.1 & 8.5 & 9 & 30.8 & 4 & 3.9 \\
240 & 29.8 & 28.9 & 61.8 & 5.9 & 57 & 648 & 18.9 & 0 & 18.3 & 45.9 & 75.3 & 102.1 & 275.2 & 190.4 & 220.8 & 70.9 & 26.1 & 19.3 & 94.7 & 213.3 & 218.5 & 28.5 & 59.9 & 5.9 & 3.9 & 44.4 & 1.4 & 0 & 1.4 & 3.4 & 5.6 & 7.5 & 17.4 & 14.1 & 14.6 & 15.1 & 31.3 & 3.9 & 3.9 \\[1em]
250 & 30.2 & 29.4 & 63.6 & 6 & 57.3 & 658.2 & 17.9 & 0 & 17.4 & 43.6 & 71.4 & 96.8 & 298.4 & 180.5 & 224.7 & 71.8 & 26 & 19.5 & 93.9 & 217.5 & 218.4 & 28.7 & 62.1 & 5.8 & 3 & 33.5 & 1 & 0 & 1 & 2.4 & 3.9 & 5.3 & 14.1 & 9.9 & 11 & 11.5 & 31.7 & 3.7 & 3.9 \\
260 & 30.6 & 29.8 & 65.4 & 6.1 & 56.9 & 660.2 & 16.8 & 0 & 16.3 & 41 & 67.1 & 91 & 315.9 & 169.7 & 225.9 & 72 & 25.6 & 19.5 & 92.1 & 218.9 & 216.2 & 29.8 & 64.2 & 6.2 & 3.5 & 40.5 & 1.1 & 0 & 1 & 2.6 & 4.3 & 5.8 & 18.8 & 10.8 & 13.4 & 13.2 & 32.2 & 3.6 & 3.8 \\
270 & 30.9 & 30.2 & 67.2 & 6.3 & 57.8 & 676.1 & 16.1 & 0 & 15.7 & 39.4 & 64.5 & 87.5 & 339.3 & 163.1 & 231.8 & 73.8 & 25.8 & 20 & 92.1 & 225 & 218.4 & 29.7 & 64.9 & 6.1 & 2.2 & 25.3 & 0.7 & 0 & 0.6 & 1.6 & 2.6 & 3.6 & 11.9 & 6.6 & 8.4 & 8.3 & 32.6 & 3.5 & 3.8 \\
280 & 31.3 & 30.6 & 69 & 6.4 & 58.7 & 692.4 & 15.6 & 0 & 15.1 & 38 & 62.2 & 84.4 & 361.9 & 157.3 & 237.8 & 75.6 & 26 & 20.4 & 92.3 & 231.2 & 221 & 30.6 & 66.8 & 6.4 & 2 & 24.1 & 0.6 & 0 & 0.6 & 1.4 & 2.3 & 3.1 & 12.1 & 5.8 & 8 & 7.6 & 33 & 3.5 & 3.8 \\
290 & 31.6 & 31 & 70.7 & 6.5 & 59.2 & 703.4 & 14.9 & 0 & 14.5 & 36.4 & 59.6 & 80.8 & 381.2 & 150.7 & 242.1 & 76.8 & 26 & 20.7 & 91.7 & 235.6 & 221.6 & 30.3 & 69.7 & 6.2 & 2.5 & 28.6 & 0.7 & 0 & 0.6 & 1.6 & 2.6 & 3.6 & 14.8 & 6.6 & 9.5 & 9.3 & 33.4 & 3.4 & 3.8 \\[1em]
300 & 31.9 & 31.4 & 72.5 & 6.6 & 58.8 & 704 & 14.1 & 0 & 13.7 & 34.4 & 56.3 & 76.4 & 394.1 & 142.4 & 242.7 & 76.9 & 25.6 & 20.7 & 89.7 & 236.6 & 219 & 30.7 & 70.3 & 6.4 & 3.2 & 37.7 & 0.8 & 0 & 0.8 & 2 & 3.3 & 4.4 & 20.1 & 8.3 & 12.6 & 12 & 33.7 & 3.3 & 3.8 \\
   \hline
\end{tabular}
\end{adjustbox}
\end{table}
\begin{table}[ht]
  \begin{adjustbox}{angle=90,height=\textheight}
\centering
\begin{tabular}{rrrrrrrrrrrrrrrrrrrrrr|rrrrrrrrrrrrrrr|rrr}
  \hline
age & ho & hg & dg & hKa & g & v & s.NH & 1b & 2a & 2b & 3a & 3b & 4+ & n & BmS & BmA & BmN & BmR & BmW & BmHarv & BmResid & hgAus & dgAus & hKaAus & gAus & vAus & s.NHAus & 1bAus & 2aAus & 2bAus & 3aAus & 3bAus & 4+Aus & nAus & BmHarvAus & BmResidAus & hoRef & lfz & dgz \\
\hline
0 & 0 & 0 & 0 & 0 & 0 & 0 & 0 & 0 & 0 & 0 & 0 & 0 & 0 & 1880 & 0 & 0 & 0 & 0 & 0 & 0 & 0 & 0 & 0 & 0 & 0 & 0 & 0 & 0 & 0 & 0 & 0 & 0 & 0 & 0 & 0 & 0 & 0 & 0 & 0 \\
10 & 0.9 & 0 & 0 & 0 & 0 & 0 & 0 & 0 & 0 & 0 & 0 & 0 & 0 & 1583.1 & 0 & 0 & 0 & 0 & 0 & 0 & 0 & 0 & 0 & 0 & 0 & 0 & 0 & 0 & 0 & 0 & 0 & 0 & 0 & 296.9 & 0 & 0 & 0.9 & 0 & 0 \\
20 & 3.6 & 3.3 & 7.8 & 0.2 & 4.2 & 8.1 & 0 & 0 & 0 & 0 & 0 & 0 & 0 & 872.9 & 2.7 & 9.9 & 7.3 & 0.4 & 3.7 & 0 & 24 & 3.2 & 7.7 & 0.1 & 3 & 5.7 & 0 & 0 & 0 & 0 & 0 & 0 & 0 & 681.5 & 0 & 17.3 & 3.8 & 0 & 0 \\
30 & 7 & 6.4 & 13 & 0.7 & 11.6 & 35.8 & 21.2 & 0 & 0 & 0 & 0 & 0 & 0 & 871 & 11.6 & 19.2 & 13 & 1.4 & 13.6 & 8.5 & 50.2 & 5.3 & 11.6 & 0.4 & 0 & 0 & 0 & 0 & 0 & 0 & 0 & 0 & 0 & 1.9 & 0 & 0.1 & 7.6 & 2.1 & 0.7 \\
40 & 10.4 & 9.5 & 17.2 & 1.4 & 20 & 87.2 & 58.6 & 0.2 & 0 & 0 & 0 & 0 & 0 & 860.7 & 28.1 & 26.8 & 17.4 & 3.2 & 27.7 & 23.3 & 79.9 & 8.5 & 15.8 & 1.2 & 0.2 & 0.8 & 0.5 & 0 & 0 & 0 & 0 & 0 & 0 & 10.3 & 0.2 & 0.8 & 11.2 & 3.8 & 1.5 \\[1em]
50 & 13.1 & 12.3 & 21.2 & 2.1 & 16.4 & 90.9 & 60.8 & 1.8 & 2.4 & 0 & 0 & 0 & 0 & 463.5 & 29.4 & 19.8 & 12.2 & 3.2 & 25.5 & 25.6 & 64.4 & 10.8 & 18.6 & 1.7 & 10.7 & 53.1 & 36.2 & 0.6 & 0 & 0 & 0 & 0 & 0 & 397.3 & 14.5 & 42.2 & 14.3 & 4.3 & 2 \\
60 & 15.5 & 14.7 & 24.9 & 2.7 & 22.4 & 146.2 & 60.4 & 4.6 & 43.3 & 0 & 0 & 0 & 0 & 460.7 & 47.4 & 25.7 & 15 & 4.9 & 37.4 & 42.4 & 88 & 13.6 & 23.2 & 2.4 & 0.1 & 0.7 & 0.4 & 0 & 0.1 & 0 & 0 & 0 & 0 & 2.8 & 0.2 & 0.5 & 16.9 & 4.4 & 2.4 \\
70 & 17.6 & 16.7 & 28.2 & 3.2 & 28.6 & 210.4 & 62.9 & 3.6 & 67.1 & 25.7 & 0 & 0 & 0 & 457.9 & 68.5 & 31.7 & 17.7 & 6.9 & 49.7 & 62.4 & 112.2 & 15.2 & 26 & 2.8 & 0.1 & 1 & 0.4 & 0 & 0.3 & 0 & 0 & 0 & 0 & 2.8 & 0.3 & 0.6 & 19.1 & 5.2 & 2.8 \\
80 & 19.5 & 18.5 & 31.2 & 3.7 & 34.7 & 280.2 & 57.7 & 2.1 & 65.6 & 83.2 & 7 & 0 & 0 & 454.1 & 91.7 & 37.7 & 20.2 & 9.1 & 62 & 84.6 & 136.2 & 18.1 & 30 & 3.6 & 0.3 & 2.1 & 0.5 & 0 & 0.5 & 0.5 & 0 & 0 & 0 & 3.7 & 0.6 & 1 & 20.9 & 5.8 & 3.2 \\
90 & 21.1 & 20.1 & 33.9 & 4.1 & 40.4 & 352.3 & 51.9 & 1.2 & 63.2 & 101.2 & 56.9 & 0 & 0 & 447.6 & 115.8 & 43.4 & 22.5 & 11.3 & 73.4 & 107.7 & 158.7 & 19.7 & 32.6 & 4 & 0.5 & 4.7 & 0.8 & 0 & 1 & 1.6 & 0.2 & 0 & 0 & 6.5 & 1.4 & 2.1 & 22.6 & 6.2 & 3.5 \\[1em]
100 & 22.5 & 21.5 & 36.5 & 4.5 & 45.7 & 422.9 & 46.8 & 0.6 & 57 & 106.9 & 110.3 & 11.1 & 0 & 436.4 & 139.5 & 48.6 & 24.5 & 13.4 & 83.6 & 130.8 & 178.8 & 20.6 & 33.9 & 4.3 & 1 & 9 & 1.3 & 0 & 1.6 & 2.6 & 1.5 & 0 & 0 & 11.2 & 2.8 & 3.9 & 24 & 6.5 & 3.8 \\
110 & 23.9 & 22.8 & 38.9 & 4.8 & 50.7 & 494 & 43.2 & 0.3 & 49.4 & 108.2 & 130.3 & 60.4 & 0 & 426.2 & 163.5 & 53.7 & 26.3 & 15.5 & 93 & 154.3 & 197.6 & 22.1 & 37.3 & 4.7 & 1.1 & 10.6 & 1.1 & 0 & 1.3 & 2.6 & 2.7 & 0.7 & 0 & 10.3 & 3.3 & 4.3 & 25.3 & 6.7 & 4.1 \\
120 & 24.9 & 23.9 & 41.2 & 5.1 & 44.6 & 452.6 & 32.9 & 0.1 & 34.9 & 85 & 115.2 & 91.5 & 1.7 & 333.8 & 150.3 & 47.2 & 22.5 & 14.1 & 81.7 & 142.6 & 173.2 & 23.5 & 40.1 & 5 & 11.7 & 116.8 & 9.2 & 0 & 10 & 23.6 & 30.2 & 20 & 0 & 92.3 & 36.7 & 45.4 & 26.4 & 6.2 & 4.3 \\
130 & 26 & 25 & 43.6 & 5.4 & 48.5 & 510.2 & 32 & 0.1 & 32.5 & 82.1 & 124.3 & 114.9 & 24 & 324.5 & 170 & 51.7 & 23.9 & 15.8 & 88.6 & 162 & 188 & 24.5 & 41.5 & 5.4 & 1.3 & 13.2 & 0.9 & 0 & 1 & 2.4 & 3.3 & 2.8 & 0.1 & 9.3 & 4.2 & 4.9 & 27.5 & 5.9 & 4.4 \\
140 & 27 & 25.9 & 46 & 5.6 & 51 & 553 & 30.5 & 0 & 30.2 & 76.6 & 122.5 & 127.7 & 59.1 & 307.7 & 184.8 & 54.8 & 24.7 & 17 & 92.6 & 176.8 & 197 & 25.6 & 44 & 5.6 & 2.5 & 27.5 & 1.7 & 0 & 1.7 & 4.3 & 6.7 & 6.6 & 1.2 & 16.8 & 8.7 & 9.8 & 28.5 & 5.9 & 4.5 \\[1em]
150 & 27.8 & 26.8 & 48.2 & 5.8 & 53 & 588.9 & 28.7 & 0 & 28.1 & 71 & 115.9 & 136.2 & 97.9 & 290 & 197.4 & 57.3 & 25.2 & 18 & 95.3 & 189.5 & 203.7 & 26.7 & 46.2 & 5.9 & 2.9 & 32.9 & 1.8 & 0 & 1.8 & 4.5 & 7 & 7.3 & 4.1 & 17.7 & 10.5 & 11.3 & 29.3 & 5.8 & 4.6 \\
160 & 28.6 & 27.6 & 50.4 & 6 & 53.6 & 609.7 & 26.6 & 0 & 25.9 & 65.1 & 107 & 135.2 & 137 & 268.6 & 204.9 & 58.5 & 25.1 & 18.5 & 95.5 & 197.4 & 205.1 & 27.4 & 49.3 & 6 & 4.1 & 46.2 & 2.1 & 0 & 2.1 & 5.3 & 8.5 & 10.4 & 9.2 & 21.4 & 14.9 & 15.6 & 30.2 & 5.7 & 4.7 \\
170 & 29.3 & 28.3 & 52.6 & 6.2 & 54.4 & 632.1 & 24.8 & 0 & 24.2 & 60.7 & 99.6 & 130.8 & 176.8 & 250.9 & 213 & 59.9 & 25.1 & 19.1 & 96 & 205.8 & 207.3 & 28.2 & 51.4 & 6.2 & 3.7 & 42.5 & 1.8 & 0 & 1.7 & 4.3 & 7 & 9.2 & 10.7 & 17.7 & 13.8 & 13.9 & 30.9 & 5.5 & 4.7 \\
180 & 30 & 29 & 54.7 & 6.3 & 55.4 & 655.2 & 23.3 & 0 & 22.7 & 57 & 93.5 & 125 & 216.1 & 235.9 & 221.4 & 61.6 & 25.2 & 19.7 & 96.5 & 214.4 & 210 & 28.9 & 53.5 & 6.4 & 3.3 & 39.5 & 1.5 & 0 & 1.4 & 3.6 & 5.9 & 8 & 12 & 14.9 & 12.9 & 12.7 & 31.7 & 5.3 & 4.7 \\
190 & 30.6 & 29.7 & 56.9 & 6.5 & 55.8 & 671 & 21.7 & 0 & 21.1 & 52.9 & 86.8 & 116.9 & 253.1 & 219.1 & 227.3 & 62.6 & 25 & 20.1 & 95.8 & 220.7 & 210.1 & 28.9 & 53.9 & 6.4 & 3.8 & 44.9 & 1.7 & 0 & 1.6 & 4.1 & 6.6 & 8.9 & 13.9 & 16.8 & 14.7 & 14.4 & 32.3 & 5.2 & 4.8 \\[1em]
200 & 31.3 & 30.4 & 59.1 & 6.7 & 57 & 695.7 & 20.6 & 0 & 20 & 50.2 & 82.3 & 111.3 & 290.2 & 208 & 236.3 & 64.6 & 25.3 & 20.8 & 96.5 & 229.9 & 213.5 & 30 & 56.9 & 6.6 & 2.8 & 34.1 & 1.1 & 0 & 1.1 & 2.7 & 4.4 & 5.9 & 12.9 & 11.2 & 11.2 & 10.6 & 33 & 5 & 4.8 \\
210 & 31.8 & 31 & 61.1 & 6.8 & 57.5 & 711.2 & 19.4 & 0 & 18.9 & 47.3 & 77.5 & 105 & 321 & 195.8 & 242.1 & 65.8 & 25.2 & 21.2 & 96 & 236.1 & 214.2 & 30.7 & 59.7 & 6.8 & 3.4 & 41.7 & 1.2 & 0 & 1.2 & 2.9 & 4.8 & 6.5 & 17.9 & 12.1 & 13.8 & 12.6 & 33.6 & 4.9 & 4.8 \\
220 & 32.3 & 31.5 & 63.2 & 7 & 57.9 & 725 & 18.3 & 0 & 17.8 & 44.6 & 73 & 99 & 349.2 & 184.6 & 247.4 & 67 & 25.2 & 21.5 & 95.3 & 241.7 & 214.6 & 31.5 & 61.8 & 7 & 3.3 & 42 & 1.1 & 0 & 1.1 & 2.7 & 4.4 & 6 & 19.5 & 11.2 & 14 & 12.4 & 34.1 & 4.8 & 4.8 \\
230 & 32.8 & 32.1 & 65.3 & 7.1 & 58.3 & 738.4 & 17.3 & 0 & 16.8 & 42.1 & 69 & 93.5 & 375.9 & 174.4 & 252.5 & 68.2 & 25.1 & 21.9 & 94.5 & 247.2 & 215 & 31.6 & 63.3 & 7 & 3.2 & 40.4 & 1 & 0 & 1 & 2.5 & 4.1 & 5.5 & 19.5 & 10.3 & 13.5 & 11.9 & 34.7 & 4.6 & 4.8 \\
240 & 33.3 & 32.6 & 67.3 & 7.2 & 59.7 & 764.1 & 16.6 & 0 & 16.2 & 40.5 & 66.4 & 90 & 407.6 & 167.9 & 261.9 & 70.5 & 25.5 & 22.6 & 95.2 & 256.8 & 218.9 & 32 & 64.5 & 7.1 & 2.1 & 26.9 & 0.6 & 0 & 0.6 & 1.6 & 2.6 & 3.5 & 13.4 & 6.5 & 9 & 7.8 & 35.2 & 4.5 & 4.8 \\[1em]
250 & 33.8 & 33 & 69.3 & 7.4 & 60.5 & 781 & 15.9 & 0 & 15.4 & 38.7 & 63.5 & 86 & 433.4 & 160.4 & 268.3 & 72.2 & 25.6 & 23 & 94.9 & 263.5 & 220.5 & 33.1 & 67.5 & 7.4 & 2.7 & 34.6 & 0.7 & 0 & 0.7 & 1.8 & 3 & 4 & 18.7 & 7.5 & 11.6 & 9.7 & 35.7 & 4.4 & 4.8 \\
260 & 34.2 & 33.5 & 71.3 & 7.5 & 61.4 & 798.4 & 15.2 & 0 & 14.8 & 37.1 & 60.9 & 82.5 & 458.2 & 153.9 & 274.8 & 73.9 & 25.7 & 23.5 & 94.8 & 270.3 & 222.4 & 33.9 & 69.8 & 7.7 & 2.5 & 33 & 0.6 & 0 & 0.6 & 1.6 & 2.6 & 3.5 & 18.7 & 6.5 & 11.1 & 9 & 36.2 & 4.3 & 4.7 \\
270 & 34.5 & 33.9 & 73.3 & 7.6 & 60.6 & 795.1 & 14.2 & 0 & 13.8 & 34.7 & 56.8 & 77 & 470.7 & 143.6 & 274.3 & 73.8 & 25.2 & 23.3 & 92.1 & 270.2 & 218.5 & 33.7 & 70.9 & 7.6 & 4 & 52.4 & 1 & 0 & 1 & 2.5 & 4.1 & 5.5 & 29.9 & 10.3 & 17.7 & 14.4 & 36.6 & 4.2 & 4.7 \\
280 & 34.9 & 34.4 & 75.3 & 7.8 & 59.8 & 791.3 & 13.3 & 0 & 12.9 & 32.4 & 53.1 & 72 & 481.5 & 134.3 & 273.5 & 73.4 & 24.7 & 23.2 & 89.3 & 269.8 & 214.3 & 33.5 & 73.5 & 7.4 & 3.9 & 50.7 & 0.9 & 0 & 0.9 & 2.3 & 3.7 & 5 & 29.7 & 9.3 & 17.2 & 14.2 & 37.1 & 4.1 & 4.7 \\
290 & 35.3 & 34.8 & 77.2 & 7.9 & 61.5 & 819 & 13 & 0 & 12.7 & 31.7 & 52 & 70.5 & 509.8 & 131.5 & 283.6 & 76.2 & 25.2 & 23.9 & 90.4 & 280.1 & 219.3 & 35.4 & 77 & 8.1 & 1.3 & 17.7 & 0.3 & 0 & 0.3 & 0.7 & 1.1 & 1.5 & 11.1 & 2.8 & 6.1 & 4.6 & 37.5 & 3.9 & 4.7 \\[1em]
300 & 35.6 & 35.2 & 79 & 8 & 62.2 & 833.7 & 12.6 & 0 & 12.2 & 30.6 & 50.2 & 68 & 529.6 & 126.8 & 289.2 & 77.7 & 25.3 & 24.3 & 90 & 286 & 220.6 & 35 & 78.5 & 7.9 & 2.3 & 30.1 & 0.5 & 0 & 0.4 & 1.1 & 1.8 & 2.5 & 19 & 4.7 & 10.3 & 8 & 37.9 & 3.9 & 4.6 \\
   \hline
\end{tabular}
\end{adjustbox}
\end{table}
\begin{table}[ht]
  \begin{adjustbox}{angle=90,height=\textheight}
\centering
\begin{tabular}{rrrrrrrrrrrrrrrrrrrrrr|rrrrrrrrrrrrrrr|rrr}
  \hline
age & ho & hg & dg & hKa & g & v & s.NH & 1b & 2a & 2b & 3a & 3b & 4+ & n & BmS & BmA & BmN & BmR & BmW & BmHarv & BmResid & hgAus & dgAus & hKaAus & gAus & vAus & s.NHAus & 1bAus & 2aAus & 2bAus & 3aAus & 3bAus & 4+Aus & nAus & BmHarvAus & BmResidAus & hoRef & lfz & dgz \\
\hline
0 & 0 & 0 & 0 & 0 & 0 & 0 & 0 & 0 & 0 & 0 & 0 & 0 & 0 & 1720 & 0 & 0 & 0 & 0 & 0 & 0 & 0 & 0 & 0 & 0 & 0 & 0 & 0 & 0 & 0 & 0 & 0 & 0 & 0 & 0 & 0 & 0 & 0 & 0 & 0 \\
10 & 1.1 & 0 & 0 & 0 & 0 & 0 & 0 & 0 & 0 & 0 & 0 & 0 & 0 & 1417 & 0 & 0 & 0 & 0 & 0 & 0 & 0 & 0 & 0 & 0 & 0 & 0 & 0 & 0 & 0 & 0 & 0 & 0 & 0 & 303 & 0 & 0 & 1.1 & 0 & 0 \\
20 & 4.4 & 4 & 9 & 0.3 & 5.1 & 11.4 & 0 & 0 & 0 & 0 & 0 & 0 & 0 & 805.4 & 3.8 & 10.8 & 7.9 & 0.5 & 4.9 & 0 & 27.8 & 3.9 & 8.8 & 0.2 & 3.5 & 7.8 & 0 & 0 & 0 & 0 & 0 & 0 & 0 & 581.4 & 0 & 19.3 & 4.6 & 0 & 0 \\
30 & 8.3 & 7.6 & 14.5 & 0.9 & 13.3 & 47.9 & 30.2 & 0 & 0 & 0 & 0 & 0 & 0 & 804.5 & 15.4 & 19.8 & 13.3 & 1.8 & 16.7 & 12 & 55 & 6.1 & 12.3 & 0.6 & 0 & 0 & 0 & 0 & 0 & 0 & 0 & 0 & 0 & 0.9 & 0 & 0 & 9.1 & 3 & 1 \\
40 & 12.1 & 11.1 & 18.9 & 1.8 & 22.4 & 114.7 & 78.4 & 1.4 & 0 & 0 & 0 & 0 & 0 & 798.2 & 37 & 27.4 & 17.7 & 4.1 & 32.8 & 31.4 & 87.6 & 10.5 & 18.1 & 1.7 & 0.2 & 0.8 & 0.5 & 0 & 0 & 0 & 0 & 0 & 0 & 6.3 & 0.2 & 0.6 & 13.2 & 5 & 2 \\[1em]
50 & 15 & 14.2 & 23.1 & 2.7 & 17.8 & 114.5 & 60.8 & 3.9 & 19.1 & 0 & 0 & 0 & 0 & 424.6 & 37.1 & 19.9 & 12.1 & 3.9 & 28.8 & 32.7 & 69 & 12.5 & 20.4 & 2.3 & 12.2 & 70.3 & 48.4 & 1.3 & 0.3 & 0 & 0 & 0 & 0 & 373.6 & 19.6 & 47.2 & 16.5 & 5.4 & 2.7 \\
60 & 17.6 & 16.8 & 27 & 3.4 & 24.2 & 181.3 & 59.4 & 6.1 & 63.9 & 7.2 & 0 & 0 & 0 & 422.8 & 58.9 & 25.7 & 14.9 & 6 & 41.5 & 53.3 & 93.8 & 15.1 & 25.9 & 2.7 & 0.1 & 0.6 & 0.2 & 0 & 0.2 & 0 & 0 & 0 & 0 & 1.8 & 0.2 & 0.4 & 19.3 & 5.3 & 3.1 \\
70 & 19.9 & 19 & 30.4 & 4 & 30.7 & 257.7 & 56.4 & 3.9 & 65.1 & 71.3 & 1.4 & 0 & 0 & 421 & 84.2 & 31.7 & 17.6 & 8.4 & 54.5 & 77.3 & 119.1 & 19.2 & 30.4 & 4.1 & 0.1 & 1.1 & 0.3 & 0 & 0.3 & 0.3 & 0 & 0 & 0 & 1.8 & 0.3 & 0.5 & 21.7 & 6.2 & 3.6 \\
80 & 21.9 & 20.9 & 33.6 & 4.5 & 37 & 340 & 51.9 & 1.6 & 63.5 & 100.5 & 47.6 & 0 & 0 & 417.5 & 111.6 & 37.6 & 20.1 & 10.9 & 67.1 & 103.7 & 143.6 & 20.6 & 33 & 4.4 & 0.3 & 2.8 & 0.4 & 0 & 0.5 & 0.8 & 0.4 & 0 & 0 & 3.6 & 0.8 & 1.2 & 23.7 & 6.9 & 4 \\
90 & 23.7 & 22.7 & 36.5 & 5 & 43 & 423.7 & 47 & 0.7 & 57.8 & 108 & 111.5 & 9.1 & 0 & 411.2 & 139.6 & 43.2 & 22.3 & 13.4 & 78.6 & 130.9 & 166.3 & 22.2 & 36 & 4.8 & 0.6 & 6.1 & 0.8 & 0 & 0.9 & 1.7 & 1.4 & 0 & 0 & 6.3 & 1.9 & 2.4 & 25.5 & 7.4 & 4.4 \\[1em]
100 & 25.3 & 24.2 & 39.2 & 5.4 & 48.7 & 508.9 & 43.3 & 0.3 & 48.9 & 109.1 & 133.4 & 69.8 & 0 & 404 & 168.4 & 48.6 & 24.3 & 15.9 & 89.4 & 158.9 & 187.7 & 23.2 & 36.6 & 5.3 & 0.8 & 7.6 & 0.8 & 0 & 1 & 1.8 & 2.1 & 0.2 & 0 & 7.2 & 2.4 & 2.9 & 27 & 7.7 & 4.7 \\
110 & 26.7 & 25.6 & 41.7 & 5.8 & 52.9 & 580.5 & 40.7 & 0.1 & 42.4 & 105 & 145.1 & 123.2 & 8.6 & 387.9 & 192.8 & 52.6 & 25.6 & 18 & 97 & 183 & 203.1 & 24.7 & 40.4 & 5.6 & 2.1 & 21.9 & 1.7 & 0 & 1.9 & 4.4 & 5.6 & 3.7 & 0.2 & 16.1 & 6.9 & 7.9 & 28.4 & 7.8 & 5 \\
120 & 27.9 & 26.8 & 44.1 & 6.2 & 46.7 & 533.5 & 31.3 & 0.1 & 31.4 & 79.8 & 122.6 & 117 & 48 & 305.5 & 177.8 & 46.5 & 22 & 16.5 & 85.2 & 169.5 & 178.4 & 26.2 & 42.9 & 6 & 11.9 & 133.3 & 8.5 & 0 & 8.7 & 21.9 & 32.3 & 29.7 & 6.1 & 82.4 & 42.2 & 45.7 & 29.6 & 7.2 & 5.2 \\
130 & 29 & 27.9 & 46.6 & 6.5 & 49.9 & 588.7 & 29.4 & 0.1 & 28.8 & 73.3 & 118.5 & 129 & 98.2 & 292 & 196.8 & 50 & 23 & 18 & 90.3 & 188.5 & 189.8 & 27.7 & 45.4 & 6.5 & 2.2 & 25.5 & 1.4 & 0 & 1.4 & 3.5 & 5.6 & 5.6 & 3.3 & 13.4 & 8.2 & 8.3 & 30.7 & 6.8 & 5.3 \\
140 & 30 & 28.9 & 49.1 & 6.7 & 52.2 & 632.4 & 27.4 & 0 & 26.7 & 67.3 & 110.6 & 134 & 149.3 & 275 & 212.1 & 52.7 & 23.6 & 19.3 & 93.5 & 203.9 & 197.4 & 28.6 & 47 & 6.8 & 2.9 & 35.7 & 1.8 & 0 & 1.7 & 4.3 & 6.9 & 7.6 & 6.8 & 17 & 11.5 & 11.1 & 31.8 & 6.7 & 5.4 \\[1em]
150 & 30.9 & 29.9 & 51.5 & 7 & 54.6 & 678.2 & 26 & 0 & 25.3 & 63.5 & 104.3 & 135 & 200.9 & 261.6 & 228.2 & 55.7 & 24.3 & 20.5 & 96.7 & 220 & 205.4 & 29.6 & 49.7 & 7 & 2.6 & 32.3 & 1.3 & 0 & 1.3 & 3.3 & 5.4 & 6.5 & 8.6 & 13.4 & 10.5 & 9.8 & 32.7 & 6.6 & 5.5 \\
160 & 31.7 & 30.7 & 53.9 & 7.2 & 56.5 & 717.9 & 24.6 & 0 & 23.9 & 59.9 & 98.4 & 131.5 & 251.5 & 248.1 & 242.2 & 58.2 & 24.8 & 21.7 & 99 & 234.3 & 211.5 & 30.5 & 52.9 & 7.2 & 2.9 & 36.9 & 1.4 & 0 & 1.3 & 3.3 & 5.4 & 7 & 11.8 & 13.4 & 12 & 10.9 & 33.6 & 6.5 & 5.5 \\
170 & 32.5 & 31.5 & 56.1 & 7.4 & 57.2 & 740.6 & 22.9 & 0 & 22.3 & 55.8 & 91.5 & 123.6 & 294.5 & 231.1 & 250.6 & 59.5 & 24.7 & 22.2 & 98.8 & 243 & 212.9 & 31.4 & 54.6 & 7.4 & 4 & 51.6 & 1.7 & 0 & 1.6 & 4.1 & 6.7 & 9.1 & 19.2 & 17 & 16.9 & 14.8 & 34.5 & 6.3 & 5.6 \\
180 & 33.3 & 32.3 & 58.4 & 7.6 & 59.9 & 788.7 & 22.2 & 0 & 21.6 & 54 & 88.6 & 120.1 & 345.8 & 224 & 267.5 & 63 & 25.6 & 23.6 & 102 & 260.1 & 221.6 & 32.5 & 56.8 & 7.8 & 1.8 & 24.2 & 0.7 & 0 & 0.7 & 1.7 & 2.8 & 3.8 & 10.2 & 7.2 & 8 & 6.7 & 35.2 & 6.2 & 5.6 \\
190 & 34 & 33 & 60.6 & 7.8 & 60.7 & 811 & 20.8 & 0 & 20.3 & 50.8 & 83.3 & 112.9 & 384.6 & 210.5 & 275.8 & 64.4 & 25.6 & 24.2 & 101.7 & 268.8 & 223 & 32.9 & 58.8 & 7.8 & 3.6 & 48.8 & 1.3 & 0 & 1.3 & 3.2 & 5.3 & 7.2 & 22 & 13.4 & 16.1 & 13.4 & 36 & 6.1 & 5.6 \\[1em]
200 & 34.6 & 33.7 & 62.8 & 7.9 & 61.3 & 830.7 & 19.6 & 0 & 19.1 & 47.8 & 78.3 & 106.2 & 420 & 198 & 283.3 & 65.7 & 25.6 & 24.7 & 101.2 & 276.6 & 223.8 & 33.4 & 60.8 & 8 & 3.6 & 49.2 & 1.2 & 0 & 1.2 & 3 & 5 & 6.7 & 23.7 & 12.5 & 16.3 & 13.3 & 36.6 & 5.9 & 5.7 \\
210 & 35.1 & 34.2 & 64.8 & 8.1 & 62.4 & 855.8 & 18.7 & 0 & 18.2 & 45.6 & 74.8 & 101.4 & 454.8 & 189 & 292.5 & 67.6 & 25.8 & 25.4 & 101.5 & 286.1 & 226.6 & 35 & 65.3 & 8.4 & 3 & 41.9 & 0.9 & 0 & 0.9 & 2.2 & 3.5 & 4.8 & 22.7 & 9 & 14 & 10.8 & 37.3 & 5.7 & 5.7 \\
220 & 35.7 & 34.9 & 67 & 8.3 & 62.5 & 866.6 & 17.6 & 0 & 17.1 & 42.8 & 70.2 & 95.1 & 481.5 & 177.4 & 296.9 & 68.3 & 25.5 & 25.6 & 99.9 & 290.9 & 225.3 & 34.6 & 66.1 & 8.2 & 4 & 54.9 & 1.2 & 0 & 1.1 & 2.8 & 4.6 & 6.2 & 29.9 & 11.6 & 18.4 & 14.4 & 37.9 & 5.7 & 5.7 \\
230 & 36.2 & 35.5 & 69.3 & 8.4 & 61.9 & 867.6 & 16.2 & 0 & 15.8 & 39.6 & 64.9 & 87.9 & 502.5 & 163.9 & 297.9 & 68.4 & 25.1 & 25.6 & 97.1 & 292.5 & 221.6 & 34.9 & 64.8 & 8.4 & 4.4 & 61.9 & 1.3 & 0 & 1.3 & 3.2 & 5.3 & 7.2 & 33.3 & 13.4 & 20.7 & 15.9 & 38.5 & 5.4 & 5.6 \\
240 & 36.7 & 36 & 71.5 & 8.6 & 63.6 & 900.8 & 15.7 & 0 & 15.3 & 38.3 & 62.7 & 85.1 & 539.3 & 158.6 & 310.1 & 71 & 25.5 & 26.5 & 98.1 & 304.9 & 226.3 & 35.2 & 69.2 & 8.3 & 2 & 27.9 & 0.5 & 0 & 0.5 & 1.3 & 2.1 & 2.9 & 16 & 5.4 & 9.4 & 7.2 & 39 & 5.3 & 5.6 \\[1em]
250 & 37.2 & 36.5 & 73.5 & 8.7 & 63.8 & 912.1 & 14.9 & 0 & 14.5 & 36.3 & 59.5 & 80.7 & 561.3 & 150.5 & 314.6 & 71.9 & 25.4 & 26.8 & 96.8 & 309.8 & 225.7 & 36.2 & 73.5 & 8.6 & 3.4 & 48.2 & 0.8 & 0 & 0.8 & 1.9 & 3.2 & 4.3 & 29.5 & 8.1 & 16.4 & 12.1 & 39.6 & 5.1 & 5.6 \\
260 & 37.6 & 37 & 75.6 & 8.9 & 63.5 & 915.5 & 14 & 0 & 13.6 & 34.2 & 56 & 75.9 & 577.9 & 141.5 & 316.5 & 72.3 & 25.1 & 26.8 & 94.7 & 312.1 & 223.2 & 36.6 & 73.4 & 8.8 & 3.8 & 54 & 0.9 & 0 & 0.9 & 2.2 & 3.5 & 4.8 & 33.2 & 9 & 18.4 & 13.3 & 40.1 & 5 & 5.6 \\
270 & 38 & 37.5 & 77.6 & 9 & 63.6 & 923 & 13.3 & 0 & 12.9 & 32.4 & 53.2 & 72.1 & 595.4 & 134.4 & 319.7 & 73.2 & 24.9 & 27 & 93.2 & 315.7 & 222.3 & 38 & 76.5 & 9.3 & 3.3 & 48.7 & 0.7 & 0 & 0.7 & 1.7 & 2.8 & 3.8 & 31.3 & 7.2 & 16.7 & 11.5 & 40.6 & 4.9 & 5.6 \\
280 & 38.5 & 37.9 & 79.7 & 9.2 & 65.3 & 954.4 & 12.9 & 0 & 12.6 & 31.6 & 51.7 & 70.2 & 628.2 & 130.8 & 331.3 & 75.8 & 25.4 & 27.8 & 94 & 327.5 & 226.8 & 37.4 & 76.3 & 9 & 1.6 & 23.8 & 0.4 & 0 & 0.3 & 0.9 & 1.4 & 1.9 & 15.2 & 3.6 & 8.1 & 5.7 & 41 & 4.8 & 5.5 \\
290 & 38.9 & 38.4 & 81.8 & 9.3 & 66.8 & 982.5 & 12.6 & 0 & 12.2 & 30.7 & 50.3 & 68.2 & 658.2 & 127.2 & 341.7 & 78.3 & 25.8 & 28.6 & 94.4 & 338.3 & 230.6 & 38.2 & 78.7 & 9.3 & 1.7 & 25.7 & 0.4 & 0 & 0.3 & 0.9 & 1.4 & 1.9 & 16.8 & 3.6 & 8.8 & 6 & 41.5 & 4.7 & 5.5 \\[1em]
300 & 39.2 & 38.8 & 83.7 & 9.4 & 67.6 & 999.9 & 12.1 & 0 & 11.8 & 29.6 & 48.6 & 65.8 & 680.4 & 122.7 & 348.4 & 80.1 & 26 & 29.1 & 94 & 345.3 & 232.2 & 39.4 & 81.9 & 9.7 & 2.3 & 35.5 & 0.4 & 0 & 0.4 & 1.1 & 1.8 & 2.4 & 24 & 4.5 & 12.3 & 8 & 41.9 & 4.6 & 5.5 \\
   \hline
\end{tabular}
\end{adjustbox}
\end{table}
\begin{table}[ht]
  \begin{adjustbox}{angle=90,height=\textheight}
\centering
\begin{tabular}{rrrrrrrrrrrrrrrrrrrrrr|rrrrrrrrrrrrrrr|rrr}
  \hline
age & ho & hg & dg & hKa & g & v & s.NH & 1b & 2a & 2b & 3a & 3b & 4+ & n & BmS & BmA & BmN & BmR & BmW & BmHarv & BmResid & hgAus & dgAus & hKaAus & gAus & vAus & s.NHAus & 1bAus & 2aAus & 2bAus & 3aAus & 3bAus & 4+Aus & nAus & BmHarvAus & BmResidAus & hoRef & lfz & dgz \\
\hline
0 & 0 & 0 & 0 & 0 & 0 & 0 & 0 & 0 & 0 & 0 & 0 & 0 & 0 & 1600 & 0 & 0 & 0 & 0 & 0 & 0 & 0 & 0 & 0 & 0 & 0 & 0 & 0 & 0 & 0 & 0 & 0 & 0 & 0 & 0 & 0 & 0 & 0 & 0 & 0 \\
10 & 1.4 & 1.3 & 2.6 & 0 & 0.7 & 0.5 & 0 & 0 & 0 & 0 & 0 & 0 & 0 & 1289.5 & 0.2 & 2.9 & 3 & 0 & 0.7 & 0 & 6.7 & 0 & 0 & 0 & 0 & 0 & 0 & 0 & 0 & 0 & 0 & 0 & 0 & 279.5 & 0 & 0 & 1.4 & 0 & 0 \\
20 & 5.2 & 4.8 & 10.1 & 0.4 & 6.2 & 16 & 5.4 & 0 & 0 & 0 & 0 & 0 & 0 & 771.1 & 5.2 & 11.8 & 8.5 & 0.7 & 6.3 & 2.2 & 30.4 & 4.7 & 10.1 & 0.4 & 4.1 & 10.4 & 3.5 & 0 & 0 & 0 & 0 & 0 & 0 & 518.4 & 1.4 & 20.1 & 5.6 & 0.9 & 0.4 \\
30 & 9.7 & 8.9 & 15.9 & 1.3 & 15.2 & 63.9 & 42 & 0 & 0 & 0 & 0 & 0 & 0 & 766.7 & 20.6 & 20.4 & 13.7 & 2.4 & 20.2 & 16.6 & 60.7 & 7.1 & 13.6 & 0.9 & 0.1 & 0.2 & 0.1 & 0 & 0 & 0 & 0 & 0 & 0 & 4.4 & 0 & 0.3 & 10.7 & 3.7 & 1.5 \\
40 & 13.8 & 12.9 & 20.5 & 2.4 & 25 & 148.9 & 102.6 & 3.2 & 0.4 & 0 & 0 & 0 & 0 & 761.4 & 48.1 & 27.9 & 18 & 5.2 & 38.2 & 41.5 & 95.9 & 10.9 & 18.4 & 1.8 & 0.1 & 0.7 & 0.5 & 0 & 0 & 0 & 0 & 0 & 0 & 5.3 & 0.2 & 0.5 & 15.2 & 6.5 & 2.8 \\[1em]
50 & 17 & 16.2 & 24.8 & 3.4 & 19 & 140 & 47.2 & 13.2 & 43.8 & 0 & 0 & 0 & 0 & 393 & 45.4 & 19.5 & 11.9 & 4.7 & 31.6 & 40.6 & 72.5 & 14.4 & 22.1 & 2.9 & 14.1 & 94.1 & 54 & 6.3 & 8.2 & 0 & 0 & 0 & 0 & 368.4 & 26.8 & 53.6 & 18.9 & 6.6 & 3.5 \\
60 & 19.8 & 18.9 & 28.9 & 4.2 & 25.5 & 217.4 & 51.4 & 10.2 & 65.6 & 38.9 & 0 & 0 & 0 & 389.5 & 70.8 & 25.2 & 14.6 & 7.1 & 44.7 & 64.7 & 97.7 & 18.3 & 28 & 4 & 0.2 & 1.8 & 0.4 & 0.1 & 0.6 & 0.2 & 0 & 0 & 0 & 3.5 & 0.5 & 0.8 & 21.9 & 6.3 & 4 \\
70 & 22.2 & 21.4 & 32.5 & 4.8 & 32.3 & 306.3 & 52.2 & 3.1 & 65.7 & 96.1 & 21.3 & 0 & 0 & 387.7 & 100.3 & 31.1 & 17.2 & 9.9 & 58.2 & 92.9 & 123.7 & 19.7 & 29 & 4.6 & 0.1 & 1 & 0.2 & 0.1 & 0.3 & 0.2 & 0 & 0 & 0 & 1.8 & 0.3 & 0.4 & 24.4 & 7.3 & 4.5 \\
80 & 24.4 & 23.4 & 35.9 & 5.4 & 38.9 & 400.7 & 47.7 & 1.2 & 59.2 & 104.8 & 101.3 & 1.8 & 0 & 384.2 & 131.8 & 37.1 & 19.7 & 12.7 & 71 & 123.4 & 148.9 & 21.8 & 32.6 & 5.1 & 0.3 & 2.8 & 0.5 & 0 & 0.6 & 0.8 & 0.2 & 0 & 0 & 3.5 & 0.9 & 1.1 & 26.5 & 8 & 4.9 \\
90 & 26.2 & 25.3 & 38.9 & 6 & 45 & 496.4 & 44.4 & 0.5 & 49.6 & 108.8 & 128.1 & 63.9 & 0 & 378.1 & 164.1 & 42.6 & 21.8 & 15.6 & 82.6 & 154.8 & 171.9 & 24.1 & 37.9 & 5.5 & 0.7 & 7.2 & 0.7 & 0 & 0.8 & 1.7 & 2 & 0.5 & 0 & 6.1 & 2.2 & 2.7 & 28.4 & 8.5 & 5.3 \\[1em]
100 & 27.9 & 26.9 & 41.7 & 6.5 & 51.3 & 597.2 & 42.8 & 0.3 & 43.1 & 106.9 & 147.6 & 129.2 & 9.5 & 374.6 & 198.2 & 48.3 & 24 & 18.5 & 94 & 188.2 & 194.8 & 26.4 & 41.3 & 6.3 & 0.5 & 5.4 & 0.4 & 0 & 0.4 & 1 & 1.4 & 1 & 0 & 3.5 & 1.7 & 1.8 & 30 & 8.8 & 5.7 \\
110 & 29.4 & 28.4 & 44.4 & 6.9 & 56.2 & 685.3 & 39.4 & 0.2 & 37.9 & 97.2 & 152.1 & 144 & 82.9 & 363.2 & 228.3 & 52.8 & 25.5 & 21.1 & 102.5 & 217.9 & 212.3 & 27.9 & 42.9 & 6.9 & 1.6 & 19.8 & 1.3 & 0 & 1.3 & 3.2 & 4.7 & 4.4 & 1 & 11.4 & 6.3 & 6.2 & 31.5 & 9 & 6 \\
120 & 30.6 & 29.7 & 47 & 7.3 & 48.8 & 617.9 & 29.6 & 0.1 & 27.9 & 71.7 & 117.2 & 127.1 & 128.6 & 281.6 & 206.6 & 45.9 & 21.6 & 18.9 & 88.2 & 198 & 183.2 & 29.1 & 45.7 & 7.1 & 13.4 & 166.5 & 8.8 & 0 & 8.3 & 21.3 & 34.5 & 34.3 & 27.8 & 81.6 & 53.2 & 50.4 & 32.8 & 8.3 & 6.2 \\
130 & 31.8 & 30.9 & 49.6 & 7.6 & 51.6 & 673.5 & 27.5 & 0.1 & 25.9 & 66.1 & 109.2 & 135.1 & 186.5 & 266.7 & 226 & 49 & 22.4 & 20.5 & 92.3 & 217.5 & 192.6 & 30.6 & 48.6 & 7.6 & 2.8 & 35.8 & 1.5 & 0 & 1.5 & 3.7 & 6.1 & 7.2 & 9.1 & 14.9 & 11.5 & 10.3 & 34 & 7.7 & 6.3 \\
140 & 32.9 & 32 & 52.3 & 7.9 & 53.9 & 722.8 & 25.3 & 0.1 & 24.2 & 61.1 & 100.8 & 132.7 & 249.2 & 250.9 & 243.3 & 51.7 & 23 & 21.9 & 95.2 & 235 & 200 & 31.5 & 49.6 & 8 & 3 & 40.5 & 1.7 & 0 & 1.6 & 4 & 6.7 & 8 & 11.1 & 15.8 & 13.1 & 11.2 & 35.1 & 7.6 & 6.4 \\[1em]
150 & 33.9 & 33 & 54.8 & 8.2 & 57.8 & 793 & 24.4 & 0 & 23.6 & 59.2 & 97.3 & 130.5 & 318.7 & 244.7 & 267.7 & 56 & 24.2 & 23.9 & 100.6 & 259.4 & 212.9 & 32.8 & 52.9 & 8.3 & 1.3 & 18.4 & 0.6 & 0 & 0.6 & 1.5 & 2.5 & 3.2 & 6.6 & 6.1 & 6 & 4.9 & 36.1 & 7.5 & 6.5 \\
160 & 34.8 & 33.9 & 57.2 & 8.4 & 60.1 & 842.8 & 23.3 & 0 & 22.6 & 56.6 & 93.1 & 125.6 & 376.1 & 234.2 & 285.4 & 58.8 & 24.8 & 25.3 & 103.2 & 277.3 & 220.2 & 34 & 57.2 & 8.5 & 2.7 & 38.1 & 1 & 0 & 1 & 2.5 & 4.2 & 5.6 & 17.1 & 10.5 & 12.5 & 9.9 & 37 & 7.5 & 6.5 \\
170 & 35.6 & 34.7 & 59.6 & 8.7 & 59 & 841.5 & 21 & 0 & 20.4 & 51.1 & 83.9 & 113.7 & 408.4 & 211.4 & 285.8 & 58.3 & 24.1 & 25.1 & 99.6 & 278.3 & 214.5 & 34.6 & 57.9 & 8.8 & 6 & 85.9 & 2.3 & 0 & 2.2 & 5.5 & 9.1 & 12.3 & 39.8 & 22.8 & 28.3 & 21.8 & 37.9 & 7.2 & 6.6 \\
180 & 36.4 & 35.6 & 62.1 & 8.9 & 59.7 & 865.3 & 19.6 & 0 & 19 & 47.7 & 78.2 & 106.1 & 450 & 197.4 & 294.7 & 59.7 & 24 & 25.7 & 99.1 & 287.7 & 215.5 & 35.1 & 59.8 & 8.9 & 3.9 & 56.6 & 1.4 & 0 & 1.4 & 3.4 & 5.6 & 7.6 & 27.6 & 14 & 18.7 & 14.2 & 38.7 & 6.9 & 6.6 \\
190 & 37.1 & 36.3 & 64.5 & 9.1 & 61 & 897 & 18.5 & 0 & 18 & 45.1 & 74 & 100.3 & 493.1 & 186.8 & 306.3 & 61.7 & 24.3 & 26.6 & 99.4 & 299.7 & 218.5 & 36 & 62.3 & 9.1 & 3.2 & 47 & 1 & 0 & 1 & 2.5 & 4.2 & 5.6 & 24.7 & 10.5 & 15.6 & 11.5 & 39.5 & 6.8 & 6.6 \\[1em]
200 & 37.8 & 37 & 66.8 & 9.3 & 61.8 & 919.1 & 17.5 & 0 & 17 & 42.5 & 69.8 & 94.6 & 528.2 & 176.3 & 314.6 & 63.2 & 24.3 & 27.2 & 98.9 & 308.5 & 219.8 & 37.5 & 65.9 & 9.6 & 3.6 & 54.4 & 1 & 0 & 1 & 2.5 & 4.2 & 5.6 & 31.1 & 10.5 & 18.2 & 12.7 & 40.2 & 6.6 & 6.6 \\
210 & 38.4 & 37.7 & 69.1 & 9.5 & 62.8 & 946.7 & 16.6 & 0 & 16.1 & 40.4 & 66.3 & 89.9 & 565.3 & 167.5 & 324.9 & 64.9 & 24.5 & 27.9 & 98.8 & 319.1 & 221.9 & 37.1 & 67.5 & 9.3 & 3.1 & 46.7 & 0.9 & 0 & 0.8 & 2.1 & 3.5 & 4.7 & 27.1 & 8.8 & 15.7 & 11.2 & 40.9 & 6.4 & 6.6 \\
220 & 39.1 & 38.4 & 71.5 & 9.7 & 65.5 & 998.1 & 16.2 & 0 & 15.7 & 39.4 & 64.5 & 87.5 & 616.4 & 163.2 & 343.4 & 68.4 & 25.3 & 29.3 & 101 & 337.9 & 229.6 & 37.1 & 64.9 & 9.5 & 1.4 & 21.8 & 0.4 & 0 & 0.4 & 1.1 & 1.7 & 2.4 & 12.3 & 4.4 & 7.3 & 5.2 & 41.6 & 6.3 & 6.6 \\
230 & 39.6 & 39 & 73.7 & 9.9 & 66.3 & 1022.1 & 15.4 & 0 & 15 & 37.5 & 61.4 & 83.3 & 649.2 & 155.3 & 352.5 & 70 & 25.4 & 30 & 100.4 & 347.4 & 230.8 & 38.1 & 71.8 & 9.5 & 3.2 & 47.7 & 0.8 & 0 & 0.8 & 1.9 & 3.1 & 4.2 & 29.3 & 7.9 & 16.2 & 11.1 & 42.2 & 6.2 & 6.6 \\
240 & 40.2 & 39.6 & 76.1 & 10 & 64.3 & 1000.3 & 14 & 0 & 13.6 & 34.1 & 55.9 & 75.8 & 651.8 & 141.2 & 345.8 & 68.6 & 24.4 & 29.3 & 95.4 & 341.4 & 222.2 & 39.2 & 73 & 10 & 5.9 & 91.2 & 1.4 & 0 & 1.4 & 3.4 & 5.6 & 7.5 & 57.6 & 14 & 31 & 20.4 & 42.8 & 6 & 6.5 \\[1em]
250 & 40.7 & 40.2 & 78.5 & 10.2 & 65.4 & 1026.9 & 13.4 & 0 & 13 & 32.6 & 53.4 & 72.5 & 684.5 & 135.1 & 355.9 & 70.6 & 24.6 & 30 & 95.1 & 351.8 & 224.3 & 38.9 & 73.7 & 9.8 & 2.6 & 40.2 & 0.6 & 0 & 0.6 & 1.5 & 2.4 & 3.3 & 25.5 & 6.1 & 13.7 & 9.1 & 43.4 & 5.8 & 6.5 \\
260 & 41.1 & 40.7 & 80.7 & 10.4 & 66.4 & 1049.9 & 12.9 & 0 & 12.5 & 31.3 & 51.4 & 69.6 & 712.7 & 129.8 & 364.6 & 72.4 & 24.8 & 30.6 & 94.8 & 360.9 & 226.2 & 41.2 & 80.5 & 10.7 & 2.6 & 42.4 & 0.5 & 0 & 0.5 & 1.3 & 2.1 & 2.8 & 28.8 & 5.3 & 14.6 & 9 & 43.9 & 5.7 & 6.5 \\
270 & 41.6 & 41.2 & 83 & 10.5 & 65.4 & 1041.9 & 12 & 0 & 11.7 & 29.2 & 47.9 & 64.9 & 719.4 & 121.1 & 362.6 & 72.1 & 24.2 & 30.3 & 91.6 & 359.4 & 221.4 & 41 & 80.7 & 10.5 & 4.4 & 71.1 & 0.9 & 0 & 0.8 & 2.1 & 3.5 & 4.7 & 48.3 & 8.8 & 24.4 & 15.1 & 44.4 & 5.5 & 6.4 \\
280 & 42 & 41.7 & 85.2 & 10.7 & 67.5 & 1082.6 & 11.7 & 0 & 11.4 & 28.6 & 46.8 & 63.5 & 759.1 & 118.4 & 377.6 & 75.2 & 24.8 & 31.4 & 92.7 & 374.7 & 227 & 41.3 & 78.9 & 10.7 & 1.3 & 20.7 & 0.3 & 0 & 0.3 & 0.6 & 1 & 1.4 & 14 & 2.6 & 7.1 & 4.4 & 44.9 & 5.4 & 6.4 \\
290 & 42.4 & 42.2 & 87.5 & 10.9 & 67.6 & 1089.2 & 11.1 & 0 & 10.8 & 27.1 & 44.4 & 60.2 & 774.6 & 112.3 & 380.7 & 76.1 & 24.7 & 31.6 & 90.9 & 378.3 & 225.6 & 41.9 & 82.7 & 10.9 & 3.3 & 53.4 & 0.6 & 0 & 0.6 & 1.5 & 2.4 & 3.3 & 36.9 & 6.1 & 18.4 & 11.1 & 45.4 & 5.2 & 6.4 \\[1em]
300 & 42.7 & 42.6 & 89.6 & 11 & 66.3 & 1074.6 & 10.4 & 0 & 10.1 & 25.4 & 41.6 & 56.5 & 773.1 & 105.3 & 376.3 & 75.4 & 24.1 & 31.1 & 87.6 & 374.3 & 220.2 & 43 & 90 & 11.2 & 4.4 & 72.7 & 0.7 & 0 & 0.7 & 1.7 & 2.8 & 3.8 & 52.4 & 7 & 25.3 & 14.7 & 45.9 & 5.1 & 6.3 \\
   \hline
\end{tabular}
\end{adjustbox}
\end{table}
\begin{table}[ht]
  \begin{adjustbox}{angle=90,height=\textheight}
\centering
\begin{tabular}{rrrrrrrrrrrrrrrrrrrrrr|rrrrrrrrrrrrrrr|rrr}
  \hline
age & ho & hg & dg & hKa & g & v & s.NH & 1b & 2a & 2b & 3a & 3b & 4+ & n & BmS & BmA & BmN & BmR & BmW & BmHarv & BmResid & hgAus & dgAus & hKaAus & gAus & vAus & s.NHAus & 1bAus & 2aAus & 2bAus & 3aAus & 3bAus & 4+Aus & nAus & BmHarvAus & BmResidAus & hoRef & lfz & dgz \\
\hline
0 & 0 & 0 & 0 & 0 & 0 & 0 & 0 & 0 & 0 & 0 & 0 & 0 & 0 & 1480 & 0 & 0 & 0 & 0 & 0 & 0 & 0 & 0 & 0 & 0 & 0 & 0 & 0 & 0 & 0 & 0 & 0 & 0 & 0 & 0 & 0 & 0 & 0 & 0 & 0 \\
10 & 1.7 & 1.6 & 3.7 & 0 & 1.3 & 1 & 0 & 0 & 0 & 0 & 0 & 0 & 0 & 1186.2 & 0.4 & 4.6 & 4.2 & 0.1 & 1.1 & 0 & 10.3 & 0 & 0 & 0 & 0 & 0 & 0 & 0 & 0 & 0 & 0 & 0 & 0 & 261.1 & 0 & 0 & 1.7 & 0 & 0 \\
20 & 6.2 & 5.6 & 11.3 & 0.6 & 7 & 20.3 & 10.6 & 0 & 0 & 0 & 0 & 0 & 0 & 704 & 6.6 & 12 & 8.6 & 0.8 & 7.6 & 4.3 & 31.3 & 5.6 & 11.3 & 0.5 & 4.8 & 13.9 & 7.3 & 0 & 0 & 0 & 0 & 0 & 0 & 482.2 & 2.9 & 21.6 & 6.7 & 1.8 & 0.9 \\
30 & 11.2 & 10.3 & 17.2 & 1.7 & 16.3 & 78.8 & 53.3 & 0.1 & 0 & 0 & 0 & 0 & 0 & 700.6 & 25.3 & 20 & 13.4 & 2.9 & 22.6 & 21 & 63.4 & 8.3 & 14.4 & 1.2 & 0.1 & 0.2 & 0.1 & 0 & 0 & 0 & 0 & 0 & 0 & 3.4 & 0.1 & 0.2 & 12.5 & 4.3 & 2 \\
40 & 15.7 & 14.6 & 21.9 & 3 & 26.1 & 177.1 & 102 & 14.3 & 12.3 & 0 & 0 & 0 & 0 & 693.7 & 57.2 & 26.9 & 17.4 & 6.1 & 41.2 & 50.1 & 98.8 & 13.1 & 20.2 & 2.6 & 0.2 & 1.3 & 0.9 & 0 & 0 & 0 & 0 & 0 & 0 & 6.9 & 0.4 & 0.8 & 17.3 & 7.6 & 3.4 \\[1em]
50 & 19 & 18.1 & 26.3 & 4.1 & 19.9 & 165.7 & 40.6 & 21.7 & 61 & 1.7 & 0 & 0 & 0 & 365.7 & 53.8 & 19 & 11.6 & 5.5 & 33.9 & 48.6 & 75.3 & 16.2 & 23.6 & 3.6 & 14.3 & 108.1 & 39.2 & 14.7 & 26 & 0 & 0 & 0 & 0 & 328 & 31.1 & 53.9 & 21.2 & 7.6 & 4.3 \\
60 & 22 & 21.1 & 30.5 & 5 & 26.5 & 253 & 50 & 7.1 & 65.5 & 71.5 & 1.3 & 0 & 0 & 362.3 & 82.6 & 24.4 & 14.2 & 8.2 & 47.2 & 76 & 100.6 & 19.6 & 29.4 & 4.4 & 0.2 & 2 & 0.5 & 0.1 & 0.6 & 0.4 & 0 & 0 & 0 & 3.4 & 0.6 & 0.9 & 24.4 & 7.2 & 4.8 \\
70 & 24.6 & 23.6 & 34.3 & 5.8 & 33.1 & 349.7 & 49.1 & 2.2 & 62.7 & 99.5 & 61.2 & 0 & 0 & 358 & 114.7 & 30 & 16.6 & 11.2 & 60.2 & 107 & 125.7 & 22.5 & 31.8 & 5.5 & 0.3 & 3.5 & 0.6 & 0.1 & 0.8 & 1.1 & 0.1 & 0 & 0 & 4.3 & 1 & 1.3 & 27.1 & 8.2 & 5.2 \\
80 & 26.8 & 25.8 & 37.8 & 6.4 & 39.8 & 454.2 & 46.9 & 1 & 54 & 107.5 & 120.2 & 31.7 & 0 & 355.4 & 149.7 & 35.7 & 19 & 14.3 & 73 & 140.9 & 150.7 & 25.7 & 38.6 & 6.3 & 0.3 & 3.3 & 0.3 & 0 & 0.4 & 0.7 & 0.7 & 0.6 & 0 & 2.6 & 1 & 1.1 & 29.3 & 8.9 & 5.7 \\
90 & 28.8 & 27.8 & 40.9 & 7 & 46.4 & 564.3 & 45.4 & 0.6 & 44 & 107.2 & 141.2 & 112.6 & 1.8 & 352.9 & 186.9 & 41.4 & 21.2 & 17.6 & 85.1 & 177.2 & 175 & 26.2 & 38 & 6.6 & 0.3 & 3.4 & 0.3 & 0 & 0.4 & 0.8 & 0.9 & 0.2 & 0 & 2.6 & 1 & 1.1 & 31.3 & 9.5 & 6.1 \\[1em]
100 & 30.6 & 29.5 & 43.8 & 7.6 & 52.1 & 667.1 & 43.9 & 0.5 & 37.7 & 100.9 & 159.3 & 145.1 & 52 & 345.2 & 221.9 & 46.3 & 23 & 20.6 & 95.1 & 211.7 & 195.2 & 28.7 & 43.3 & 7.2 & 1.1 & 14.1 & 1 & 0 & 0.8 & 2.3 & 3.5 & 3.4 & 0.3 & 7.7 & 4.5 & 4.3 & 33 & 9.8 & 6.5 \\
110 & 32.2 & 31.1 & 46.5 & 8.1 & 57.4 & 767.6 & 41.4 & 0.5 & 34.3 & 92.2 & 157.4 & 162.9 & 135.9 & 337.5 & 256.4 & 51 & 24.7 & 23.5 & 104 & 245.7 & 213.9 & 31.3 & 46.3 & 8.2 & 1.3 & 17.5 & 1 & 0 & 0.8 & 2.1 & 3.8 & 3.8 & 2.8 & 7.7 & 5.6 & 4.8 & 34.5 & 9.9 & 6.8 \\
120 & 33.4 & 32.5 & 49.3 & 8.5 & 48.3 & 670.6 & 29.4 & 0.3 & 24.8 & 65.5 & 113.2 & 137.6 & 178 & 253.5 & 224.8 & 43.1 & 20.3 & 20.4 & 86.6 & 216.4 & 178.9 & 31.7 & 47.7 & 8.3 & 15 & 204 & 10.1 & 0.1 & 8.3 & 22.3 & 38.7 & 42.9 & 43.9 & 83.9 & 65.5 & 55.7 & 35.9 & 9.1 & 7 \\
130 & 34.6 & 33.7 & 52.1 & 8.8 & 51.8 & 739.9 & 26.8 & 0.2 & 23.6 & 61 & 104.3 & 136.6 & 255.9 & 243.2 & 248.9 & 46.7 & 21.3 & 22.4 & 91.6 & 240.4 & 190.5 & 33.4 & 50.4 & 8.9 & 2 & 29.2 & 1.2 & 0 & 1 & 2.6 & 4.6 & 6 & 8.5 & 10.3 & 9.5 & 7.5 & 37.2 & 8.4 & 7.1 \\
140 & 35.8 & 34.9 & 54.8 & 9.1 & 55.2 & 809.5 & 24.7 & 0.1 & 22.6 & 57.6 & 97 & 129.9 & 336.7 & 233.8 & 273.2 & 50.3 & 22.3 & 24.4 & 96.2 & 264.9 & 201.6 & 34.3 & 51.8 & 9.2 & 2 & 28.8 & 1.1 & 0 & 0.9 & 2.4 & 4.2 & 5.4 & 9.7 & 9.4 & 9.4 & 7.2 & 38.4 & 8.4 & 7.2 \\[1em]
150 & 36.8 & 35.9 & 57.4 & 9.4 & 56.8 & 850.4 & 22.7 & 0.1 & 21.1 & 53.7 & 89.7 & 120.9 & 397 & 219.3 & 287.9 & 52.3 & 22.6 & 25.5 & 97.3 & 279.9 & 205.7 & 36.1 & 56.6 & 9.6 & 3.7 & 55.4 & 1.5 & 0 & 1.4 & 3.6 & 6 & 8.1 & 25.4 & 14.6 & 18.2 & 13.2 & 39.4 & 8.2 & 7.3 \\
160 & 37.8 & 36.9 & 60 & 9.7 & 58.2 & 888.3 & 21 & 0.1 & 19.8 & 50 & 83 & 112.2 & 453 & 205.6 & 301.7 & 54.2 & 22.8 & 26.5 & 97.9 & 294.1 & 209.1 & 36.6 & 58.1 & 9.7 & 3.6 & 55.4 & 1.4 & 0 & 1.3 & 3.3 & 5.4 & 7.4 & 27.2 & 13.7 & 18.3 & 13.1 & 40.4 & 8 & 7.3 \\
170 & 38.7 & 37.8 & 62.5 & 10 & 60.5 & 939.9 & 19.8 & 0 & 19 & 47.7 & 78.8 & 106.7 & 512.6 & 197 & 320.2 & 57 & 23.4 & 27.9 & 100 & 312.9 & 215.7 & 37.8 & 61.2 & 10.1 & 2.5 & 39.4 & 0.8 & 0 & 0.8 & 2.1 & 3.4 & 4.6 & 21.1 & 8.6 & 13.1 & 9 & 41.4 & 7.8 & 7.3 \\
180 & 39.5 & 38.7 & 65 & 10.2 & 61.8 & 974.9 & 18.6 & 0 & 17.9 & 45 & 74.1 & 100.5 & 560.1 & 185.9 & 333 & 58.8 & 23.6 & 28.9 & 100.2 & 326.2 & 218.4 & 38.3 & 62.6 & 10.2 & 3.4 & 54 & 1.1 & 0 & 1.1 & 2.7 & 4.5 & 6.1 & 29.7 & 11.1 & 18 & 12.2 & 42.2 & 7.7 & 7.4 \\
190 & 40.3 & 39.5 & 67.5 & 10.4 & 61.7 & 986.6 & 17.2 & 0 & 16.6 & 41.7 & 68.7 & 93.1 & 590.9 & 172.2 & 338 & 59.4 & 23.4 & 29.1 & 98.2 & 331.7 & 216.4 & 39.5 & 65.3 & 10.6 & 4.6 & 74.1 & 1.4 & 0 & 1.3 & 3.3 & 5.4 & 7.4 & 43.3 & 13.7 & 24.8 & 16.1 & 43 & 7.4 & 7.4 \\[1em]
200 & 41 & 40.2 & 70 & 10.7 & 64.2 & 1040.5 & 16.6 & 0 & 16.1 & 40.4 & 66.4 & 90 & 646.2 & 167 & 357.4 & 62.6 & 24.1 & 30.6 & 100.3 & 351.4 & 223.5 & 39.8 & 67.1 & 10.7 & 1.8 & 29.3 & 0.5 & 0 & 0.5 & 1.3 & 2.1 & 2.8 & 17.4 & 5.1 & 9.8 & 6.3 & 43.8 & 7.2 & 7.3 \\
210 & 41.6 & 40.9 & 72.2 & 10.9 & 63.9 & 1046.4 & 15.5 & 0 & 15 & 37.7 & 61.9 & 83.9 & 668.7 & 155.9 & 360.3 & 62.9 & 23.7 & 30.7 & 97.9 & 354.8 & 220.7 & 41.3 & 72.5 & 11 & 4.6 & 75.7 & 1.1 & 0 & 1.1 & 2.7 & 4.4 & 6 & 48.7 & 11.1 & 25.7 & 15.8 & 44.5 & 7.1 & 7.3 \\
220 & 42.3 & 41.6 & 74.6 & 11.1 & 64.8 & 1072.4 & 14.7 & 0 & 14.3 & 35.8 & 58.7 & 79.6 & 703.5 & 148.2 & 370.2 & 64.5 & 23.8 & 31.4 & 97.3 & 365.2 & 222 & 41.6 & 72.8 & 11.2 & 3.2 & 53.4 & 0.8 & 0 & 0.7 & 1.9 & 3 & 4.1 & 34.6 & 7.7 & 18.1 & 11 & 45.2 & 6.9 & 7.3 \\
230 & 42.9 & 42.3 & 77 & 11.3 & 65 & 1087.7 & 13.8 & 0 & 13.4 & 33.7 & 55.3 & 75 & 730.1 & 139.6 & 376.4 & 65.4 & 23.7 & 31.8 & 95.8 & 371.9 & 221.2 & 41.9 & 74.2 & 11.2 & 3.7 & 61.2 & 0.8 & 0 & 0.8 & 2.1 & 3.4 & 4.6 & 40 & 8.6 & 20.8 & 12.5 & 45.9 & 6.7 & 7.3 \\
240 & 43.4 & 42.9 & 79.4 & 11.5 & 65.3 & 1102 & 13.1 & 0 & 12.7 & 31.8 & 52.2 & 70.8 & 754.8 & 131.9 & 382.2 & 66.5 & 23.6 & 32.1 & 94.3 & 378.2 & 220.6 & 43 & 77 & 11.7 & 3.5 & 60.1 & 0.8 & 0 & 0.7 & 1.9 & 3 & 4.1 & 40.4 & 7.7 & 20.6 & 11.9 & 46.5 & 6.5 & 7.3 \\[1em]
250 & 43.9 & 43.5 & 81.9 & 11.7 & 64 & 1088.1 & 12 & 0 & 11.7 & 29.3 & 48.1 & 65.2 & 758.9 & 121.6 & 378.3 & 65.9 & 23 & 31.7 & 90.4 & 374.9 & 214.4 & 43.1 & 79.1 & 11.6 & 5 & 85.5 & 1 & 0 & 1 & 2.5 & 4.1 & 5.5 & 58.5 & 10.3 & 29.3 & 17 & 47.1 & 6.3 & 7.2 \\
260 & 44.4 & 44 & 84.2 & 11.8 & 65.4 & 1119.4 & 11.6 & 0 & 11.3 & 28.3 & 46.4 & 62.9 & 793 & 117.3 & 390.1 & 68.1 & 23.3 & 32.5 & 90.5 & 387 & 217.4 & 44.2 & 80.5 & 12 & 2.2 & 37.9 & 0.4 & 0 & 0.4 & 1 & 1.7 & 2.3 & 26.3 & 4.3 & 13 & 7.3 & 47.7 & 6 & 7.2 \\
270 & 44.9 & 44.6 & 86.5 & 12 & 67.4 & 1161.6 & 11.4 & 0 & 11.1 & 27.7 & 45.4 & 61.6 & 834 & 114.8 & 405.6 & 70.9 & 23.8 & 33.7 & 91.5 & 402.9 & 222.6 & 45.3 & 86.2 & 12.4 & 1.5 & 26 & 0.3 & 0 & 0.2 & 0.6 & 1 & 1.4 & 18.7 & 2.6 & 9 & 4.9 & 48.2 & 6 & 7.1 \\
280 & 45.4 & 45.1 & 88.8 & 12.2 & 70 & 1212.9 & 11.2 & 0 & 10.9 & 27.3 & 44.7 & 60.6 & 881.7 & 113.1 & 424.4 & 74.3 & 24.5 & 35.1 & 93 & 422.1 & 229.3 & 45 & 84.9 & 12.3 & 1 & 16.7 & 0.2 & 0 & 0.2 & 0.4 & 0.7 & 0.9 & 11.9 & 1.7 & 5.8 & 3.1 & 48.8 & 6 & 7.1 \\
290 & 45.8 & 45.6 & 91 & 12.4 & 69 & 1203.8 & 10.5 & 0 & 10.2 & 25.6 & 42 & 57 & 884.8 & 106.2 & 422.1 & 74 & 24.1 & 34.8 & 90 & 420 & 225 & 45.4 & 89.6 & 12.3 & 4.3 & 75 & 0.7 & 0 & 0.7 & 1.7 & 2.7 & 3.7 & 54.8 & 6.9 & 26.1 & 14.1 & 49.3 & 5.8 & 7 \\[1em]
300 & 46.1 & 46 & 93.2 & 12.6 & 70.1 & 1227.8 & 10.2 & 0 & 9.9 & 24.8 & 40.7 & 55.1 & 911.5 & 102.8 & 431.3 & 75.9 & 24.3 & 35.4 & 89.6 & 429.2 & 227.3 & 46.6 & 92 & 12.9 & 2.3 & 40.4 & 0.3 & 0 & 0.3 & 0.8 & 1.4 & 1.8 & 29.9 & 3.4 & 14.1 & 7.3 & 49.7 & 5.7 & 7 \\
   \hline
\end{tabular}
\end{adjustbox}
\end{table}
\begin{table}[ht]
  \begin{adjustbox}{angle=90,height=\textheight}
\centering
\begin{tabular}{rrrrrrrrrrrrrrrrrrrrrr|rrrrrrrrrrrrrrr|rrr}
  \hline
age & ho & hg & dg & hKa & g & v & s.NH & 1b & 2a & 2b & 3a & 3b & 4+ & n & BmS & BmA & BmN & BmR & BmW & BmHarv & BmResid & hgAus & dgAus & hKaAus & gAus & vAus & s.NHAus & 1bAus & 2aAus & 2bAus & 3aAus & 3bAus & 4+Aus & nAus & BmHarvAus & BmResidAus & hoRef & lfz & dgz \\
\hline
0 & 0 & 0 & 0 & 0 & 0 & 0 & 0 & 0 & 0 & 0 & 0 & 0 & 0 & 1400 & 0 & 0 & 0 & 0 & 0 & 0 & 0 & 0 & 0 & 0 & 0 & 0 & 0 & 0 & 0 & 0 & 0 & 0 & 0 & 0 & 0 & 0 & 0 & 0 & 0 \\
10 & 2 & 1.9 & 4.1 & 0.1 & 1.4 & 1.4 & 0 & 0 & 0 & 0 & 0 & 0 & 0 & 1122.1 & 0.5 & 4.5 & 4.1 & 0.1 & 1.2 & 0 & 10.4 & 0 & 0 & 0 & 0 & 0 & 0 & 0 & 0 & 0 & 0 & 0 & 0 & 247 & 0 & 0 & 2.1 & 0 & 0 \\
20 & 7.3 & 6.6 & 12 & 0.8 & 7.6 & 25.6 & 14.4 & 0 & 0 & 0 & 0 & 0 & 0 & 673.3 & 8.3 & 11.4 & 8.3 & 1 & 8.5 & 5.8 & 31.8 & 6.6 & 12.1 & 0.8 & 5.1 & 17.2 & 9.7 & 0 & 0 & 0 & 0 & 0 & 0 & 448.8 & 3.9 & 21.4 & 8 & 2.4 & 1.2 \\
30 & 12.8 & 11.9 & 18.1 & 2.3 & 17.3 & 97.3 & 65.7 & 1.7 & 0 & 0 & 0 & 0 & 0 & 668.4 & 31.3 & 19 & 13 & 3.5 & 24.7 & 26.2 & 65.2 & 8.6 & 14.5 & 1.3 & 0.1 & 0.3 & 0.2 & 0 & 0 & 0 & 0 & 0 & 0 & 4.9 & 0.1 & 0.3 & 14.3 & 5.3 & 2.6 \\
40 & 17.6 & 16.5 & 22.9 & 3.8 & 27.4 & 211.9 & 86.6 & 30.6 & 38.9 & 0 & 0 & 0 & 0 & 662.7 & 68.5 & 25.7 & 16.8 & 7.3 & 44.2 & 60.5 & 102 & 13.8 & 19.7 & 3 & 0.2 & 1.1 & 0.7 & 0.1 & 0 & 0 & 0 & 0 & 0 & 5.7 & 0.3 & 0.6 & 19.5 & 9 & 4.2 \\[1em]
50 & 21.1 & 20.2 & 27.5 & 5 & 20.6 & 192.9 & 42.9 & 20.7 & 70.2 & 13.3 & 0 & 0 & 0 & 346.8 & 62.7 & 18 & 11.2 & 6.4 & 35.6 & 56.9 & 77 & 18.2 & 24.6 & 4.4 & 15 & 128.6 & 33.3 & 21.2 & 41.6 & 0 & 0 & 0 & 0 & 316 & 37.3 & 55.8 & 23.7 & 8.7 & 5.1 \\
60 & 24.3 & 23.4 & 31.8 & 6 & 27.2 & 289.4 & 49.8 & 5.3 & 68.4 & 91.5 & 10.5 & 0 & 0 & 341.9 & 94.5 & 23.2 & 13.6 & 9.4 & 48.8 & 87.3 & 102.2 & 21.8 & 29.8 & 5.5 & 0.3 & 3.4 & 0.7 & 0.2 & 1 & 0.8 & 0 & 0 & 0 & 4.9 & 1 & 1.3 & 27 & 8.1 & 5.6 \\
70 & 27 & 26 & 35.7 & 6.8 & 34 & 397.9 & 48.3 & 1.7 & 60.9 & 105.6 & 96.4 & 2 & 0 & 339.5 & 130.7 & 28.7 & 16 & 12.6 & 62.2 & 122.2 & 128.1 & 23.6 & 32.1 & 6.1 & 0.2 & 2.1 & 0.4 & 0 & 0.5 & 0.8 & 0 & 0 & 0 & 2.4 & 0.6 & 0.7 & 29.8 & 9.1 & 6.1 \\
80 & 29.3 & 28.3 & 39.3 & 7.6 & 40.7 & 512.1 & 46.5 & 0.8 & 49.7 & 110 & 130.3 & 72.6 & 0 & 335.4 & 169.1 & 34.2 & 18.3 & 16 & 74.7 & 159.5 & 152.8 & 26.8 & 36 & 7.3 & 0.4 & 4.9 & 0.6 & 0 & 0.8 & 1.2 & 1 & 0.3 & 0 & 4.1 & 1.5 & 1.5 & 32.1 & 9.9 & 6.6 \\
90 & 31.4 & 30.4 & 42.6 & 8.2 & 47.5 & 633.2 & 45.6 & 0.7 & 40.4 & 105.5 & 156.8 & 139.4 & 23.1 & 333 & 210.1 & 39.8 & 20.5 & 19.6 & 86.9 & 199.5 & 177.4 & 28.5 & 38.8 & 7.8 & 0.3 & 3.6 & 0.4 & 0 & 0.4 & 0.9 & 0.8 & 0.4 & 0 & 2.4 & 1.1 & 1.1 & 34.2 & 10.4 & 7 \\[1em]
100 & 33.2 & 32.2 & 45.6 & 8.8 & 53.6 & 751.4 & 43.2 & 0.6 & 35 & 94.1 & 162.3 & 157.6 & 118.6 & 328.1 & 250.4 & 45 & 22.5 & 23.1 & 97.5 & 239.2 & 199.2 & 32 & 44.1 & 8.9 & 0.7 & 10.3 & 0.7 & 0 & 0.6 & 1.5 & 2.4 & 2.2 & 1.1 & 4.9 & 3.3 & 2.7 & 36 & 10.8 & 7.4 \\
110 & 34.9 & 33.8 & 48.4 & 9.3 & 60 & 876.2 & 41.7 & 0.6 & 33 & 87.4 & 157.3 & 183.4 & 213.8 & 325.7 & 293.1 & 50.3 & 24.5 & 26.7 & 107.9 & 281.3 & 221.3 & 33.1 & 48.2 & 9 & 0.4 & 6.3 & 0.3 & 0 & 0.2 & 0.6 & 1.1 & 1.2 & 1.7 & 2.4 & 2 & 1.6 & 37.6 & 11.1 & 7.7 \\
120 & 36.3 & 35.3 & 51.1 & 9.8 & 52.5 & 794.9 & 32.2 & 0.4 & 25.3 & 67.1 & 121.1 & 157.1 & 250.6 & 256 & 267 & 44.2 & 20.9 & 24.1 & 93.4 & 257.2 & 192.4 & 34.5 & 49.9 & 9.5 & 13.6 & 202.3 & 8.8 & 0.1 & 6.9 & 18.4 & 33.3 & 41.7 & 56.7 & 69.7 & 65.2 & 50.1 & 39.1 & 10.3 & 7.9 \\
130 & 37.6 & 36.6 & 53.9 & 10.2 & 55.8 & 868.6 & 30.2 & 0.4 & 24 & 63.5 & 113.8 & 152.5 & 333.4 & 244.7 & 292.7 & 47.4 & 21.8 & 26.2 & 97.7 & 283 & 202.8 & 36.5 & 53.5 & 10.2 & 2.5 & 39.6 & 1.4 & 0 & 1.1 & 2.9 & 5.3 & 7.2 & 14.8 & 11.3 & 12.9 & 9.3 & 40.4 & 9.7 & 8.1 \\
140 & 38.8 & 37.8 & 56.6 & 10.5 & 59.6 & 950 & 28 & 0.3 & 23.1 & 60.4 & 106.5 & 144.1 & 426 & 236.6 & 321.3 & 51.2 & 22.9 & 28.5 & 102.6 & 311.6 & 214.8 & 37.8 & 54.7 & 10.7 & 1.9 & 30.5 & 1 & 0 & 0.8 & 2.1 & 3.8 & 5.2 & 12.4 & 8.1 & 10 & 6.9 & 41.6 & 9.6 & 8.2 \\[1em]
150 & 39.8 & 38.9 & 59.2 & 10.8 & 61.1 & 994.8 & 25.3 & 0.2 & 21.5 & 56 & 97.1 & 131.7 & 496.7 & 222 & 337.4 & 53 & 23.2 & 29.7 & 103.5 & 328.2 & 218.6 & 39.2 & 58.8 & 11 & 4 & 65.1 & 1.8 & 0 & 1.4 & 3.8 & 6.7 & 9.1 & 31.5 & 14.6 & 21.5 & 14.1 & 42.7 & 9.4 & 8.3 \\
160 & 40.9 & 39.9 & 61.8 & 11.1 & 64 & 1062.2 & 23.6 & 0.2 & 20.6 & 53.2 & 91.3 & 123.9 & 575 & 213.1 & 361.4 & 56.1 & 23.9 & 31.6 & 106.4 & 352.4 & 227 & 39.4 & 59.2 & 11.1 & 2.4 & 40.2 & 1 & 0 & 0.9 & 2.3 & 4 & 5.4 & 20 & 8.9 & 13.3 & 8.7 & 43.8 & 9.3 & 8.3 \\
170 & 41.8 & 40.9 & 64.3 & 11.4 & 66.6 & 1124.2 & 22.1 & 0.1 & 19.8 & 50.7 & 86.1 & 116.7 & 646.8 & 205 & 383.6 & 59 & 24.6 & 33.3 & 108.7 & 374.9 & 234.3 & 41.3 & 63.9 & 11.7 & 2.6 & 44.2 & 0.9 & 0 & 0.8 & 2 & 3.5 & 4.8 & 25 & 8.1 & 14.7 & 9.1 & 44.7 & 9.2 & 8.4 \\
180 & 42.7 & 41.8 & 66.8 & 11.7 & 67.3 & 1154.5 & 20 & 0.1 & 18.5 & 47 & 78.7 & 106.8 & 699.3 & 192 & 395.1 & 60.3 & 24.5 & 34.1 & 107.8 & 387 & 234.9 & 41.6 & 64.6 & 11.8 & 4.2 & 73.1 & 1.4 & 0 & 1.2 & 3.2 & 5.5 & 7.5 & 42.5 & 13 & 24.4 & 14.9 & 45.7 & 8.9 & 8.4 \\
190 & 43.5 & 42.7 & 69.2 & 12 & 67.1 & 1165.7 & 18.3 & 0.1 & 17.2 & 43.5 & 72.3 & 98.1 & 732.9 & 178.2 & 400 & 60.7 & 24.2 & 34.4 & 105.4 & 392.5 & 232.1 & 42.7 & 68.6 & 12 & 5.1 & 88.3 & 1.4 & 0 & 1.3 & 3.3 & 5.5 & 7.5 & 55.4 & 13.8 & 29.7 & 17.6 & 46.5 & 8.6 & 8.4 \\[1em]
200 & 44.2 & 43.5 & 71.7 & 12.2 & 68.3 & 1201.7 & 17.1 & 0 & 16.3 & 41.1 & 67.8 & 92 & 780.8 & 169.3 & 413.4 & 62.5 & 24.4 & 35.3 & 105.2 & 406.4 & 234.4 & 43.7 & 70.1 & 12.5 & 3.4 & 60.9 & 0.9 & 0 & 0.9 & 2.2 & 3.6 & 4.9 & 39 & 8.9 & 20.6 & 11.8 & 47.3 & 8.4 & 8.4 \\
210 & 44.9 & 44.2 & 74 & 12.4 & 67.3 & 1195.4 & 15.7 & 0 & 15.1 & 37.9 & 62.3 & 84.5 & 796.4 & 156.4 & 412.3 & 62.3 & 23.8 & 35 & 101.6 & 406 & 229.1 & 44.7 & 73.6 & 12.7 & 5.5 & 99.4 & 1.3 & 0 & 1.2 & 3.1 & 5.1 & 7 & 66.4 & 13 & 33.7 & 18.7 & 48.1 & 8.1 & 8.4 \\
220 & 45.5 & 44.9 & 76.3 & 12.6 & 67.8 & 1216.7 & 14.8 & 0 & 14.3 & 35.9 & 59 & 79.9 & 828.2 & 148.3 & 420.6 & 63.5 & 23.8 & 35.6 & 100.5 & 414.8 & 229.2 & 45.5 & 77 & 12.9 & 3.8 & 68.5 & 0.8 & 0 & 0.8 & 2 & 3.2 & 4.3 & 47.1 & 8.1 & 23.4 & 12.7 & 48.8 & 7.8 & 8.4 \\
230 & 46.2 & 45.6 & 78.9 & 12.8 & 69.7 & 1261.5 & 14.2 & 0 & 13.7 & 34.5 & 56.6 & 76.7 & 876.5 & 142.6 & 437.2 & 66.1 & 24.3 & 36.8 & 101 & 431.8 & 233.6 & 45 & 73.4 & 12.9 & 2.4 & 43.2 & 0.6 & 0 & 0.5 & 1.4 & 2.4 & 3.2 & 28.5 & 5.7 & 14.7 & 8.1 & 49.5 & 7.7 & 8.3 \\
240 & 46.8 & 46.2 & 81.3 & 13 & 69.8 & 1272.8 & 13.4 & 0 & 13 & 32.5 & 53.3 & 72.3 & 899.4 & 134.5 & 442.1 & 66.9 & 24.1 & 37.1 & 99.1 & 437.3 & 231.9 & 46.2 & 80 & 13.1 & 4.1 & 74.3 & 0.8 & 0 & 0.8 & 2 & 3.2 & 4.3 & 52.1 & 8.1 & 25.5 & 13.5 & 50.2 & 7.5 & 8.3 \\[1em]
250 & 47.3 & 46.8 & 83.6 & 13.2 & 71.1 & 1307 & 12.9 & 0 & 12.5 & 31.3 & 51.3 & 69.6 & 937.3 & 129.6 & 455 & 68.9 & 24.4 & 38 & 99 & 450.7 & 234.6 & 47.1 & 83.7 & 13.4 & 2.7 & 49.5 & 0.5 & 0 & 0.5 & 1.2 & 1.9 & 2.6 & 35.6 & 4.9 & 17.1 & 8.8 & 50.8 & 7.3 & 8.3 \\
260 & 47.8 & 47.4 & 85.9 & 13.4 & 70.9 & 1313 & 12.1 & 0 & 11.8 & 29.5 & 48.5 & 65.7 & 954.3 & 122.3 & 458.1 & 69.5 & 24.1 & 38.1 & 96.7 & 454.3 & 232.2 & 47.3 & 84.5 & 13.5 & 4.1 & 75.5 & 0.7 & 0 & 0.7 & 1.8 & 2.9 & 3.9 & 54.5 & 7.3 & 26.1 & 13.4 & 51.4 & 7.1 & 8.2 \\
270 & 48.3 & 47.9 & 88.3 & 13.6 & 71.9 & 1338.5 & 11.6 & 0 & 11.3 & 28.3 & 46.5 & 63 & 984.7 & 117.5 & 468.1 & 71.2 & 24.3 & 38.8 & 96 & 464.7 & 233.6 & 48.4 & 87.2 & 13.9 & 2.8 & 53.8 & 0.5 & 0 & 0.5 & 1.2 & 2 & 2.7 & 39.2 & 4.9 & 18.7 & 9.2 & 52 & 7 & 8.2 \\
280 & 48.8 & 48.5 & 90.6 & 13.8 & 71.1 & 1332.5 & 10.9 & 0 & 10.6 & 26.6 & 43.6 & 59.1 & 991.3 & 110.2 & 466.9 & 71.2 & 23.9 & 38.5 & 93 & 464.2 & 229.4 & 48.2 & 88 & 13.8 & 4.4 & 83 & 0.7 & 0 & 0.7 & 1.8 & 2.9 & 3.9 & 61.1 & 7.3 & 28.8 & 14.4 & 52.5 & 6.8 & 8.1 \\
290 & 49.2 & 49.1 & 93 & 14 & 71 & 1338.7 & 10.3 & 0 & 10.1 & 25.2 & 41.3 & 56.1 & 1006.1 & 104.5 & 470.1 & 71.8 & 23.7 & 38.6 & 91 & 467.8 & 227.4 & 48.3 & 91 & 13.7 & 3.7 & 68.3 & 0.6 & 0 & 0.5 & 1.4 & 2.2 & 3 & 50.8 & 5.7 & 23.8 & 11.8 & 53 & 6.6 & 8.1 \\[1em]
300 & 49.5 & 49.6 & 95.4 & 14.2 & 70.6 & 1339.5 & 9.8 & 0 & 9.5 & 23.9 & 39.1 & 53 & 1016.2 & 98.8 & 471.4 & 72.1 & 23.4 & 38.6 & 88.6 & 469 & 225.1 & 48.8 & 93 & 13.9 & 3.8 & 71.5 & 0.6 & 0 & 0.5 & 1.4 & 2.2 & 3 & 53.6 & 5.7 & 25 & 12.2 & 53.6 & 6.4 & 8 \\
   \hline
\end{tabular}
\end{adjustbox}
\end{table}
\begin{table}[ht]
  \begin{adjustbox}{angle=90,height=\textheight}
\centering
\begin{tabular}{rrrrrrrrrrrrrrrrrrrrrr|rrrrrrrrrrrrrrr|rrr}
  \hline
age & ho & hg & dg & hKa & g & v & s.NH & 1b & 2a & 2b & 3a & 3b & 4+ & n & BmS & BmA & BmN & BmR & BmW & BmHarv & BmResid & hgAus & dgAus & hKaAus & gAus & vAus & s.NHAus & 1bAus & 2aAus & 2bAus & 3aAus & 3bAus & 4+Aus & nAus & BmHarvAus & BmResidAus & hoRef & lfz & dgz \\
\hline
0 & 0 & 0 & 0 & 0 & 0 & 0 & 0 & 0 & 0 & 0 & 0 & 0 & 0 & 1320 & 0 & 0 & 0 & 0 & 0 & 0 & 0 & 0 & 0 & 0 & 0 & 0 & 0 & 0 & 0 & 0 & 0 & 0 & 0 & 0 & 0 & 0 & 0 & 0 & 0 \\
10 & 2.4 & 2.2 & 5.2 & 0.1 & 2.2 & 3.8 & 0 & 0 & 0 & 0 & 0 & 0 & 0 & 1051 & 1.3 & 6.3 & 5.3 & 0.2 & 1.8 & 0 & 15 & 0 & 0 & 0 & 0 & 0 & 0 & 0 & 0 & 0 & 0 & 0 & 0 & 235.4 & 0 & 0 & 2.5 & 0 & 0 \\
20 & 8.4 & 7.7 & 13.1 & 1.1 & 8.4 & 32 & 19.4 & 0 & 0 & 0 & 0 & 0 & 0 & 618.4 & 10.3 & 11.5 & 8.4 & 1.3 & 9.9 & 7.8 & 33.5 & 7.6 & 13.1 & 1.1 & 5.9 & 22.1 & 13.3 & 0 & 0 & 0 & 0 & 0 & 0 & 432.6 & 5.3 & 23.5 & 9.3 & 3.3 & 1.6 \\
30 & 14.6 & 13.5 & 19.4 & 2.9 & 18.2 & 117.1 & 78 & 4.9 & 0 & 0 & 0 & 0 & 0 & 617.5 & 37.6 & 18.4 & 12.7 & 4.2 & 27 & 32.5 & 67.4 & 7.5 & 12.6 & 1.1 & 0 & 0 & 0 & 0 & 0 & 0 & 0 & 0 & 0 & 0.9 & 0 & 0 & 16.3 & 6.4 & 3.2 \\
40 & 19.7 & 18.4 & 24.2 & 4.6 & 28.4 & 246.6 & 64.7 & 43.3 & 76.2 & 0 & 0 & 0 & 0 & 615.6 & 79.8 & 24.7 & 16.3 & 8.4 & 46.9 & 72 & 104 & 15.2 & 21.5 & 3.4 & 0.1 & 0.5 & 0.3 & 0 & 0 & 0 & 0 & 0 & 0 & 1.8 & 0.1 & 0.2 & 21.8 & 10.2 & 4.9 \\[1em]
50 & 23.3 & 22.4 & 28.8 & 6 & 21.1 & 219.6 & 40.2 & 16 & 72.1 & 40.7 & 0 & 0 & 0 & 323.5 & 71.4 & 17.3 & 10.8 & 7.2 & 37 & 66 & 77.7 & 20.3 & 26 & 5.3 & 15.5 & 147.5 & 29.6 & 24 & 57.8 & 0.2 & 0 & 0 & 0 & 292.1 & 43.7 & 56.5 & 26.1 & 9.7 & 5.9 \\
60 & 26.6 & 25.6 & 33.2 & 7.1 & 28 & 328.6 & 47.2 & 4.6 & 68.7 & 101 & 36.7 & 0 & 0 & 323.5 & 107.5 & 22.4 & 13.2 & 10.5 & 50.7 & 101 & 103.5 & 0 & 0 & 0 & 0 & 0 & 0 & 0 & 0 & 0 & 0 & 0 & 0 & 0 & 0 & 0 & 29.6 & 8.9 & 6.4 \\
70 & 29.4 & 28.4 & 37.2 & 8 & 34.5 & 442.3 & 46.5 & 1.8 & 57.5 & 110.6 & 115.3 & 20.9 & 0 & 318 & 145.5 & 27.4 & 15.4 & 14 & 63.2 & 138.1 & 127.2 & 26.2 & 34.9 & 7.1 & 0.5 & 6.3 & 0.8 & 0 & 1 & 1.7 & 1.4 & 0 & 0 & 5.5 & 1.9 & 2 & 32.5 & 9.9 & 6.9 \\
80 & 31.9 & 30.9 & 40.8 & 8.8 & 41.4 & 569.5 & 45.8 & 1.1 & 48.4 & 106.7 & 144.5 & 110.3 & 2.1 & 317.1 & 188.4 & 32.7 & 17.6 & 17.7 & 75.9 & 180.3 & 152.1 & 29 & 42.2 & 7.5 & 0.1 & 1.6 & 0.1 & 0 & 0.1 & 0.3 & 0.4 & 0.4 & 0 & 0.9 & 0.5 & 0.5 & 35 & 10.8 & 7.4 \\
90 & 34.1 & 33 & 44.1 & 9.5 & 48.1 & 700.7 & 44 & 0.9 & 41.6 & 95.9 & 160.6 & 144.1 & 82.3 & 315.2 & 232.9 & 38 & 19.8 & 21.6 & 87.8 & 224.3 & 175.7 & 32.6 & 43.6 & 9.4 & 0.3 & 4 & 0.3 & 0 & 0.2 & 0.5 & 0.9 & 0.8 & 0.4 & 1.8 & 1.3 & 1 & 37.1 & 11.4 & 7.8 \\[1em]
100 & 36 & 34.9 & 47.2 & 10.1 & 54.5 & 830.4 & 41.9 & 0.8 & 36.7 & 87.2 & 156.2 & 171.4 & 185.1 & 311.5 & 277.2 & 43 & 21.7 & 25.4 & 98.4 & 268.4 & 197.4 & 34.1 & 45.1 & 10 & 0.6 & 8.8 & 0.5 & 0 & 0.5 & 1.1 & 1.9 & 1.4 & 1.8 & 3.7 & 2.8 & 2.1 & 39 & 11.7 & 8.2 \\
110 & 37.8 & 36.6 & 50.1 & 10.7 & 60.7 & 963.8 & 40.5 & 0.7 & 33.9 & 83.3 & 150.4 & 190.9 & 293.2 & 308.7 & 323.1 & 48.1 & 23.6 & 29.3 & 108.5 & 314.2 & 218.4 & 36.2 & 48 & 10.8 & 0.5 & 7.9 & 0.4 & 0 & 0.4 & 0.8 & 1.4 & 1.7 & 1.9 & 2.8 & 2.6 & 1.8 & 40.7 & 12 & 8.6 \\
120 & 39.2 & 38.2 & 53 & 11.2 & 53.3 & 875.2 & 31.2 & 0.5 & 25.6 & 64.3 & 116.1 & 154.4 & 331.6 & 242.2 & 294.5 & 42.5 & 20.2 & 26.4 & 93.8 & 287.5 & 190.1 & 37.3 & 51.1 & 10.9 & 13.7 & 220 & 8.7 & 0.1 & 7.3 & 17.9 & 32.3 & 42.2 & 72.7 & 66.6 & 72 & 48.9 & 42.2 & 11.2 & 8.8 \\
130 & 40.6 & 39.6 & 55.8 & 11.6 & 57.3 & 965.4 & 29.8 & 0.5 & 23.9 & 61.5 & 110.9 & 149.9 & 425.4 & 233.9 & 326 & 46.2 & 21.3 & 29 & 99.1 & 319.3 & 202.3 & 39.4 & 54.9 & 11.6 & 2 & 33 & 1.1 & 0 & 0.8 & 2.2 & 3.9 & 5.4 & 13.9 & 8.3 & 10.9 & 6.9 & 43.6 & 10.5 & 8.9 \\
140 & 41.9 & 40.9 & 58.6 & 12 & 62.3 & 1076 & 29.2 & 0.4 & 23.4 & 60.5 & 109 & 148 & 526.6 & 231.1 & 364.6 & 50.7 & 22.8 & 32.2 & 105.9 & 358.2 & 218.1 & 40.7 & 57.7 & 12 & 0.7 & 12.5 & 0.3 & 0 & 0.3 & 0.7 & 1.3 & 1.8 & 6 & 2.8 & 4.2 & 2.5 & 44.8 & 10.6 & 9 \\[1em]
150 & 43.1 & 42.1 & 61.3 & 12.4 & 65.4 & 1153.4 & 27.4 & 0.4 & 22 & 57.5 & 103 & 139.9 & 614.8 & 221.8 & 392.1 & 53.8 & 23.6 & 34.4 & 109.1 & 386.2 & 226.8 & 41.7 & 60 & 12.3 & 2.6 & 45.7 & 1.2 & 0 & 0.9 & 2.4 & 4.4 & 6 & 23.3 & 9.2 & 15.3 & 9 & 46 & 10.6 & 9.1 \\
160 & 44.2 & 43.2 & 64 & 12.7 & 67.6 & 1214 & 25.3 & 0.3 & 20.6 & 53.9 & 95.6 & 129.9 & 693.6 & 209.8 & 414.1 & 56.3 & 24.1 & 36 & 110.5 & 408.8 & 232.1 & 42.6 & 59.4 & 12.9 & 3.3 & 59.7 & 1.5 & 0 & 1.3 & 3.2 & 5.8 & 7.7 & 30.3 & 12 & 19.9 & 11.5 & 47.1 & 10.4 & 9.2 \\
170 & 45.1 & 44.2 & 66.7 & 13 & 68.2 & 1245.1 & 22.6 & 0.2 & 19 & 49.5 & 86.4 & 117.3 & 753.3 & 195 & 426 & 57.5 & 24 & 36.8 & 109.3 & 421.5 & 232.2 & 44 & 63.5 & 13.2 & 4.7 & 85.9 & 1.9 & 0 & 1.5 & 3.9 & 7 & 9.5 & 48.4 & 14.8 & 28.9 & 16 & 48.1 & 10.1 & 9.3 \\
180 & 46.1 & 45.2 & 69.3 & 13.3 & 69.8 & 1293 & 20.8 & 0.2 & 17.9 & 46.4 & 80 & 108.6 & 817.7 & 184.9 & 443.6 & 59.5 & 24.3 & 38.1 & 109.5 & 439.8 & 235.3 & 44.5 & 67 & 13.1 & 3.6 & 65.7 & 1.2 & 0 & 1 & 2.6 & 4.6 & 6.3 & 39.6 & 10.2 & 22.3 & 12.1 & 49.1 & 9.9 & 9.3 \\
190 & 46.9 & 46 & 71.9 & 13.5 & 70.2 & 1316.3 & 18.8 & 0.1 & 16.7 & 42.9 & 73 & 99 & 863.6 & 172.9 & 452.9 & 60.5 & 24.2 & 38.7 & 107.9 & 449.8 & 234.3 & 46.1 & 70 & 13.8 & 4.6 & 86.9 & 1.4 & 0 & 1.2 & 3 & 5.2 & 7.1 & 55.5 & 12 & 29.6 & 15.4 & 50 & 9.6 & 9.3 \\[1em]
200 & 47.7 & 46.9 & 74.4 & 13.8 & 68.8 & 1305.5 & 16.6 & 0.1 & 15.2 & 38.8 & 65.1 & 88.3 & 883.4 & 158.1 & 450.4 & 60 & 23.4 & 38.3 & 103.5 & 448.1 & 227.4 & 46.5 & 73.3 & 13.7 & 6.2 & 117.3 & 1.6 & 0 & 1.4 & 3.6 & 6.1 & 8.3 & 78.3 & 14.8 & 40.2 & 20.6 & 50.8 & 9.3 & 9.3 \\
210 & 48.5 & 47.7 & 77 & 14.1 & 71 & 1362.8 & 15.7 & 0.1 & 14.7 & 37.2 & 62.1 & 84.2 & 944.6 & 152.5 & 471.3 & 62.6 & 24 & 39.8 & 104.5 & 469 & 233.3 & 46.7 & 73.9 & 13.7 & 2.4 & 45 & 0.6 & 0 & 0.5 & 1.4 & 2.3 & 3.1 & 30.3 & 5.5 & 15.4 & 7.9 & 51.6 & 8.9 & 9.3 \\
220 & 49.2 & 48.5 & 79.5 & 14.3 & 72.1 & 1396.6 & 14.7 & 0 & 14 & 35.3 & 58.4 & 79.2 & 988.2 & 145.1 & 484.3 & 64.4 & 24.1 & 40.7 & 103.9 & 481.8 & 235.6 & 48.3 & 76.5 & 14.4 & 3.4 & 65.7 & 0.8 & 0 & 0.7 & 1.8 & 3.1 & 4.2 & 45.3 & 7.4 & 22.6 & 11.1 & 52.4 & 8.7 & 9.3 \\
230 & 49.8 & 49.2 & 82.1 & 14.5 & 71.9 & 1403.9 & 13.7 & 0 & 13.1 & 32.9 & 54.3 & 73.6 & 1010.7 & 135.9 & 488 & 65.1 & 23.9 & 40.8 & 101.3 & 485.6 & 233.6 & 49.3 & 79.2 & 14.8 & 4.5 & 89.9 & 1 & 0 & 0.9 & 2.3 & 3.9 & 5.2 & 63.3 & 9.2 & 31 & 14.8 & 53.1 & 8.5 & 9.2 \\
240 & 50.4 & 49.9 & 84.6 & 14.8 & 70.7 & 1391.9 & 12.6 & 0 & 12.1 & 30.4 & 50.2 & 68 & 1016.9 & 125.7 & 485 & 64.7 & 23.3 & 40.4 & 97.5 & 482.6 & 228.3 & 49.8 & 82.8 & 14.9 & 5.4 & 106.9 & 1.1 & 0 & 1 & 2.5 & 4.2 & 5.6 & 76.9 & 10.2 & 37 & 17.5 & 53.8 & 8.3 & 9.2 \\[1em]
250 & 50.9 & 50.5 & 87.1 & 15 & 69.4 & 1374.5 & 11.6 & 0 & 11.2 & 28.2 & 46.3 & 62.8 & 1017.3 & 116.5 & 480 & 64.3 & 22.7 & 39.8 & 93.6 & 477.6 & 222.9 & 50.9 & 86 & 15.2 & 5.3 & 107 & 0.9 & 0 & 0.9 & 2.2 & 3.7 & 5 & 79 & 9.2 & 37.2 & 17.2 & 54.4 & 7.9 & 9.1 \\
260 & 51.4 & 51.1 & 89.6 & 15.2 & 70 & 1395 & 11 & 0 & 10.7 & 26.8 & 44 & 59.7 & 1044.6 & 110.9 & 488.3 & 65.6 & 22.7 & 40.3 & 92.3 & 485.9 & 223.5 & 51 & 87.4 & 15.2 & 3.3 & 66.6 & 0.5 & 0 & 0.5 & 1.3 & 2.2 & 3 & 49.5 & 5.5 & 23.2 & 10.7 & 55 & 7.7 & 9.1 \\
270 & 51.9 & 51.7 & 92.2 & 15.4 & 70.3 & 1409.5 & 10.5 & 0 & 10.1 & 25.5 & 41.8 & 56.7 & 1066.6 & 105.4 & 494.5 & 66.7 & 22.7 & 40.7 & 90.7 & 492 & 223.2 & 51.3 & 90.1 & 15.3 & 3.5 & 70.5 & 0.5 & 0 & 0.5 & 1.3 & 2.2 & 3 & 52.9 & 5.5 & 24.6 & 11.2 & 55.6 & 7.5 & 9 \\
280 & 52.2 & 52.3 & 95 & 15.6 & 70.2 & 1414.3 & 9.8 & 0 & 9.5 & 23.9 & 39.2 & 53.2 & 1081.8 & 98.9 & 497.4 & 67.5 & 22.5 & 40.8 & 88.3 & 494.9 & 221.6 & 51.4 & 86.9 & 15.5 & 3.8 & 77.6 & 0.7 & 0 & 0.6 & 1.6 & 2.6 & 3.6 & 57.4 & 6.5 & 26.9 & 12.3 & 56.2 & 7.3 & 9 \\
290 & 52.7 & 52.8 & 97.6 & 15.8 & 71.2 & 1440.7 & 9.4 & 0 & 9.2 & 23 & 37.7 & 51.1 & 1111.5 & 95.2 & 507.8 & 69.4 & 22.7 & 41.4 & 87.6 & 505.2 & 223.6 & 53.6 & 94.7 & 16.3 & 2.6 & 53.8 & 0.4 & 0 & 0.4 & 0.9 & 1.5 & 2 & 41.3 & 3.7 & 18.8 & 8.2 & 56.8 & 7.1 & 8.9 \\[1em]
300 & 53.2 & 53.4 & 100 & 16 & 72.6 & 1477 & 9.2 & 0 & 8.9 & 22.3 & 36.6 & 49.6 & 1148.4 & 92.4 & 521.7 & 71.5 & 23 & 42.4 & 87.4 & 519.1 & 226.9 & 52.7 & 98.5 & 15.7 & 2.1 & 42.4 & 0.3 & 0 & 0.3 & 0.7 & 1.1 & 1.5 & 32.8 & 2.8 & 14.9 & 6.6 & 57.3 & 7 & 8.8 \\
   \hline
\end{tabular}
\end{adjustbox}
\end{table}
\begin{table}[ht]
  \begin{adjustbox}{angle=90,height=\textheight}
\centering
\begin{tabular}{rrrrrrrrrrrrrrrrrrrrrr|rrrrrrrrrrrrrrr|rrr}
  \hline
age & ho & hg & dg & hKa & g & v & s.NH & 1b & 2a & 2b & 3a & 3b & 4+ & n & BmS & BmA & BmN & BmR & BmW & BmHarv & BmResid & hgAus & dgAus & hKaAus & gAus & vAus & s.NHAus & 1bAus & 2aAus & 2bAus & 3aAus & 3bAus & 4+Aus & nAus & BmHarvAus & BmResidAus & hoRef & lfz & dgz \\
\hline
0 & 0 & 0 & 0 & 0 & 0 & 0 & 0 & 0 & 0 & 0 & 0 & 0 & 0 & 1240 & 0 & 0 & 0 & 0 & 0 & 0 & 0 & 0 & 0 & 0 & 0 & 0 & 0 & 0 & 0 & 0 & 0 & 0 & 0 & 0 & 0 & 0 & 0 & 0 & 0 \\
10 & 2.9 & 2.6 & 6.3 & 0.1 & 3.1 & 5.4 & 0 & 0 & 0 & 0 & 0 & 0 & 0 & 987.3 & 1.8 & 8.2 & 6.4 & 0.3 & 2.5 & 0 & 19.2 & 0 & 0 & 0 & 0 & 0 & 0 & 0 & 0 & 0 & 0 & 0 & 0 & 216.6 & 0 & 0 & 3 & 0 & 0 \\
20 & 9.6 & 8.8 & 14.2 & 1.4 & 9.2 & 40 & 25.3 & 0 & 0 & 0 & 0 & 0 & 0 & 580.9 & 12.8 & 11.5 & 8.4 & 1.5 & 11.4 & 10 & 35.8 & 8.7 & 14.3 & 1.4 & 6.5 & 27.8 & 17.6 & 0 & 0 & 0 & 0 & 0 & 0 & 406.4 & 7 & 25.2 & 10.8 & 4.3 & 2.1 \\
30 & 16.2 & 15.1 & 20.5 & 3.5 & 19.2 & 138.8 & 88.5 & 11 & 0.4 & 0 & 0 & 0 & 0 & 579.2 & 44.7 & 17.9 & 12.4 & 4.9 & 29.3 & 38.9 & 70.3 & 12.9 & 18 & 2.8 & 0 & 0.3 & 0.2 & 0 & 0 & 0 & 0 & 0 & 0 & 1.7 & 0.1 & 0.2 & 18.3 & 7.5 & 3.9 \\
40 & 21.6 & 20.3 & 25.4 & 5.4 & 29.3 & 281.3 & 56.4 & 49.5 & 106.2 & 0.3 & 0 & 0 & 0 & 575.7 & 91.2 & 23.8 & 15.7 & 9.5 & 49.3 & 82.6 & 106.8 & 16.3 & 21.5 & 4 & 0.1 & 1 & 0.5 & 0.1 & 0 & 0 & 0 & 0 & 0 & 3.5 & 0.3 & 0.5 & 24.1 & 11.3 & 5.8 \\[1em]
50 & 25.5 & 24.4 & 30.1 & 7 & 21.7 & 247.4 & 40.3 & 12.7 & 72.7 & 65.6 & 0.9 & 0 & 0 & 305.7 & 80.6 & 16.7 & 10.5 & 8.1 & 38.5 & 74.7 & 79.7 & 22.2 & 27.2 & 6.2 & 15.6 & 164 & 29.2 & 22.2 & 66.1 & 7.8 & 0 & 0 & 0 & 270.1 & 48.8 & 57 & 28.7 & 10.5 & 6.7 \\
60 & 28.9 & 27.8 & 34.5 & 8.1 & 28.3 & 361.3 & 45 & 3.9 & 65 & 105.1 & 66.7 & 0.1 & 0 & 303.1 & 118.4 & 21.4 & 12.7 & 11.5 & 51.5 & 111.3 & 104.2 & 26.3 & 32.8 & 7.6 & 0.2 & 2.7 & 0.4 & 0 & 0.6 & 0.9 & 0.1 & 0 & 0 & 2.6 & 0.8 & 0.8 & 32.3 & 9.6 & 7.2 \\
70 & 31.8 & 30.7 & 38.5 & 9.1 & 35.1 & 487.1 & 45.2 & 1.8 & 55.3 & 109.1 & 125.6 & 53.6 & 0 & 301.3 & 160.6 & 26.3 & 14.9 & 15.3 & 64.3 & 152.5 & 128.9 & 27.8 & 37.2 & 7.6 & 0.2 & 2.4 & 0.2 & 0 & 0.3 & 0.5 & 0.6 & 0.3 & 0 & 1.7 & 0.7 & 0.7 & 35.3 & 10.7 & 7.7 \\
80 & 34.3 & 33.2 & 42.2 & 9.9 & 42.1 & 625.1 & 45.3 & 1.3 & 50 & 100.5 & 156.6 & 132.8 & 19.7 & 301.3 & 207.1 & 31.5 & 17.1 & 19.4 & 77.1 & 198.3 & 153.9 & 0 & 0 & 0 & 0 & 0 & 0 & 0 & 0 & 0 & 0 & 0 & 0 & 0 & 0 & 0 & 37.8 & 11.6 & 8.2 \\
90 & 36.6 & 35.5 & 45.6 & 10.7 & 49 & 767.4 & 43.7 & 1.1 & 44.6 & 90 & 159.7 & 161.2 & 126.3 & 300.4 & 255.5 & 36.8 & 19.2 & 23.6 & 89 & 246.1 & 178 & 33.8 & 41.6 & 10.4 & 0.1 & 1.8 & 0.1 & 0 & 0.1 & 0.3 & 0.5 & 0.4 & 0 & 0.9 & 0.6 & 0.4 & 40 & 12.2 & 8.6 \\[1em]
100 & 38.6 & 37.4 & 48.7 & 11.4 & 55.6 & 911.3 & 41.9 & 1 & 40.9 & 84.4 & 151.8 & 187.9 & 241 & 298.7 & 304.8 & 41.8 & 21.2 & 27.8 & 99.9 & 295 & 200.5 & 36.7 & 48.4 & 11 & 0.3 & 5.1 & 0.2 & 0 & 0.2 & 0.5 & 0.9 & 1 & 1.4 & 1.7 & 1.7 & 1.2 & 42 & 12.7 & 9 \\
110 & 40.5 & 39.2 & 51.6 & 12 & 61.4 & 1046 & 40 & 0.9 & 37.5 & 80.7 & 145.4 & 193.4 & 366.4 & 293.5 & 351.3 & 46.3 & 22.8 & 31.7 & 108.8 & 341.4 & 219.5 & 38.4 & 49.8 & 11.8 & 1 & 17 & 0.7 & 0 & 0.8 & 1.4 & 2.6 & 3.5 & 4.9 & 5.2 & 5.5 & 3.6 & 43.7 & 12.9 & 9.4 \\
120 & 41.9 & 40.9 & 54.5 & 12.5 & 53.9 & 947.7 & 30.9 & 0.6 & 27.8 & 62.5 & 112.9 & 153.1 & 399 & 231 & 319.5 & 40.9 & 19.6 & 28.5 & 93.9 & 311.6 & 190.8 & 40 & 53 & 12.2 & 13.8 & 237.7 & 8.4 & 0.2 & 7.7 & 16.9 & 30.5 & 41.8 & 91.3 & 62.5 & 77.9 & 49 & 45.3 & 11.9 & 9.6 \\
130 & 43.3 & 42.3 & 57.4 & 13 & 58 & 1047 & 29.5 & 0.6 & 25.6 & 60 & 108.4 & 147.7 & 501.2 & 224 & 354.2 & 44.6 & 20.7 & 31.4 & 99.3 & 346.6 & 203.7 & 42.7 & 56.3 & 13.4 & 1.7 & 31.6 & 0.9 & 0 & 0.9 & 1.9 & 3.4 & 4.7 & 14.5 & 6.9 & 10.5 & 6 & 46.8 & 11.3 & 9.7 \\
140 & 44.7 & 43.6 & 60.3 & 13.4 & 62.1 & 1146.6 & 28.1 & 0.5 & 23.4 & 57.6 & 104.1 & 141.7 & 604.4 & 218 & 389.3 & 48.3 & 21.9 & 34.2 & 104.3 & 382 & 216 & 43.8 & 59.4 & 13.6 & 1.7 & 31.3 & 0.8 & 0 & 0.7 & 1.6 & 2.9 & 4 & 16.1 & 6.1 & 10.4 & 5.8 & 48.1 & 11.3 & 9.8 \\[1em]
150 & 45.9 & 44.9 & 63.1 & 13.8 & 63.8 & 1201.6 & 26 & 0.4 & 21.3 & 53.6 & 96.7 & 131.5 & 679.6 & 204.1 & 409.3 & 50.2 & 22.1 & 35.7 & 104.9 & 402.7 & 219.6 & 44 & 60.3 & 13.6 & 3.9 & 73.5 & 1.8 & 0 & 1.6 & 3.7 & 6.7 & 9.2 & 38.4 & 13.9 & 24.5 & 13.7 & 49.3 & 11.1 & 9.9 \\
160 & 47.1 & 46 & 65.8 & 14.1 & 66.8 & 1280.5 & 24.5 & 0.4 & 19.8 & 51.1 & 91.8 & 124.9 & 766.3 & 196.2 & 437.5 & 53.2 & 22.9 & 37.9 & 107.7 & 431.5 & 227.7 & 45.9 & 63.2 & 14.4 & 2.4 & 46.4 & 1 & 0 & 0.9 & 2.1 & 3.8 & 5.3 & 25.8 & 7.8 & 15.6 & 8.2 & 50.4 & 10.9 & 10 \\
170 & 48.1 & 47.1 & 68.6 & 14.4 & 68.7 & 1335.9 & 22.9 & 0.3 & 18.4 & 48.2 & 86.3 & 117.2 & 835.4 & 185.8 & 457.9 & 55.4 & 23.2 & 39.4 & 108.3 & 452.5 & 231.8 & 47 & 64.9 & 14.8 & 3.4 & 67.5 & 1.4 & 0 & 1.1 & 2.8 & 5 & 6.8 & 39.7 & 10.4 & 22.7 & 11.7 & 51.5 & 10.7 & 10 \\
180 & 49 & 48 & 71.2 & 14.7 & 69.5 & 1370.6 & 21.1 & 0.3 & 17.1 & 44.9 & 79.9 & 108.4 & 889.2 & 174.5 & 471.1 & 56.8 & 23.3 & 40.3 & 107.3 & 466.5 & 232.4 & 48.3 & 69.6 & 15 & 4.3 & 85.5 & 1.4 & 0 & 1.1 & 2.9 & 5.3 & 7.2 & 54.4 & 11.3 & 29 & 14.3 & 52.4 & 10.5 & 10 \\
190 & 49.9 & 49 & 73.8 & 15 & 70.2 & 1401.7 & 19.1 & 0.2 & 16 & 41.7 & 73 & 99.1 & 941 & 164.1 & 483.1 & 58 & 23.3 & 41.1 & 106.1 & 479.2 & 232.3 & 48.7 & 72.6 & 15 & 4.3 & 85.6 & 1.3 & 0 & 1 & 2.7 & 4.7 & 6.4 & 56.4 & 10.4 & 29.2 & 14.3 & 53.4 & 10.2 & 10 \\[1em]
200 & 50.7 & 49.8 & 76.4 & 15.3 & 69.7 & 1406.3 & 17.3 & 0.2 & 14.9 & 38.3 & 66.5 & 90.3 & 969.3 & 152 & 486 & 58.4 & 22.9 & 41.1 & 103 & 482.9 & 228.5 & 49.9 & 74.3 & 15.5 & 5.2 & 107 & 1.5 & 0 & 1.2 & 3.1 & 5.7 & 7.7 & 71.7 & 12.2 & 36.6 & 17.3 & 54.2 & 9.8 & 10 \\
210 & 51.5 & 50.6 & 79 & 15.5 & 71.6 & 1458.5 & 16 & 0.1 & 14.2 & 36.3 & 62.1 & 84.2 & 1030.8 & 145.9 & 505.4 & 60.7 & 23.3 & 42.5 & 103.5 & 502.9 & 232.6 & 51.5 & 76.4 & 16.3 & 2.7 & 57.1 & 0.8 & 0 & 0.6 & 1.6 & 2.8 & 3.9 & 38.9 & 6.1 & 19.6 & 8.9 & 55.1 & 9.6 & 10 \\
220 & 52.2 & 51.4 & 81.8 & 15.8 & 72.1 & 1481.2 & 14.7 & 0.1 & 13.3 & 33.9 & 57.3 & 77.8 & 1068.6 & 137.2 & 514.6 & 61.9 & 23.3 & 43.1 & 101.8 & 512 & 232.7 & 51.2 & 77.3 & 16 & 4 & 83.3 & 1 & 0 & 0.9 & 2.2 & 3.9 & 5.2 & 57.8 & 8.7 & 28.6 & 13.1 & 55.9 & 9.3 & 10 \\
230 & 52.8 & 52.1 & 84.3 & 16 & 71.7 & 1485.6 & 13.3 & 0.1 & 12.4 & 31.4 & 52.4 & 71 & 1091.2 & 128.5 & 517.4 & 62.4 & 23 & 43.1 & 99.1 & 514.8 & 230.3 & 52.8 & 82.9 & 16.4 & 4.7 & 98.6 & 1 & 0 & 0.8 & 2.2 & 3.8 & 5.1 & 71.5 & 8.7 & 34.1 & 15 & 56.6 & 9 & 9.9 \\
240 & 53.3 & 52.8 & 86.7 & 16.2 & 71.3 & 1485.8 & 12.3 & 0 & 11.7 & 29.4 & 48.8 & 66.2 & 1105.6 & 120.7 & 518.7 & 62.8 & 22.7 & 43.1 & 96.4 & 516.1 & 227.6 & 53.7 & 87.2 & 16.7 & 4.7 & 98.9 & 0.8 & 0 & 0.8 & 1.9 & 3.2 & 4.3 & 74 & 7.8 & 34.4 & 14.8 & 57.3 & 8.7 & 9.9 \\[1em]
250 & 53.9 & 53.5 & 89.4 & 16.4 & 70.8 & 1486.7 & 11.4 & 0 & 10.9 & 27.4 & 45.3 & 61.4 & 1120.6 & 112.9 & 520.2 & 63.2 & 22.4 & 43 & 93.6 & 517.6 & 224.8 & 53.2 & 85.9 & 16.5 & 4.5 & 95.1 & 0.8 & 0 & 0.8 & 1.9 & 3.1 & 4.2 & 70.8 & 7.8 & 33 & 14.4 & 58 & 8.4 & 9.8 \\
260 & 54.4 & 54.1 & 92.2 & 16.6 & 71.9 & 1516.6 & 10.7 & 0 & 10.4 & 26 & 42.8 & 58 & 1156.9 & 107.7 & 532 & 65 & 22.6 & 43.8 & 92.7 & 529.3 & 226.8 & 54.3 & 84.9 & 17.2 & 2.9 & 62.6 & 0.6 & 0 & 0.5 & 1.3 & 2.2 & 3 & 46 & 5.2 & 21.7 & 9.2 & 58.6 & 8.2 & 9.8 \\
270 & 54.8 & 54.7 & 94.7 & 16.9 & 72.1 & 1530.5 & 10.2 & 0 & 9.9 & 24.8 & 40.7 & 55.1 & 1178.2 & 102.5 & 538 & 66 & 22.5 & 44.1 & 91 & 535.3 & 226.4 & 54.6 & 94 & 16.8 & 3.6 & 76.1 & 0.5 & 0 & 0.5 & 1.3 & 2.1 & 2.8 & 58.4 & 5.2 & 26.6 & 11.3 & 59.3 & 7.9 & 9.7 \\
280 & 55.2 & 55.3 & 97.2 & 17.1 & 72.8 & 1552.7 & 9.7 & 0 & 9.4 & 23.7 & 38.9 & 52.7 & 1205.4 & 98.1 & 547 & 67.4 & 22.6 & 44.7 & 89.9 & 544.3 & 227.3 & 55.9 & 95 & 17.5 & 3.1 & 66.4 & 0.4 & 0 & 0.4 & 1 & 1.7 & 2.3 & 51.3 & 4.3 & 23.3 & 9.6 & 59.9 & 7.8 & 9.6 \\
290 & 55.8 & 55.9 & 99.7 & 17.3 & 74.5 & 1596.3 & 9.5 & 0 & 9.2 & 23.1 & 37.9 & 51.3 & 1248.4 & 95.5 & 563.5 & 69.7 & 22.9 & 45.9 & 89.9 & 560.7 & 231.3 & 56.3 & 99.4 & 17.5 & 2 & 43.6 & 0.3 & 0 & 0.3 & 0.6 & 1 & 1.4 & 34.1 & 2.6 & 15.3 & 6.3 & 60.4 & 7.7 & 9.6 \\[1em]
300 & 56.4 & 56.4 & 102.2 & 17.5 & 74.8 & 1609.9 & 9 & 0 & 8.8 & 22 & 36.1 & 48.9 & 1268.2 & 91.2 & 569.6 & 70.8 & 22.9 & 46.2 & 88.3 & 566.7 & 231 & 55.9 & 98.4 & 17.4 & 3.3 & 70.6 & 0.4 & 0 & 0.4 & 1 & 1.7 & 2.3 & 55.1 & 4.3 & 24.8 & 10.2 & 61 & 7.5 & 9.5 \\
   \hline
\end{tabular}
\end{adjustbox}
\end{table}
\begin{table}[ht]
  \begin{adjustbox}{angle=90,height=\textheight}
\centering
\begin{tabular}{rrrrrrrrrrrrrrrrrrrrrr|rrrrrrrrrrrrrrr|rrr}
  \hline
age & ho & hg & dg & hKa & g & v & s.NH & 1b & 2a & 2b & 3a & 3b & 4+ & n & BmS & BmA & BmN & BmR & BmW & BmHarv & BmResid & hgAus & dgAus & hKaAus & gAus & vAus & s.NHAus & 1bAus & 2aAus & 2bAus & 3aAus & 3bAus & 4+Aus & nAus & BmHarvAus & BmResidAus & hoRef & lfz & dgz \\
\hline
0 & 0 & 0 & 0 & 0 & 0 & 0 & 0 & 0 & 0 & 0 & 0 & 0 & 0 & 1200 & 0 & 0 & 0 & 0 & 0 & 0 & 0 & 0 & 0 & 0 & 0 & 0 & 0 & 0 & 0 & 0 & 0 & 0 & 0 & 0 & 0 & 0 & 0 & 0 & 0 \\
10 & 3.4 & 3.1 & 6.7 & 0.2 & 3.4 & 6.6 & 0 & 0 & 0 & 0 & 0 & 0 & 0 & 955.5 & 2.2 & 7.8 & 6.3 & 0.3 & 2.8 & 0 & 19.5 & 0 & 0 & 0 & 0 & 0 & 0 & 0 & 0 & 0 & 0 & 0 & 0 & 209.6 & 0 & 0 & 3.6 & 0 & 0 \\
20 & 11 & 10.1 & 14.8 & 1.9 & 9.7 & 48.6 & 31.5 & 0 & 0 & 0 & 0 & 0 & 0 & 563 & 15.6 & 10.8 & 8.1 & 1.9 & 12.3 & 12.4 & 36.2 & 10.1 & 14.9 & 1.9 & 6.8 & 33.9 & 22 & 0 & 0 & 0 & 0 & 0 & 0 & 392.4 & 8.7 & 25.5 & 12.4 & 5.4 & 2.7 \\
30 & 18 & 16.8 & 21.1 & 4.3 & 19.7 & 160.4 & 88.4 & 25.2 & 3.1 & 0 & 0 & 0 & 0 & 561.3 & 51.6 & 16.6 & 11.8 & 5.6 & 30.5 & 45.3 & 70.8 & 16.3 & 20.2 & 4.2 & 0.1 & 0.4 & 0.2 & 0.1 & 0 & 0 & 0 & 0 & 0 & 1.7 & 0.1 & 0.2 & 20.5 & 8.6 & 4.6 \\
40 & 23.6 & 22.2 & 26 & 6.5 & 29.5 & 314.3 & 53.5 & 57.5 & 126.4 & 1.6 & 0 & 0 & 0 & 558 & 101.9 & 22 & 14.9 & 10.5 & 50.1 & 92.7 & 106.7 & 20.6 & 25.5 & 5.7 & 0.2 & 1.7 & 0.3 & 0.3 & 0.6 & 0 & 0 & 0 & 0 & 3.4 & 0.5 & 0.6 & 26.5 & 12.4 & 6.6 \\[1em]
50 & 27.5 & 26.5 & 30.6 & 8.1 & 21.6 & 270.1 & 36.6 & 15.6 & 79.2 & 76.9 & 2.5 & 0 & 0 & 295 & 88.1 & 15.4 & 9.9 & 8.8 & 38.5 & 81.8 & 78.8 & 24.2 & 27.6 & 7.4 & 15.8 & 182.2 & 26.5 & 25.4 & 72.9 & 15.3 & 0 & 0 & 0 & 263 & 54.4 & 57.1 & 31.2 & 11.2 & 7.5 \\
60 & 31.1 & 30.1 & 35 & 9.4 & 28.1 & 391.3 & 43.2 & 5.3 & 68.6 & 111.9 & 80.9 & 1.1 & 0 & 292.4 & 128.3 & 19.7 & 11.9 & 12.5 & 51.3 & 120.8 & 102.8 & 26.2 & 29.7 & 8.2 & 0.2 & 2.2 & 0.3 & 0.2 & 0.7 & 0.6 & 0 & 0 & 0 & 2.5 & 0.7 & 0.6 & 35 & 10.2 & 7.9 \\
70 & 34.1 & 33.1 & 39 & 10.4 & 34.7 & 522.3 & 43.9 & 2.7 & 55.8 & 112.3 & 135.3 & 70.3 & 0.1 & 289.9 & 172.2 & 24.3 & 14 & 16.4 & 63.6 & 163.8 & 126.7 & 32 & 37.6 & 10.1 & 0.3 & 4.1 & 0.4 & 0 & 0.5 & 1 & 1 & 0.3 & 0 & 2.5 & 1.3 & 1 & 38.1 & 11.3 & 8.4 \\
80 & 36.7 & 35.6 & 42.8 & 11.3 & 40.9 & 656 & 43.8 & 1.9 & 49.9 & 99.1 & 160.8 & 137.8 & 39.9 & 284.9 & 217.4 & 28.7 & 15.9 & 20.3 & 74.9 & 208.4 & 148.8 & 33.9 & 39.6 & 10.9 & 0.6 & 9.5 & 0.8 & 0 & 1 & 2 & 2.5 & 1.4 & 0 & 5 & 3 & 2.2 & 40.7 & 12.1 & 8.9 \\
90 & 39 & 37.9 & 46.3 & 12.1 & 47.4 & 799.5 & 41.7 & 1.5 & 45 & 86.4 & 154.6 & 166.4 & 159.4 & 282.4 & 266.3 & 33.5 & 17.8 & 24.5 & 86 & 256.7 & 171.4 & 37.1 & 43.7 & 12.1 & 0.4 & 6 & 0.3 & 0 & 0.5 & 0.9 & 1.3 & 1 & 0.8 & 2.5 & 1.9 & 1.3 & 43 & 12.7 & 9.3 \\[1em]
100 & 41 & 40 & 49.5 & 12.8 & 53.7 & 943.2 & 40.1 & 1.3 & 42 & 80.4 & 145.5 & 186.7 & 281.8 & 279 & 315.6 & 38.1 & 19.6 & 28.7 & 96.1 & 305.7 & 192.4 & 39.6 & 47.4 & 12.9 & 0.6 & 10.3 & 0.5 & 0 & 0.5 & 1.1 & 1.7 & 2.1 & 2.6 & 3.4 & 3.3 & 2.1 & 45 & 13.1 & 9.7 \\
110 & 42.9 & 41.8 & 52.5 & 13.4 & 60.1 & 1094.5 & 39.2 & 1.2 & 40.3 & 78 & 141.5 & 191.5 & 415.6 & 277.3 & 367.7 & 42.9 & 21.4 & 33.1 & 105.9 & 357.7 & 213.4 & 41.5 & 50.3 & 13.6 & 0.3 & 6.1 & 0.2 & 0 & 0.2 & 0.5 & 0.9 & 1.2 & 2 & 1.7 & 2 & 1.2 & 46.8 & 13.5 & 10 \\
120 & 44.5 & 43.6 & 55.5 & 14 & 51.8 & 975.1 & 29.9 & 0.8 & 29.9 & 59.3 & 107.5 & 148.2 & 436.5 & 214.3 & 328.9 & 37.3 & 18.1 & 29.3 & 89.8 & 321 & 182.3 & 42.3 & 54 & 13.5 & 14.4 & 264.4 & 8.8 & 0.2 & 8.8 & 17.5 & 31.6 & 43.6 & 109 & 63 & 86.7 & 51 & 48.4 & 12.4 & 10.2 \\
130 & 46 & 45 & 58.5 & 14.5 & 55.3 & 1066.5 & 28.3 & 0.7 & 27.3 & 56.3 & 102 & 141.1 & 536.4 & 205.9 & 361.1 & 40.3 & 19 & 31.9 & 94 & 353.6 & 192.7 & 44.8 & 58.4 & 14.3 & 2.2 & 43.1 & 1.1 & 0 & 1.1 & 2.3 & 4.1 & 5.7 & 21.7 & 8.4 & 14.3 & 7.8 & 49.9 & 11.6 & 10.3 \\
140 & 47.4 & 46.4 & 61.4 & 14.9 & 59.9 & 1180.4 & 27.2 & 0.6 & 25 & 54.7 & 99 & 136.6 & 647.6 & 202.5 & 401 & 44.2 & 20.3 & 35.2 & 99.8 & 393.7 & 206.7 & 46.8 & 62.4 & 15 & 1 & 20.3 & 0.5 & 0 & 0.4 & 0.9 & 1.6 & 2.3 & 11.4 & 3.4 & 6.8 & 3.5 & 51.3 & 11.6 & 10.4 \\[1em]
150 & 48.7 & 47.7 & 64.3 & 15.3 & 63.1 & 1266.4 & 25.7 & 0.5 & 22.6 & 51.9 & 93.9 & 129.1 & 743 & 194.1 & 431.7 & 47.2 & 21 & 37.6 & 102.8 & 425 & 215.3 & 47.2 & 60.7 & 15.5 & 2.4 & 48.5 & 1.2 & 0 & 1.2 & 2.3 & 4.3 & 5.9 & 25.8 & 8.4 & 16.2 & 8.3 & 52.5 & 11.7 & 10.5 \\
160 & 49.8 & 48.9 & 67.1 & 15.6 & 65.7 & 1341.3 & 24.2 & 0.5 & 20.8 & 49.4 & 89.2 & 122.3 & 826.9 & 185.7 & 458.7 & 49.8 & 21.6 & 39.6 & 104.8 & 452.6 & 221.9 & 48.9 & 65.8 & 15.8 & 2.8 & 58.4 & 1.1 & 0 & 1 & 2.2 & 4.1 & 5.6 & 35.2 & 8.4 & 19.7 & 9.6 & 53.7 & 11.6 & 10.6 \\
170 & 50.8 & 50 & 69.9 & 16 & 65.9 & 1364.1 & 22 & 0.4 & 18.2 & 45.1 & 81.4 & 111.2 & 877.6 & 171.4 & 467.9 & 50.6 & 21.4 & 40.2 & 102.7 & 462.7 & 220.2 & 50 & 67.5 & 16.3 & 5.1 & 106.3 & 1.9 & 0 & 1.7 & 3.8 & 7 & 9.6 & 65.9 & 14.3 & 35.9 & 17.1 & 54.8 & 11.2 & 10.6 \\
180 & 51.9 & 51 & 72.7 & 16.3 & 69.3 & 1456.4 & 21.1 & 0.3 & 17.3 & 43.7 & 78.6 & 107.2 & 968.9 & 167.2 & 501.1 & 54 & 22.3 & 42.8 & 105.8 & 496.4 & 229.6 & 51 & 71.4 & 16.5 & 1.7 & 34.9 & 0.5 & 0 & 0.4 & 1.1 & 2 & 2.7 & 22.7 & 4.2 & 11.9 & 5.5 & 55.8 & 11.1 & 10.6 \\
190 & 52.8 & 52 & 75.4 & 16.6 & 69.5 & 1477.2 & 19.1 & 0.3 & 15.6 & 40.2 & 71.7 & 97.8 & 1013.1 & 155.5 & 509.7 & 54.8 & 22.2 & 43.2 & 103.6 & 505.9 & 227.7 & 51.9 & 73.1 & 16.8 & 4.9 & 104.2 & 1.5 & 0 & 1.2 & 3.1 & 5.7 & 7.7 & 69.4 & 11.8 & 35.6 & 16.1 & 56.8 & 10.9 & 10.6 \\[1em]
200 & 53.5 & 52.9 & 78.2 & 16.9 & 67 & 1438 & 16.7 & 0.2 & 13.7 & 35.8 & 63.2 & 86 & 1011.7 & 139.5 & 497.6 & 53.6 & 21.2 & 42 & 97.6 & 494.6 & 217.3 & 53.1 & 76.3 & 17.2 & 7.3 & 157.8 & 2 & 0 & 1.6 & 4.1 & 7.4 & 10.1 & 109.3 & 16 & 54.2 & 23.7 & 57.7 & 10.4 & 10.6 \\
210 & 54.3 & 53.8 & 81 & 17.1 & 69.2 & 1499.6 & 15.7 & 0.2 & 13.1 & 34.2 & 59.9 & 81.5 & 1077.9 & 134.5 & 520.3 & 56.1 & 21.7 & 43.7 & 98.4 & 517.7 & 222.5 & 53.7 & 78.6 & 17.3 & 2.4 & 53.1 & 0.6 & 0 & 0.5 & 1.3 & 2.2 & 3.1 & 37.7 & 5 & 18.3 & 7.9 & 58.5 & 10.1 & 10.6 \\
220 & 55.1 & 54.7 & 84 & 17.4 & 69.4 & 1516.5 & 14.1 & 0.1 & 12.2 & 31.5 & 54.3 & 73.8 & 1113.7 & 125.2 & 527.7 & 57.1 & 21.6 & 44 & 96.1 & 525 & 221.5 & 53.6 & 78.1 & 17.3 & 4.3 & 94.6 & 1.1 & 0 & 0.9 & 2.4 & 4.2 & 5.8 & 66.2 & 9.2 & 32.5 & 14.1 & 59.3 & 9.8 & 10.6 \\
230 & 55.8 & 55.4 & 86.7 & 17.7 & 69.9 & 1539.3 & 12.8 & 0.1 & 11.6 & 29.4 & 49.9 & 67.7 & 1150.2 & 118.5 & 537 & 58.4 & 21.6 & 44.6 & 94.6 & 534.3 & 221.9 & 56.1 & 84.5 & 18.2 & 3.8 & 84.3 & 0.8 & 0 & 0.6 & 1.7 & 3 & 4.1 & 62 & 6.7 & 29.2 & 11.9 & 60.1 & 9.4 & 10.5 \\
240 & 56.5 & 56.2 & 89.5 & 17.9 & 71.4 & 1581.7 & 12 & 0.1 & 11 & 28 & 47.1 & 64 & 1198.4 & 113.4 & 553.2 & 60.4 & 21.9 & 45.8 & 94.2 & 550.4 & 225 & 55.8 & 85.1 & 18 & 2.9 & 63.7 & 0.6 & 0 & 0.5 & 1.3 & 2.2 & 3 & 47 & 5 & 22.1 & 9.1 & 60.8 & 9.4 & 10.5 \\[1em]
250 & 57.1 & 56.9 & 92.3 & 18.2 & 71.4 & 1593 & 11 & 0 & 10.4 & 26.1 & 43.5 & 59 & 1222.7 & 106.7 & 558.5 & 61.3 & 21.7 & 46 & 92 & 555.7 & 223.8 & 56.5 & 88.1 & 18.2 & 4.1 & 91.7 & 0.7 & 0 & 0.6 & 1.7 & 2.8 & 3.9 & 69.1 & 6.7 & 31.9 & 12.9 & 61.5 & 9.1 & 10.4 \\
260 & 57.8 & 57.6 & 95 & 18.4 & 74.5 & 1671.9 & 10.7 & 0 & 10.1 & 25.6 & 42.4 & 57.5 & 1296.7 & 105 & 587.6 & 64.7 & 22.5 & 48.2 & 93.6 & 584.7 & 231.9 & 56.8 & 86.1 & 18.5 & 1 & 22.1 & 0.2 & 0 & 0.2 & 0.4 & 0.7 & 1 & 16.5 & 1.7 & 7.7 & 3.1 & 62.2 & 8.9 & 10.4 \\
270 & 58.2 & 58.2 & 97.5 & 18.7 & 75.2 & 1698.1 & 10.2 & 0 & 9.7 & 24.5 & 40.4 & 54.8 & 1328 & 100.8 & 598.1 & 66 & 22.6 & 48.9 & 92.5 & 595.1 & 233 & 59.3 & 98.7 & 19.2 & 3.2 & 73.7 & 0.4 & 0 & 0.4 & 1 & 1.7 & 2.3 & 58 & 4.2 & 25.8 & 9.9 & 62.9 & 8.8 & 10.3 \\
280 & 58.8 & 58.8 & 100.1 & 18.9 & 75.3 & 1707.8 & 9.6 & 0 & 9.2 & 23.2 & 38.2 & 51.8 & 1346 & 95.8 & 602.9 & 66.9 & 22.5 & 49.1 & 90.5 & 599.9 & 232 & 59.3 & 97.6 & 19.2 & 3.8 & 86.5 & 0.5 & 0 & 0.5 & 1.2 & 2 & 2.7 & 67.9 & 5 & 30.3 & 11.6 & 63.5 & 8.5 & 10.2 \\
290 & 59.3 & 59.4 & 102.6 & 19.1 & 73.7 & 1677 & 8.9 & 0 & 8.6 & 21.6 & 35.5 & 48.1 & 1330.5 & 89.1 & 593.3 & 66.2 & 21.8 & 48.1 & 86.6 & 590.3 & 225.7 & 60 & 101.2 & 19.5 & 5.4 & 124.7 & 0.7 & 0 & 0.6 & 1.6 & 2.7 & 3.6 & 98.8 & 6.7 & 43.8 & 16.6 & 64.1 & 8.3 & 10.2 \\[1em]
300 & 59.9 & 60 & 105.4 & 19.4 & 71.9 & 1643.8 & 8.2 & 0 & 7.9 & 19.9 & 32.8 & 44.5 & 1313 & 82.4 & 582.9 & 65.4 & 21.2 & 47.1 & 82.4 & 580 & 219 & 59.3 & 101.3 & 19.1 & 5.4 & 123 & 0.7 & 0 & 0.6 & 1.6 & 2.7 & 3.6 & 97.3 & 6.7 & 43.2 & 16.6 & 64.7 & 8 & 10.1 \\
   \hline
\end{tabular}
\end{adjustbox}
\end{table}


\section{Diskussion}

Die Waldinventurdaten macht derzeit nur von Bäumen mit einem
BHD~>~5\,cm wiederholte Messungen. Damit basiert die geschätzte
Höhenentwicklung kleiner Bäume auf Extrapolation des Verlaufes der
größeren Bäume.

Die Beimischung anderer Baumarten kann die Oberhöhenbonität
verändern. Dennoch können hier weitere Faktoren die ermittelten
Koeffizienten verändern. Beispielsweise kann eine bestimmte Baumart
gehäuft auf besseren oder schlechteren Standorten beigemischt sein,
was zu einer Verschiebung der Bonität führt, aber nicht durch die
beigemischte Baumart verursacht wird. Auch ist es möglich, dass im
Zuge von Mischungsregulierungen auch die Oberhöhenstämme entnommen
werden, um der beigemischten Baumart den stärksten Konkurrenten
auszuschalten. Auch hier wird zwar die Oberhöhe reduziert aber nicht
unmittelbar durch die beigemischte Baumart.

Bei der Anwendung des Höhenzuwachsmodells ist nicht sichergestellt,
dass die Oberhöhenbäume dem vorgegebenen Oberhöhenfächer folgen. Daher
ist ein Korrekturfaktor zu bestimmen um hierin Übereinstimmung zu
schaffen.

Mit der Refferenzoberhöhe können zwar Behandlungsvarianten der
gleichen Baumart verglichen werden, ein direkter Vergleich zwischen
verschiedenen Baumarten ist damit allerdings nicht möglich da
verschiedene Baumarten unterschiedliche Höhenentwicklungen und auch
unterschiedliche Wuchsleistungen auf gleichen Standorten haben
können. Die Bestimmung der Bonität, für verschiedene Baumarten, über
den Standort, mag hier eine Lösung bieten, dieser ist aber Derzeit
noch mit größeren Zufallsstreuungen behaftet. Zur Darstellung
großräumiger Tendenzen scheint die Bonitierung über den Standort eine
gewisse Hilfestellung bei der waldbaulichen Entscheidung zu bieten,
für Entscheidungen auf Bestandesebene ist sie derzeit höchstens als
letzter Ausweg, wenn keine anderen Informationen zur Verfügung stehen,
zu betrachten.

\section*{Danksagung}

\emph{Gerhard Pelzmann} Möchte ich für die anregende Diskussion bei
der Modellerstellung als auch für die Aufstellung der finanziellen
Mittel, für ein Projekt in dem ein Waldwahstumsmodell entwickelt werden
konnte, danken. Für die Finanzierung möchte ich dem
\emph{Bundesministerium für Landwirtschaft, Regionen und Tourismus}
danken. Für die Hinweise, welche Kriterien für die Praxis relevant
sind, möchte in \emph{Günther Bronner} danken.

\addcontentsline{toc}{section}{Literatur:}
\bibliography{literature}

%Autor: Georg Kindermann

\end{document}

